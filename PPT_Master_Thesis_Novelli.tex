\documentclass{beamer}
\usepackage{lmodern}
\usepackage[english]{babel}
\usepackage{newlfont}
\usepackage{color}
\usepackage{multicol}
\usepackage{float}
\usepackage{frontespizio}
\usepackage{amsmath,amssymb}
\usepackage{amsthm}
\usepackage{geometry}
\usepackage{tikz}
\usepackage{biblatex}
\usepackage{csquotes}
\usepackage{pgfplots}
\usepackage{hyperref}
\usepackage{amssymb}
\usepackage{comment}
\usepackage[compat=1.0.0]{tikz-feynman}
\usepackage{tikz-cd}
\usepackage{mathtools}
\usepackage{braket}

\usetheme{Madrid}

\colorlet{beamer@blendedblue}{green!40!black}

\definecolor{NormalBlue}{RGB}{50,100,250}
\setbeamercolor{block title}{bg=NormalBlue}

\title{The geometry of the Yang-Mills theory}
\author{Novelli Gabriele}
\begin{document}
\begin{frame}[plain]
    \maketitle
\end{frame}
\begin{frame}{Introduction to Y-M}
	The Y-M fields are usually introduced by imposing a local $SU(N)$ symmetry on fermionic fields:
	$$\mathcal{L}_f=-\overline{\psi}(\gamma^\mu\partial_\mu+m)\psi$$
	where $\psi:M\rightarrow V^N\otimes \mathbb{C}^4$. By imposing $SU(N)$ local invariance one needs to promote $\partial_\mu\rightarrow D_\mu=\partial_\mu+A_\mu$.
	The kinetic part of the new field $A_\mu$ is calculated with $[D_\mu,D_\nu]=F_{\mu\nu}$.
	$$\mathcal{L}_{tot}=\mathcal{L}_f-{1\over 2}Tr(F_{\mu\nu} F^{\mu\nu})$$
\end{frame}
\begin{frame}{Introduction to Y-M}
	This works introduces Y-M theory like follows:
	\begin{itemize}
		\item Start from Maxwell in empty 
		$\mathbb{R}^{1,3}$;
		\item Recover Maxwell's theory as a Y-M $U(1)$ theory;
		\item Generalize the process to higher dimensional groups. 
	\end{itemize}
\end{frame}
\begin{frame}{The Cartan-Killing form}
	\begin{exampleblock}{Def: Cartan-Killing form}
		If $\mathfrak{g}$ is a Lie algebra, define:
		$$B_\mathfrak{g}(X,Y)=\hbox{Tr}(\hbox{ad}_X\circ \hbox{ad}_Y)$$
	\end{exampleblock}
	\begin{exampleblock}{Def: simple-semisimple Lie algebras}
		Let $\mathfrak{g}$ be a Lie algebra and $\mathfrak{h}\subset \mathfrak{g}$ a subspace. We say $\mathfrak{h}$ is an ideal if $[\mathfrak{g},\mathfrak{h}]\subset\mathfrak{h}$. \\
		We say $\mathfrak{g}$ is:
		\begin{itemize}
			\item \textit{simple} if it is not abelian and has no non trivial ideals;
			\item \textit{semisimple} if it has no non 0 abelina ideals.
		\end{itemize}
	\end{exampleblock}
\end{frame}
\begin{frame}{The Cartan-Killing form}
	\begin{block}{Result}
		\begin{itemize}
			\item The C-K form is invariant under automorphisms: $$B_\mathfrak{g}(X,Y)=B_\mathfrak{g}(\sigma X,\sigma Y)$$
			\item The C-K form of a compact semisimple $\mathfrak{g}$ is negative definite;
			\item If $\mathfrak{g}$ is simple and compact, $B_\mathfrak{g}$ is the only invariant negative definite product (up to a re-scaling).
			
		\end{itemize}
	\end{block}
\end{frame}
\begin{frame}{Fiber bundles}
	\begin{exampleblock}{Def: fiber bundle}
		Let $M, E$ and $F$ be manifolds. Let $\pi: E\rightarrow M$ be a map such that:
		\begin{itemize}
			\item $\pi$ is smooth, surjective and continuous;
			\item given any open set $U\subset M$ we can find a diffeomorphism $\varphi: \pi^{-1}(U)\rightarrow U\times F$ called \textit{local trivialization} such that $\pi=Proj\circ \varphi$.
		\end{itemize}
		Where $Proj: U\times F\rightarrow U$ is the standard projection. $Proj(x_M,x_F)=x_M$.\\
		We call $(E,M,\pi,F)$ a \textit{fiber bundle}.
	\end{exampleblock}
\end{frame}
\begin{frame}{Principal bundles}
	\begin{exampleblock}{Def: principal bundle}
		Let $(E,M,\pi,G)$ be a fiber bundle, where $G$ is a Lie group. Let $\pi$ be smooth and the trivializations diffeomorphic; let $G$ be acting on $E$ such that:
		\begin{itemize}
			\item $Stab(E)=\{e\in G\}$;
			\item Trivializations are equivariant: $\phi(p)\cdot g=\phi(p)\cdot g$ where $(x,h)\cdot g=(x,hg)$.
		\end{itemize} 
		Then we say that this fiber bundle is a \textit{principal $G$-bundle}.
	\end{exampleblock}
	\begin{block}{Idea}
		The local symmetry of gauge theories will be described by a principal $SU(n)$ bundle: a manifold that locally looks like $M\times SU(n)$.
	\end{block}
\end{frame}
\begin{frame}{Sections}
	\begin{exampleblock}{Def: section}
			Let $(E,M,\pi,\mathbb{R}^n)$ be a vector bundle. We say that a map \\$s:U\subset M\rightarrow E$ is a \textit{section} if $\pi\circ s=\mathbb{I}_d$. A section of a principal bundle is called \textit{local/global gauge}.
	\end{exampleblock}
	\begin{block}{Result}
		Two gauges defined on the same open set are related by a smooth map $g_{21}:U\rightarrow G$ like: $s_1(x)=s_2(x)\cdot g_{21}(x)$
	\end{block}
	\begin{alertblock}{Key concept}
		Gauge transformations$\Longrightarrow$change of sections.
	\end{alertblock}
\end{frame}
\begin{frame}{Fundamental vector fields}
	\begin{exampleblock}{Def: fundamental vector field}
		Let $G$ be acting on $M$ and $A\in\mathfrak{g}$, define the \textit{fundamental vector field associated to} $A$:
		$$\overline{A}_p={d\over dt}\bigg|_0 \mu(p,e^{tA})$$
	\end{exampleblock}
	\begin{block}{Result}
		If $j_p(g)=p\cdot g$, then $dj_p(A)=\overline{A}_p$
	\end{block}
\end{frame}
\begin{frame}{Distributions}
	\begin{exampleblock}{Def: distribution}
			A \textit{distribution} on a manifold $M$ is an assignment $M\ni p\rightarrow \mathcal{D}_p\subset T_pM$.
			\begin{itemize}
				\item canonical vertical distribution on a principal bundle: $\mathcal{V}=Ker(d\pi)$.
			\end{itemize}
	\end{exampleblock}
	\begin{alertblock}{Note}
		There is no canonical horizontal distribution $TE=\mathcal{V}\oplus\mathcal{H}$.
	\end{alertblock}
	\begin{block}{Result}
		$(dj_p)_e:\mathfrak{g}\rightarrow \mathcal{V}_p$ is an isomorphism.
	\end{block}
\end{frame}
\begin{frame}{Connection}
	\begin{exampleblock}{Def: connection}
			We call an \textit{Ehresmann connection on a principal bundle} a 1-form $\omega:TE\rightarrow \mathfrak{g}$ such that:
		\begin{itemize}
			\item $\forall A\in\mathfrak{g}, x\in E$ we have $\omega_{x}(\bar{A}_{x})=A$;
			\item $\forall g\in G$ we have $r^*_g\omega=\hbox{Ad}(g^{-1})\omega$;
			\item $\omega$ is $C^\infty$.
		\end{itemize}
	\end{exampleblock}
	\begin{alertblock}{Theorem}
		If $\omega$ is a connection then $Ker(\omega)=\mathcal{H}$ is an horizontal distribution. If instead $\mathcal{H}$ is an horizontal distribution then $\omega=dj^{-1}\circ \nu$ is a connection.
	\end{alertblock}
\end{frame}
\begin{frame}{Curvature}
	\begin{exampleblock}{Def: curvature}
		Let $(E,M,\pi,G)$ be a principal bundle and $\omega$ an Ehresmann connection. The \textit{curvature} of $\omega$ is:
		$$\Omega=d\omega+{1\over 2}[\omega,\omega]$$
		Where $[\omega,\omega]=\omega^i\wedge\omega^j[T_i,T_j]$.
	\end{exampleblock}
	\begin{block}{Result}
		$d\Omega=[\Omega,\omega]$; $\Omega$ is horizontal;$r^*_g\Omega=Ad(g^{-1})\Omega$.
	\end{block}		
	\begin{alertblock}{Note}
		If $G$ is abelian then $Ad$ is trivial and $\Omega=d\omega$.
	\end{alertblock}
\end{frame}
\begin{frame}{Associated bundles}
	If $\rho:G\rightarrow GL(V)$ is a representation and $(P,M,\pi,G)$ a principal bundle, define the quotient $E=P\times_\rho V$
	under the relation:
	$$(p,v)\sim(p\cdot g, \rho(g^{-1})v)$$
	\begin{exampleblock}{Def: associated bundle}
			Let $E=P\times_\rho V$, $\pi_E([p,g])=\pi(p)$. Then  $(E,M,\pi_E,V)$ is called associated bundle.
	\end{exampleblock}
		We indicate the associated bundle induced by the adjoint representation as $Ad(P)$.
		$$(Ad(P),M,\pi_{Ad},\mathfrak{g})$$
\end{frame}
\begin{frame}{Tensorial forms}
	\begin{exampleblock}{Def: tensorial forms}
		Let $(P,M,\pi,G)$ be a principal bundle, $\rho$ a representation of $G$. Then any $\omega\in\Omega^k(P,V)$ is said to be:
		\begin{itemize}
			\item \textit{right-equivariant of type $\rho$} if:
			$r^*_g\omega=\rho(g^{-1})\omega$
			\item \textit{horizontal} if it vanishes whenever one of its arguments is a vertical vector. 
		\end{itemize} If a form is both right-equivariant of type $\rho$ and horizontal it is called \textit{tensorial of type $\rho$}. We label the space of all smooth tensorial $k$-forms of type $\rho$ with $\Omega^k_\rho(P,V)$.
	\end{exampleblock}
	\begin{block}{Result}
		The curvature $\Omega$ of a connection $\omega$ is tensorial of type $Ad$.
	\end{block}
\end{frame}
\begin{frame}{Musical Isomorphism}
	\begin{alertblock}{Musical Isomorphism}
		$$\Omega^k_\rho(P,V)\simeq\Omega^k(M,E)$$
	\end{alertblock}
	To any tensorial form on $P$ we can associate a form on $M$ with values on $E$.
	Let $F\in\Omega^k(P,V)$, $x\in M$, $p\in P_x$, $\{v_i\}\in T_xM$ and $\{u_i\}\in T_pP$ lifts, i.e. $d\pi(u_i)=v_i$. Then:
	$$F^\flat_x(v_1,...,v_k)=[p,F_p(u_1,...,u_k)]$$
	\begin{alertblock}{Key idea}
		If $F\in\Omega^2_{Ad}(P,\mathfrak{g})$ is the curvature of $A$, there is $F_M\in\Omega^2(M,Ad(P))$.
	\end{alertblock}
\end{frame}
\begin{frame}{Covariant derivative}
	\begin{block}{Obs}
		The exterior derivative preserves the right equivariance, but not the horizontality.
	\end{block}
	\begin{exampleblock}{Def: horizontal component of a form}
		$$(F_p)^h(u_1,..,u_k)=F_p(\hbox{Hor}(v_1),...,\hbox{Hor}(v_k))$$
	\end{exampleblock}
	\begin{exampleblock}{Def: covariant derivative}
		If $F\in\Omega^k_\rho(P,V)$, define the covariant derivative as $DF:=(dF)^h$.
	\end{exampleblock}
	\begin{alertblock}{Prop}
		$$D:\Omega^k_\rho(P,V)\rightarrow \Omega^{k+1}_\rho(P,V)$$
	\end{alertblock}
\end{frame}
\begin{frame}{Covariant derivative}
	\begin{block}{Result}
		It is possible topo show that if $A$ is a connection and $F$ is it's curvaure then $F=DA$.
	\end{block}
	\begin{alertblock}{Theorem}
		Let $A$ be a connection. For tensorial forms it holds:
		$$D\omega=d\omega+A\cdot\omega $$
		where in general:
		$$\omega\cdot \tau(v_1,...,v_{k+l})={1\over k!l!}\sum_\sigma sgn(\sigma)d\rho(\omega(v_{\sigma(1)},...,v_{\sigma(k)}))\tau(v_{\sigma(k+1),...,v_{\sigma(k+l)}})$$
	\end{alertblock}
\end{frame}
\begin{frame}{Covariant derivative}
	\begin{exampleblock}{Obs}
		If $F$ is the curvature of a connection $A$. We found $dF=[F,A]$ so that:
		$$DF=dF+[A,F]=0$$
	\end{exampleblock}
	\begin{exampleblock}{Obs}
		Since $F\in\Omega^k_{Ad}(P,\mathfrak{g})$ can be associated to $F_M\in\Omega^k(M,Ad(P))$, the covariant derivative extends to:
		$$D:\Omega^k(M,Ad(P))\rightarrow \Omega^{k+1}(M,Ad(P))$$
	\end{exampleblock}
	Note that in general $D^2\neq 0$.
\end{frame}
\begin{frame}{Gauge transformations}
	\begin{exampleblock}{Def: Gauge transformation}
		Let $(P,M,\pi,G)$ be a principal $G$-bundle. We call a \textit{gauge transformation} any bundle automorphism, i.e a diffeomorphism $f:P\rightarrow P$ such that:
		\begin{itemize}
			\item $f$ is fiber preserving: $\pi\circ f=\pi$;
			\item $f$ is $G$-equivariant: $f(p\cdot g)=f(p)\cdot g$ for any $p\in P, g\in G$.
		\end{itemize}
		The set of all gauge transformations is called $\mathcal{G}(P)$.
	\end{exampleblock}
	\begin{alertblock}{Theorem}
		Let $C^\infty(P,G)^G=\{\sigma:P\rightarrow G|\hbox{ }\sigma\hbox{ is smooth },\sigma\circ r_g=c_{g^{-1}}\circ \sigma\}$.
		Then there is an isomorphism:
		$$\mathcal{G}(P)\longrightarrow C^\infty(P,G)^G\hbox{ like }f\rightarrow \sigma_f\hbox{, where } f(p)=p\cdot \sigma_f(p)$$
	\end{alertblock}
\end{frame}
\begin{frame}{Gauge transformations}
	\begin{alertblock}{Theorem}
		Let $(P,M,\pi,G)$ be a principal bundle and $s:U\rightarrow P$ a smooth local gauge. Then there is an isomorphism:
		$$C^\infty(P|_U,G)^G\longrightarrow C^\infty(U,G)\hbox{ such that: }\sigma\rightarrow \sigma\circ s$$
	\end{alertblock}
	\begin{block}{Result}
		Recall that if $s_{1,2}$ are two gauges defined on the same open set then: $s_1(x)=s_2(x)\cdot g_{21}(x)$, with $g_{21}\in C^\infty(U,G)$.
	\end{block}
	\begin{alertblock}{Key result}
		Gauge transformations correspond to changes in local gauges.
	\end{alertblock}
\end{frame}
\begin{frame}{Gauge transformations}
	One can look at how gauge transformations act on the associated bundle:
	\begin{alertblock}{Prop}
		There is a natural action of gauge transformations on the associated bundle:
		$$\mathcal{G}(P)\times E\rightarrow E$$
		$$(f,[p,v])\rightarrow [f(p),v]$$
	\end{alertblock}
\end{frame}
\begin{frame}{Gauge transformations}
	Given a form $\omega\in\Omega^k(P,V)$ and a local gauge $s:U\rightarrow P$, we can pull back the form on the manifold: $s^*\omega \in\Omega(U,P)$
	$$s^*\omega(X)=\omega(ds(X))$$
	\begin{block}{Result}
		If $s_{i,j}$ are two sections and $A$ is a connection, then:
		$$s_i^*A=Ad(g_{ji}^{-1})s^*_jA+\mu_{ji} \hbox{ with } \mu_{ji}=g^*_{ji}\theta$$
	\end{block}
	Can be shown using the differential of the group action:
	$$s_i=s_j\cdot g_{ji}=\mu(s_j,g_{ji})$$
	$$d\mu_{p,g}(X_p,d\ell_gA)=r_g^*X_p+\overline{A}_{p\cdot g}$$
\end{frame}
\begin{frame}{Gauge transformations}
	\begin{alertblock}{Note}
		For a matrix lie group $\mu_{ji}=g_{ji}^{-1}dg_{ji}$
	\end{alertblock}
	\begin{block}{Result}
		The gauge transformation of the curvature is:
		$$s_i^*F\rightarrow Ad(g_{ji}^{-1})s_j^*F$$
	\end{block}
	\begin{exampleblock}{Obs}
		If $G$ is abelian, $Ad$ is trivial and so $F$ is gauge invariant i.e. it does not depend on the choice of the section.
	\end{exampleblock}
\end{frame}
\begin{frame}{Gauge transformations}
	Recall that there is an associated form $F_M\in\Omega^2(M,Ad(P))$.
	\begin{block}{Result}
		Under a gauge transformation $F_M\rightarrow \phi^{-1}\cdot F_M$.
		\\\vspace{9 pt}
		Let $s:U\rightarrow P$ be a local gauge and $\phi\in \mathcal{G}(P)$ and $g\in C^\infty(U,G)$ the corresponding smooth function. Since $F_{M,x}(X_x,Y_x)=[s(x),F_{s(x)}(dx(X_x),dx(Y_x))]$, from this it follows:
		$$\phi^{-1}\cdot F_{M,x}(X_x,Y_x)=[s(x)\cdot g^{-1},F_{s(x)}(dx(X_x),dx(Y_x))]=$$$$[s(x)\cdot ,Ad(g^{-1})F_{s(x)}(dx(X_x),dx(Y_x))]$$
	\end{block}
\end{frame}
\begin{frame}{Bundle metrics}
	We wish for a metric on the associated bundle $E$ induced by $\rho:G\rightarrow GL(V)$.
	\begin{exampleblock}{Obs}
		If $V$ has a $G$ invariant product $\braket{,}_V$, then we define:
		$$\braket{[p,v],[q,w]}_{E_x}=\braket{v,w}_V$$
	\end{exampleblock}
	\begin{alertblock}{Note}
		In the case of a compact and semisimple Lie algebra $\mathfrak{g}$, the scalar product on $Ad(P)$ is the Cartan-Killing form, which is automatically gauge invariant.
		$$\braket{[p,v],[p,w]}_{Ad}=-Tr(\hbox{ad}_v\circ \hbox{ad}_w)$$
	\end{alertblock}
\end{frame}
\begin{frame}{Bundle metrics}
	\begin{block}{Result}
		Let $s:U\rightarrow P$ be a local gauge, $F$ be the curvature of $A$. Then if the bundle metric is gauge invariant:
		$$\braket{F_M,F_M}_{Ad}=\braket{s^*F,s^*F}_\mathfrak{g}$$
	\end{block}
	By gauge invaraince, the choice of the section does not matter!
	\\
	$$\braket{F_M,F_M}_{Ad}=\braket{[s,s^*F],[s,s^*F]}_{Ad}=\braket{s^*F,s^*F}_\mathfrak{g}$$
\end{frame}
\begin{frame}{Bundle metrics}
	\begin{block}{Result}
		If $s:U\rightarrow P$ is a gauge and $\varphi:\pi_{Ad}^{-1}(U)\rightarrow U\times V$ is the induced local trivialization, then:
		$$F_{M,x}=\varphi^{-1}(x,s^*F_{s(x)})$$
	\end{block}
	By construction $\varphi^{-1}(x,v)=[s(x),v]$ so that:
	$$\varphi^{-1}(x,s^*F_{s(x)})=[s(x),s^*F_{s(x)}]=F_{M,x}$$
\end{frame}
\begin{frame}{Bundle metrics}
	\begin{exampleblock}{Def: Hodge operator}
		Let $\omega\in\Omega^k(M,E)$ a form with values on an associated bundle and $\{e_i\}$ a basis of sections. Define $\star$ like:
		$$\star\omega=\star\omega^i\otimes e_i$$
	\end{exampleblock}
	\begin{exampleblock}{Def: co-differential}
		If $D:\Omega^k(M,E)\rightarrow\Omega^{k+1}(M,E)$ is the covariant derivative on the associated bundle, we define the co-differential as:
		$$D^*=(-)^{k(n-k)}sgn(\braket{,}_E)\star\circ D\circ\star$$
	\end{exampleblock}
\end{frame}
\begin{frame}{Maxwell's equations}
	\begin{exampleblock}{Obs}
		In empty $\mathbb{R}^{1,3}$ Maxwell's equations are:
		$$dF=0\hspace{20 pt} d^*F=0$$ 
		$F\in\Omega^2(\mathbb{R}^{1,3},\mathbb{R})$ is called \textit{field strength}.
		$$\hbox{Poincaré Lemma}\Longrightarrow F=dA$$
	\end{exampleblock}
	\begin{block}{In local coordinates}
		$$\partial_{[\rho} F_{\mu\nu]}=0\hspace{20 pt} \nabla_\mu F^{\mu\nu}=0$$
	\end{block}
\end{frame}
\begin{frame}{Recovering Maxwell}
	\begin{exampleblock}{Obs}
		The Maxwell (empty-flat) theory can be recovered as a $U(1)$ Yang-Mills theory with the following assumptions:
		\begin{itemize}
			\item Spacetime is $\mathbb{R}^{1,3}$;
			\item There is a principal bundle $(P,\mathbb{R}^{1,3},\pi,U(1))$;
		\end{itemize}
	\end{exampleblock}
	In local coordinates: $$s^*A=A_{\mu}dx^\mu\otimes T$$
	where $\{T\}$ is the generator of $\mathfrak{u}(1)$, usually $T=-i$.
\end{frame}
\begin{frame}
	\begin{block}{Result}
		The connection $A$ plays the role of the vector potential, as it transforms like:
		$$s^*A\rightarrow s^*A+d\alpha$$
	\end{block}
	To see this recall:
	$$s^*A\rightarrow Ad(g^{-1})s^*A+g^{-1}dg$$
	$g(x)=e^{\alpha(x) T}$ implies $g^{-1}dg=\partial_\mu \alpha(x)dx^\mu\otimes T$.
	\begin{alertblock}{Note}
		$d(exp)_X=\sum_{n=0}^{\infty}{(-1)^n\over (n+1)!}(\hbox{ad}_X)^n$ in the general non abelian case.
	\end{alertblock}
\end{frame}
\begin{frame}{Recovering Maxwell}
	\begin{exampleblock}{Obs}
		The curvature is $F=DA=dA+{1\over 2}[A,A]$ and it holds $DF=0;$\\
		$Ad$ is trivial$\Longrightarrow$ 
		$\begin{cases}
			DA=dA;\\
			DF=dF=0;\\
			Ad(P)\hbox{ is trivial;}\\  
			F \hbox{ is gauge invariant.}\\
		\end{cases}$
	\end{exampleblock}
	\begin{block}{In local coordinates}
		$$s^*F={1\over 2}F_{\mu\nu}dx^\mu\wedge dx^\nu\otimes T\hspace{20 pt} F_{\mu\nu}=\partial_\mu A_\nu-\partial_\nu A_\mu$$
	\end{block}
	We can also construct $F_M\in\Omega^2(\mathbb{R}^{1,3},Ad(P))$, from which we will define the lagrangian.
\end{frame}
\begin{frame}{Recovering Maxwell}
	\begin{block}{Product}
		The Lie algebra of $U(1)$ is 1 dimensional. Suppose $\braket{,}_{\mathfrak{u}(1)}$ is a scalar product on it.
	\end{block}
	\begin{exampleblock}{Def}
		$$\mathcal{L}=-{1\over 2}\braket{F_M,F_M}_{Ad}$$
	\end{exampleblock}
	Choosing any gauge $s:U\rightarrow P$ we get in local coordinates:
	$$\mathcal{L}=-{1\over 2}\braket{s^*F,s^*F}_{\mathfrak{u}(1)}=-{1\over 4}F_{\mu\nu}F^{\mu\nu}$$
	This is clearly gauge invariant.
\end{frame}
\begin{frame}{Generalizing Maxwell}
	\begin{exampleblock}{Assumptions}
	We make the following assumptions:
	\begin{itemize}
		\item Spacetime is $\mathbb{R}^{1,3}$;
		\item There is a principal bundle $(P,\mathbb{R}^{1,3},\pi,SU(N))$;
	\end{itemize}
	In general the theory works well with any simple/semisimple compact Lie algebra.
\end{exampleblock}
Let $A$ be a connection, then, if $s:U\rightarrow P$ is a local guage, in local coordinates:
$$s^*A=-iA_{\mu,a} dx^\mu\otimes T^a$$
Where $\{-iT^a\}$ are the generators of $\mathfrak{su}(n)$ like: $$[T^a,T^b]=if^{ab}_{\hspace{9pt}c}T^c$$
\end{frame}
\begin{frame}{Generalizing Maxwell}
	\begin{block}{Result}
		The gauge transformation of the connection in the $SU(n)$ case is:
		$$s^*A\rightarrow Ad(g^{-1})s^*A+g^{-1}dg$$
	\end{block}
	\begin{block}{Result}
		The gauge transformation of the curvature in the $SU(n)$ case is:
		$$s^*F\rightarrow Ad(g^{-1})s^*F$$
	\end{block}
	\begin{exampleblock}{Obs}
		$F=dA+{1\over 2}[A,A]$ and since $Ad$ is not trivial, new term arise:
		$$s^*F={-i\over 2}[\partial_{\mu}A_{\nu,a}-\partial_{\nu}A_{\mu,a}+f^{bc}_{\hspace{9pt}a}A_{\mu,b}A_{\nu,c}]dx^\mu\wedge dx^\nu \otimes T^a$$
	\end{exampleblock}
\end{frame}
\begin{frame}{Generalizing Maxwell}
	\begin{exampleblock}{Obs}
		By generalizing the previous lagrangian:
		$$\mathcal{L}=-{1\over 2}\braket{F_M,F_M}_{Ad}$$
	\end{exampleblock}
	In an orthonormal base for the C-K form:
	$$\mathcal{L}=-{1\over 4}F_{\mu\nu,a}F^{\mu\nu,a}$$
	\begin{block}{Result}
		$$DF_M=0\hspace{20 pt} D^*F_M=0$$
	\end{block}
\end{frame}
\end{document}
