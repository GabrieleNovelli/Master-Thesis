\documentclass[12pt,a4paper]{report}

\usepackage[english]{babel}
\usepackage{newlfont}
\usepackage{color}
\usepackage{multicol}
\usepackage{float}
\usepackage{frontespizio}
\usepackage{amsmath,amssymb}
\usepackage{amsthm}
\usepackage{geometry}
\usepackage{tikz}
\usepackage{biblatex}
\usepackage{csquotes}
\usepackage{pgfplots}
\usepackage{hyperref}
\usepackage{amssymb}
\usepackage{comment}
\usepackage[compat=1.0.0]{tikz-feynman}
\usepackage{tikz-cd}
\usepackage{mathtools}
\usepackage{braket}

\hypersetup{
	colorlinks=true,
	linkcolor=blue,
	filecolor=magenta,      
	urlcolor=cyan,
	pdftitle={Overleaf Example},
	pdfpagemode=FullScreen,
}

\textwidth=450pt\oddsidemargin=0pt
\geometry{a4paper, top=3cm, bottom=3cm, left=3cm, right=3cm, % heightrounded, bindingoffset=5mm 
}
\theoremstyle{definition}
\newtheorem{Def}{Definition}[chapter]

\theoremstyle{Theorem}
\newtheorem{Theo}[Def]{Theorem}
\newtheorem{Prop}[Def]{Proposition}

\newtheorem{Lm}[Def]{Lemma}

\theoremstyle{definition}
\newtheorem{Ex}[Def]{Example}

\theoremstyle{definition}
\newtheorem{Obs}[Def]{Observation}
\begin{document}
	\tableofcontents
	\chapter{Preliminaries on differential geometry}
	In this chapter we will introduce the topic of fiber bundles vector bundles and some basic notions that will be used afterwards. These concepts will be fundamental for the study of Gauge Theories. In what follows, we will assume that the reader has familiarity with basic notions of smooth manifold and Lie groups. For these concepts, the reader is advised to consult [1] chap. 1, 2, 3, 4 and [2] chap. 1,2.
	\section{Smooth manifolds}
	In this section we introduce the notion of smooth manifold, tangent spaces and vector fields; with some examples. More details can be found in [1] chap. 1 pag. 48-57.
	\begin{Def}
		A \textit{topological manifold} $M$ of dimension $n$ T2, second countable, locally euclidean topological space of dimension $n$. With locally euclidean we mean that for every point $p\in M$ there exists an open subset $U\subset M$ containing $p$, and an homeomprphism $\phi:U\rightarrow\mathbb{R}^n$. The pair $(U,\phi)$ is called \textit{chart}. 
	\end{Def}
	\begin{Ex}
		The space $\mathbb{R}^n$ with $(\mathbb{R}^n, id)$ where $id:\mathbb{R}^n\rightarrow \mathbb{R}^n$ is the identity map, is a topological manifold. 
	\end{Ex}
	\begin{Def}
		Two charts $(U_1,\phi:U_1\rightarrow\mathbb{R}^n)$ and $(U_2,\varphi:U_2\rightarrow\mathbb{R}^n)$ on the same topological manifold $M$ are said to be \textit{compatible} if
		$\phi\circ\varphi^{-1}:\varphi(U_1\cap U_2)\rightarrow \phi(U_1\cap U_2)$ and $\varphi\circ\phi^{-1}:\phi(U_1\cap U_2)\rightarrow \varphi(U_1\cap U_2)$ are $C^\infty$.\\
		A collection of compatible charts $\mathbb{U}=\{(U_i,\phi_{i})\}$ on $M$ such that $M=\bigcup_i U_i$ is called \textit{atlas}. A topological manifold endowed with a maximal atlas is called \textit{smooth manifold}.
	\end{Def}
	It is possible to show that if two charts are compatible with some other charts of a given atlas, then they are also compatible with one another. For more details see [1] (pag. 51, cap. 2).
	\begin{Obs}\label{Obs:1.1.1}
		Every open subset of a smooth manifold is still a smooth manifold. In fact if $\{(U_i,\phi_i)\}$ is an atlas for $M$, considering an open set $A\subset M$ the collection $\{(U_i\cap A,\phi_i|_{U_i\cap A})\}$ is an atlas for $A$.
	\end{Obs}
	We now give some important examples of smooth manifolds:
	\begin{Ex}
		The set $\mathbb{R}^n$ with the chart $(\mathbb{R}^n,\phi)$, where $\phi=(r^1,...,r^n)$ and $r^i$ are the standard coordinate of $\mathbb{R}^n$, is a smooth manifold.
	\end{Ex}
	\begin{Ex}\label{Ex 1.1}
		The set $GL_n(\mathbb{R})=\{A\in M_{n\times n}|\det(A)\neq0\}$ si a smooth manifold. One can see this by considering the map $\det:\mathbb{R}^{n^2}\rightarrow \mathbb{R}$, by definition $GL_n(\mathbb{R})=\det^{-1}(\mathbb{R}-\{0\})$. Since the map $\det$ is continuous, the pre-images of open sets are open as well, thus $GL_n(\mathbb{R})$ is an open subset of $\mathbb{R}^{n^2}$. By observation \ref{Obs:1.1.1}, $GL_n(\mathbb{R})$ is a smooth manifold.
	\end{Ex}
	\begin{Ex}
		Consider the unit circle $S^1=\{x^2+y^2=1\}\subset \mathbb{R}^2$. Consider two charts: $$(U_1=\{x^2+y^2=1;y>0\},\phi_1) \, and \,  (U_2=\{x^2+y^2=1;y<0\},\phi_2)$$ like in figure \ref{figura 1}, where the coordinate maps are defined like: $\phi_1(x,y)=x$ e $\phi_2(x,y)=x$.
		\begin{figure}[H]
			\centering
			\begin{tikzpicture}
				\draw[->] (-2,0) -- (2,0) node[anchor=north west] {$x$};
				\draw[->] (0,-2.) -- (0,2) node[anchor=south east] {$y$};
				\draw[thick] (0,0) circle (1cm);
				\draw (0.4,1.5) node{$U_1$};
				\draw (-0.4,-1.5) node{$U_2$};
				\draw[thick, ->] (0,1)--(0,0.2) ;
				\draw (0.5,0.5) node{$\phi_1$};
				\draw (-0.5,-0.5) node{$\phi_2$};
				\draw[thick, ->] (0,-1)--(0,-0.2);
			\end{tikzpicture}
			\label{figura 1}
			\caption{The two charts $U_1$ and $U_2$ on the unit circle.}
		\end{figure}
		To those charts we add analogously $$(U_3=\{x^2+y^2=1;x>0\},\phi_3)\, and \, (U_4=\{x^2+y^2=1;x<0\},\phi_4)$$ with $\phi_3(x,y)=y$ and $\phi_4(x,y)=y$. By construction, $\phi_i$ is an homeomorphism for every $i$. It remains to show that the charts are compatible.\\
		Consider the composition $\phi_3\circ\phi_2^{-1}$, this is such that: $$(\phi_3\circ\phi_2^{-1})(x)=\phi_3(x,-\sqrt{1-x^2})=-\sqrt{1-x^2}$$ for $x\in ]0,1[$, which is $C^\infty$. The compatibility of the remaining charts follows by an analogous proof. This proves the circle is a smooth manifold.
	\end{Ex}
	\begin{Def}
		A subset $S\subset M$ of a manifold $M$ is called \textit{regular} or \textit{embedded submanifold} of dimension $k$ if for every $p\in M$ there exists a chart $(U,\phi)$ centered in $p$ such that $U\cap S$ is defined by the vanishing of $n-k$ coordinates.
	\end{Def}
	Thus if $\phi=(x^1,...,x^n)$ is a coordinate map on $M$, then on $U\cap S$ we will have $\phi=(x^1,...,x^k,0,0,...0)$.
	\begin{Ex}
		Consider the smooth manifold $\mathbb{R}^n$ and the space $\mathbb{R}^k\subset\mathbb{R}^n$ for $k<n$. Consider also the chart $(\mathbb{R}^n,\phi)=(\mathbb{R}^n,r^1,...,r^n)$ centered in $p$. Since $\mathbb{R}^n\cap \mathbb{R}^k=\mathbb{R}^k$ it immediately follows that $\phi|_{\mathbb{R}^k}=(r^1,...,r^k,0,0,...0)$. This shows that $\mathbb{R}^k$ is a regular submanifold of $\mathbb{R}^n$.
	\end{Ex}
	\begin{Obs} \label{Obs:1.1.2}
		In the definition of regular submanifold, the dimension of $S$ can coincide with the one of $M$. In this case $U\cap S=U$. By observation \ref{Obs:1.1.1}, every open subset $S$ of $M$ is a regular smooth submanifold of dimension equal to $M$.
	\end{Obs}
	\section{Differentiable maps}
	In this section we define smooth maps between smooth manifolds and describe some important properties they enjoy. More details can be found in [1] cap. 2 pag. 59-70.
	\begin{Def}
		Let $M$ be a smooth manifold and $f:M\rightarrow\mathbb{R}$ a map. Then $f$ is said to be \textit{smooth} or $C^\infty$ in $p\in M$ if there exists a chart $(U,\phi)$ centered in $p$ such that $f\circ \phi^{-1}$ is $C^\infty$. The map $f:M\rightarrow \mathbb{R}$ is said to be $C^\infty$ on $M$ if it is $C^\infty$ on every $p\in M$.
	\end{Def}
	\begin{Obs}
		The definition of smoothness we gave does not depend on the choice of the chart. In fact if $(U,\phi)$ and $(V,\psi)$ are two charts of $M$ and $f\circ\phi^{-1}$ is $C^\infty$, then $$f\circ\psi^{-1}=(f\circ\phi^{-1})\circ(\phi\circ\psi^{-1})$$ which is still $C^\infty$.
	\end{Obs}
	\begin{Obs}
		If $f:M\rightarrow \mathbb{R}$ is $C^\infty$ then it is continuous. One can in fact write $f=(f\circ\phi^{-1})\circ \phi$ where $\phi$ and $f\circ\phi^{-1}$ are continuous: $f\circ\phi^{-1}$ is $C^\infty$ and $\phi$ is an homeomprphism. It follows that, by composition of continuous maps, that $f$ is continuous.\\
	\end{Obs}
	\begin{Def}
		Let $N$ and $M$ be smooth manifolds of dimensions $n$ and $m$. A continuous map $f:N\rightarrow M$ is said to be $C^\infty$ in $p\in N$ if there are two charts $(U,\phi)$ centered in $p\in N$ and $(V,\psi)$ centered in $f(p)\in M$ such that $\psi\circ f\circ \phi$ is $C^\infty$.
	\end{Def}
	The composition has as domain $\phi(f^{-1}(V))\cap U$ subset of $\mathbb{R}^n$. $$\psi\circ f\circ \phi:\phi(f^{-1}(V)\cap U)\rightarrow \mathbb{R}^m$$
	The continuity of $f$ is requested to ensure that the pre-image of $f^{-1}(V)$ is an open subset of $N$. It is also possible to show that the composition of $C^\infty$ maps between smooth manifolds is still $C^\infty$. The proof can be found in [1] (pag. 62 cap.2). 
	\begin{Def}
		A map $f:M\rightarrow N$ between two smooth manifolds is called \textit{diffeomorphism} if it is bijective, $C^\infty$ and with a $C^\infty$ inverse.
	\end{Def}
	It is also possible to show that all coordinate maps of any given chart $(U,\phi)$ of a smooth manifold are diffeomorphisms. The proof can be found in [1] pag. 63 cap. 2.
	\section{Tangent space and vector fields}
	In this section we define the notions tangent space and vector fields. From now on we will indicate with $M$ a generic smooth manifold of dimension $n$. More informations can be found in [1] chap. 3 pag. 86-98.
	\begin{Def}
		Consider all pairs $(f,U)$, where $U\subset M$ is an open set containing $p\in M$ and $f:U\rightarrow \mathbb{R}$ is a $C^\infty$ map. We say that $(f,U)\sim(g,V)$ if there exists an open set $W\subset U\cap V$ containing $p$, such that $f=g$ when restricted to $W$. We define the \textit{germ} of $f$ in $p$ as the equivalence class of $(f,U)$.
		The set of all germs of $C^\infty$ functions at $p\in M$ is labeled with $C^\infty_p(M)$.
	\end{Def}
	It is not difficult to verify that the so defined relation is an equivalence relation: if $f\sim g$ then obviously $g\sim f$ at $p$. Moreover we clearly have that $f\sim f$ and if $f\sim g$, $g\sim h$, then $f\sim h$ since all of the above functions are equal in a neighborhood of $p$.\\
	\\
	By generalizing the concept of derivation in $\mathbb{R}^n$, we call \textit{derivation} in $p\in M$ any linear map $D_p:C^\infty_p(M)\rightarrow\mathbb{R}$ which respects the Leibniz rule:
	$$D_p(fg)=D_p(f)g(p)+f(p)D_p(g)$$
	\begin{Def}
		A derivation in $p\in M$ is called tangent vector in $p$. The set of all tangent vectors in $p$ is called \textit{tangent space} and will be referred to as $T_pM$.
	\end{Def} 
	Let $(U,\phi)$ be a chart of $M$ centered in $p\in M$, we set:
	$${\partial\over \partial x^i}\bigg\rvert_p(f)={\partial\over \partial r^i}\bigg\rvert_{\phi(p)}(f\circ \phi^{-1})$$ where $r^i$ are the coordinates of $\mathbb{R}^n$. This definition makes ${\partial\over \partial x^i}$ a vector field since it follows Leibniz.\\
	\\
	An important result is the following: considering the tangent space $T_pM$ and a chart $(U,\phi)$ centered in $p$, then the vectors $\partial\over \partial x^i$ form a basis for $T_pM$. This comes from the fact that the tangent vectors $\partial\over \partial r^i$ are a basis for the tangent space in $x_0\in\mathbb{R}^n$, which has the same dimensions as $M$. Thus, once we choose a chart, a generic tangent vector can be expressed as a linear combination: $$\vec{v}=\sum_{i=1}^{n}c_i{\partial\over \partial x^i}$$ 
	From now on, we will also refer to $\partial\over \partial x^i$ with $\partial_i$.
	\begin{Obs} \label{Obs 1.1.3}
		Looking at the open subset $GL_n(\mathbb{R})$ of $M_{n\times n}$, by the previous observations \ref{Obs:1.1.1} and \ref{Obs:1.1.2}, $GL_n(\mathbb{R})$ is a smooth submanifold of $M_{n\times n}$ and its dimension is $n^2$, equal to the one of $M_{n\times n}$. However, the tangent space at the identity $T_\mathbb{I}M_{n\times n}$ has itself dimension $n^2$. From this we have $T_\mathbb{I}GL_n(\mathbb{R})\simeq T_\mathbb{I}M_{n\times n}$.
	\end{Obs}
	\begin{Def}
		Given a $C^\infty$ map between smooth manifolds like $F:N\rightarrow M$, we call \textit{differential} of $F$ in a point $p\in N$ the map $dF_p:T_pN\rightarrow T_{F(p)}M$ acting as follows: for any vector $X_p\in T_pN$ and for any map $f\in C_{F(p)}^\infty(M)$, it holds $dF_p(X_p)f=X_p(f\circ F)\in\mathbb{R}$.
	\end{Def}
	From the fact that tangent vectors are derivations, it follows that the differential is a derivation as well. It is possible to show that the differential of a composite function follows the \textit{chain rule:} 
	$$d(F\circ G)_p=dF_{G(p)}\circ dG_p$$ 
	For the full proof the reader can look at [1] (pag. 88 cap 3).
	\begin{Ex}
		Let $x^1,...,x^n$ be the coordinates of $\mathbb{R}^n$ and $y^1,...,y^m$ the coordinates of $\mathbb{R}^m$. Let $F:\mathbb{R}^n\rightarrow \mathbb{R}^m$ be a $C^\infty$ map and $p\in \mathbb{R}^n$. Then, the differential of $F$ evaluated in $p$ is a map $dF_p:T_p\mathbb{R}^n\rightarrow T_{F(p)}\mathbb{R}^m$ such that, given any tangent vector at $p$ like $X_p\in T_p\mathbb{R}^n$ and a map $f\in C_{F(p)}^\infty(\mathbb{R}^m)$, the following relation holds $dF_p(X_p)f=X_p(f\circ F)$.\\
		Recalling that a basis for $T_p\mathbb{R}^n$ is made up by the vectors $\{{\partial\over\partial x^i}\}$, taking a vector $X_p\in T_p\mathbb{R}^n$ defined as $X_p={\partial\over \partial x^j}$, we can write $$dF_p\bigg{(}{\partial\over \partial x^j}\bigg{)}=\sum_{k=1}^{m}d_j^k{\partial\over \partial y^k}\bigg{\rvert}_p$$
		The coefficients $d$ can be found by evaluating the following 
		$$dF_p\bigg{(}{\partial\over \partial x^j}\bigg{)}y^i=\sum_{k=1}^{m}d_j^k{\partial\over \partial y^k}\bigg{\rvert}_{F(p)}y^i=d_j^i$$
		Moreover, knowing that, by definition of differential, it holds $dF_p(X_p)f=X_p(f\circ F)$, we can further expand:
		$$dF_p\bigg{(}{\partial\over \partial x^j}\bigg{)}y^i={\partial\over \partial x^j}\bigg{\rvert}_p(y^i\circ F)={\partial F^i\over \partial x^j}(p)=d^i_j$$
		Thus, the matrix which defines the differential of $F$ in a point $p$ is exactly the jacobian matrix of $F$ evaluated at $p$.
	\end{Ex}
	We now define the concept of vector field.
	\begin{Def}
		We call \textit{vector field} on $M$ a map $X$ such that to any point it associates a vector in the tangent space at that point: $X:p\mapsto X_p$. 
	\end{Def}
	\begin{Obs}
		We saw that a vector can be identified with a map $\vec{v}:f\mapsto\sum_{i=1}^{n}c_i{\partial f\over \partial x^i}$ for a generic point $p$ inside a chart. Then a vector field can also be seen as a map $X:f\mapsto\vec{v}(f)$ such that $X(f)(p)=\sum_{i=1}^{n}c_i(p){\partial f(p)\over \partial x^i}$
	\end{Obs}
	\begin{Def}
		A vector field $X$ on $M$ is said to be \textit{smooth} or $C^\infty$ if for every $f\in C^\infty(M)$, $X(f)$ is $C^\infty$. Equivalently, $X=\sum_{i=1}^{n}c_i{\partial\over \partial x^i}$ is said to be $C^\infty$ if the functions $c_i$ are all $C^\infty$.
		\begin{Def}
			A curve $c_p:]-\epsilon,\epsilon[\rightarrow M$ is said to be an \textit{integral curve} of a vector field $X$ on $M$ passing through $p\in M$ if $c_p(0)=p$ and $c_p'(0)=X_p$.
		\end{Def}
		From the theory of differential equations, given any point $p\in M$ and a vector field $X$ defined in a neighborhood of $p$, there always exists a unique integral curve of $X$ passing through $p$. The reader can find more details about this part in [1] pag. 154 cap.3.\\
		\\
		It is useful to define the concept of \textit{flow} of a vector field $X$ as the map $\Phi_X:\mathbb{R}\times M\rightarrow\mathbb{M}$ such that $\Phi_X(0,p)=p$; $\Phi_X(t,p)=c_p(t)$ is the integral curve of $X$.\\
		\\
		More details on the flux of a vector field can be found in [1] pag. 155, 156 cap. 3.
	\end{Def}
	Lastly, we define the concept of metric manifold.
	\begin{Def}
		A \textit{metric} on a manifold $M$ is a smooth assignment to any point $p\in M$ of an inner product $\braket{,}_p:T_pM\times T_pM\rightarrow \mathbb{R}$:
		\begin{itemize}
			\item $\braket{,}_p$ is bilinear;
			\item $\braket{,}_p$ is non degenerate;
			\item $\braket{,}_p$ is symmetric.
		\end{itemize}
		We also refer to any manifold $M$ endowed with a metric with the name \textit{metric manifold} and the notation $(M,g)$.
	\end{Def}
	One can think of a metric as a map $g$ that reduces to $\braket{,}_p$ at any point. By definition, $g$ is smooth and in local coordinates we identify its components with $g_{ij}$.
	In general, one can consider metrics which are not positive definite. An example is the Minkowski metric.
	\begin{Def}
		We call \textit{Riemanniann manifold} a metric manifold in which the metric tensor is positive definite. If the metric is not positive definite we call the manifold \textit{pseudo-Riemannian}.
	\end{Def}
	\section{The tangent bundle}
	In this section we introduce the notion of tangent bundle of a smooth manifold. For more details one can check [1] pag. 129-139.\\
	\\
	Let $M$ be a manifold. At any point $p\in M$ we can construct the tangent space $T_pM\simeq \mathbb{R}^n$. We define the tangent bundle as the disjoint union of all the tangent spaces.
	$$TM=\bigsqcup_{p\in M}T_pM=\bigcup_{p\in M} \{p\}\times T_pM$$
	This set has a natural projection 
	$$\pi:TM\rightarrow M\hbox{ acting like }\pi(p,v_p)=p \hbox{ where }v_p\in T_pM$$  
	We now endow the tangent bundle with a manifold structure. To achieve this, we first of all need to give the tangent bundle a T2 and second countable topology. The idea is to induce a manifold structure on $TM$ by using the one of the underlying manifold $M$.\\
	\\
	Let $(U,\phi)$ be a chart on $M$ with $\phi:U\rightarrow \mathbb{R}^n$. Then, any vector in $p\in U$ can be written in local coordinates: $v_p=v^i_p\partial_i|_p$. Now we construct the following mapping:
	$$\Phi:TU\rightarrow \phi(U)\times \mathbb{R}^n; \hbox{ like: }
	\Phi(p,v_p)=(x^i_pe_i,v^i_p\partial_i|_p)$$
	This mapping is 1-1 and surjective so it is a bijection. Moreover, it is continuous. Thus, it is an homeomorphism.
	\begin{Theo} \label{Theo_1.1}
		The tangent bundle $TM$ has a second countable and Hausdorff topology
	\end{Theo}
	\begin{proof}
		The reader can find the complete proof in [1] Chap. 3 pag 132.
	\end{proof}
	Theorem \ref{Theo_1.1} ensures that $TM$ has the topological qualities needed to be a topological manifold. We now show that it is locally Euclidean and that it possesses a smooth atlas.
	\begin{Theo}
		$TM$ is a smooth manifold.
	\end{Theo}
	\begin{proof}
		Consider an Atlas $\mathfrak{A}_M=\{U_\alpha,\phi_\alpha\}$ for the manifold $M$. We want to show that the induced collection $\mathfrak{A}_{TM}=\{TU_\alpha,\Phi_\alpha\}$ is an atlas on $TM$, knowing that $TM=\bigcup TU_\alpha$ and $\Phi_\alpha:TU_\alpha\rightarrow U_\alpha\times \mathbb{R}^n$ is the map constructed in the previous section.\\
		\\
		As we have said previously, the maps $\Phi_\alpha$ are homeomorphisms between open sets of $TM$ and $\mathbb{R}^{2n}$. This makes $TM$ locally euclidean. It remains to check the compatibility between two overlapping charts.\\
		\\
		Consider two overlapping charts $\Phi_1,\Phi_2$ on $TU_1,TU_2$ open sets of the tangent bundle, with $TU_1\cap TU_2\neq \emptyset$. Then on the two corresponding charts $(U_1,\phi_1)$, $(U_2,\phi_2)$ on $M$, we can express tangent vectors in two different coordinate basis:
		$$v=a^i{\partial\over \partial{x^i}}=b^j{\partial \over \partial y^j}$$
		Clearly at any point:
		$$a^i=b^j{\partial x^i\over\partial y^j} \hbox{ and } b^j=a^i{\partial y^j\over\partial x^i}$$
		Now it only remains to show that the composition $\Phi_1\circ \Phi_2^{-1}$ is a diffeomorphism.\\
		By definition:
		$$\Phi_1\circ \Phi_2^{-1}:\phi_2(U_2\cap U_1)\times \mathbb{R}^n\rightarrow \phi_1(U_2\cap U_1)\times \mathbb{R}^n$$
		This map, being a composition of homeomorphisms, is still an homeomorphism. Now, taking a point $p\in U_2\cap U_1$ and a tangent vector $v_p$ we write:
		$$\Phi_1\circ \Phi_2^{-1}(x^i_p,a^i_p)=(y^j_p,b^j_p)$$
		where we have shortened the notation:
		$$(x^i,a^i)=(x^1_p,...,x^n_p,a^1_p,...,a^n_p)$$
		However, we can make the following substitution:
		$$(y^j_p,b^j_p)=\bigg((\phi_2\circ \phi_1^{-1})(\phi_1(p)),a^i_p{\partial y^j\over\partial x^i}\bigg|_p\bigg)=\bigg((\phi_2\circ \phi_1^{-1})(\phi_1(p)),a^i_p{\partial (\phi_2\circ \phi_1^{-1})^j\over\partial r^i}\bigg|_p(\phi_1(p))\bigg)$$
		Due to $\phi_2\circ \phi_1^{-1}$ being a diffeomorphism, $\Phi_1\circ \Phi_2^{-1}$ is also a diffeomorphism.\\
		This completes the proof.
	\end{proof}
	Thus, the tangent bundle $TM$ of a manifold of dimension $n$ is also a manifold, but of dimension $2n$, with an atlas given by:
	$$\mathfrak{A}_{TM}=\{TU_\alpha,\Phi_\alpha\} \hbox{ where } \Phi_\alpha:TU_\alpha\rightarrow U_\alpha\times \mathbb{R}^n$$
	\section{Forms on manifolds}
	In this section we will introduce the concept of forms on smooth manifolds and see some properties of them. More informations on this topic can be found in [1] chap. 3 and [2] chap. 1.\\
	\\
	Given a smooth manifold $M$, at any point $p\in M$ we can consider it's tangent space $T_pM$. Given that this is a vector space, we can look at its dual $T_pM^*$. 
	\begin{Def}
		An element of $T_pM^*$ is called \textit{co-vector}. We call a function that assigns to any point of $M$ a co-vector a \textit{1-form}.
	\end{Def}
	\begin{Obs}
		Once we choose a chart $(U,\phi)$ on $M$, with coordinates $\{x^i\}$, we have a basis of $T_pM$ of vectors $\{{\partial\over \partial x^i}|_p\}$ for any $p\in U$. We can then induce another basis on the dual space $T_pM^*$, indexed with $\{dx^i_p\}$ defined by:
		$$dx^i_p\bigg({\partial\over \partial x^j}\bigg|_p\bigg)=\delta^i_j$$
	\end{Obs}
	\begin{Obs}
		Note that on a metric manifold $(M,g)$, after choosing a chart, we can expand the metric tensor in local coordinates:
		$$g=g_{ij}dx^i\otimes dx^j$$  
	\end{Obs}
	The most important example of a form is the differential:
	\begin{Prop} \label{df_is_a_form}
		Let $f$ be a $C^\infty$ function on $M$. Then it's differential is a 1-form.
	\end{Prop}
	\begin{proof}
		By definition of differential, if $X$ is a vector field, at any $p$ we can write:
		$$df_p(X_p)=X_pf$$
		This is clearly a map that assigns to any point a co-vector. Namely:
		$$df(p)(X)=X_pf$$
	\end{proof}
	\begin{Obs}
		Clearly, the differential of a coordinate function corresponds exactly to the dual basis:
		$$dx^i_p\bigg({\partial\over \partial x^j}|_p\bigg)={\partial\over \partial x^j}\bigg|_px^i=\delta^i_j$$
		One can then show that the coordinate chart $\phi^*(p)=(dx^1_p,...,dx^n_p)$ induced on the cotangent bundle is smooth. The proof can be found in [1] chap. 5 pag. 193-195.
		We can use this result to find a local expression for the differential of a map:
		knowing from proposition \ref{df_is_a_form} that $df$ is a 1-form, once we fix a chart $(U,\phi)$, we can expand locally like:
		$$df_p=a_i(p)dx^i_p$$
		Where $a_i(p)$ are real numbers, depending on the point.
		Then, by feeding this differential our coordinate functions we get:
		$$df_p\bigg({\partial\over \partial x^j}\bigg|_p\bigg)=a_i(p)\delta^i_j={\partial\over \partial x^j}\bigg|_pf$$
		So that we can expliticly write for any point inside $U$:
		$$df={\partial\over \partial x^i}\bigg|_pfdx^i=\partial_ifdx^i$$
	\end{Obs}
	We now look at the exterior algebra of the tangent space, in order to generalize the concept of 1-forms. The dual tangent space is a vector space and so one can define it's exterior power $\bigwedge^k T_pM^*$. This set contains skew symmetric elements of the form:
	$$\omega^1_p\wedge \omega^2_p\wedge...\wedge \omega^k_p$$
	Furthermore, we can define the set:
	$$\bigwedge^k TM^*=\bigsqcup_{p\in M}\bigwedge^k T_pM^*$$
	This is the set of all alternating dual vectors at all points on the manifold. One can then give the set $\bigwedge^k TM^*$ the structure  of a smooth manifold in the same way one does with the tangent bundle.
	\begin{Theo}
		$\bigwedge^kTM^*$ is a smooth manifold.
	\end{Theo}
	\begin{proof}
		The proof is similar to the one seen for the tangent bundle and can be found in [1] chap. 5 pag. 192-193.
	\end{proof}
	\begin{Def}
		Let $M$ be a smooth manifold. We define a \textit{$k$-form} on $M$ as a map that associates to each $p\in M$ an element of $\bigwedge^k TM^*$ in the corresponding point. 
	\end{Def}
	Sometimes we will index elements of $\bigwedge^k TM^*$ with the following notation: $dx^I$. The index $I$ is called multi-index and represents a subset of indices of $\{1,2,3,...,k\}$. In particular, for a $k$-form, $dx^I$ stands for $dx^{i_1}\wedge dx^{i_2}\wedge...\wedge dx^{i_k}$.
	\begin{Def}
		We say that a $k$-form $\omega$ is \textit{smooth} if for any chart $(U,\phi)$, the functions $a_I:M\rightarrow \mathbb{R}$ in the expansion $\omega=a_Idx^I$ are smooth.
		We denote the space of smooth $k$-forms with $\Omega^k(M)$. 
	\end{Def}
	Later on, we will see that $k$-forms can be defined as sections of a precise vector bundle. We now define a key concept for our analysis on gauge theories: the pullback.
	\begin{Def}
		Let $F:N\rightarrow M$ be a smooth map between manifolds.
		Let $\omega_{F(p)}$ be a smooth $k$-form on $F(p)\in M$. The \textit{pullback} of $\omega_{F(p)}$ is a covector in $p\in N$ defined as:
		$$F^*(\omega_{F(P)})(v_1,...,v_n)=\omega_{F(p)}(dF_p(v_1),...,dF_p(v_n))$$ 
	\end{Def}
	The pullback has some useful properties:
	\begin{Prop}
		Let $f:N\rightarrow M$ and $\omega,\sigma\in \Omega^k(M),\tau\in\Omega^l(M)$, then:
		\begin{itemize}
			\item the pullback is linear: $f^*(a\omega+b\sigma)=af^*\omega+bf^*\sigma$ for $a,b\in \mathbb{R}$;
			\item $f^*(\omega\wedge \tau)=f^*\omega\wedge f^*\tau$;
			\item the pullback commutes with the exterior derivative:
			$$df^*\omega=f^*d\omega$$
		\end{itemize}
	\end{Prop}
	\begin{proof}
		The proof can be found in [2] Appendix pag. 303.
	\end{proof}
	\section{Orientation on smooth manifolds}
	In this section we will briefly sudy the notion of orientation and integration on a smooth manifold. In particular, we will define the volume form and the notion of orientable manifold. More references on this topic can be found in [1] chap. 6.
	\\
	\\
	We first of all define the notion of orientation on a generic vector space. We will then generalize those definitions to our framework of smooth manifolds.
	\begin{Def}
		An \textit{orientation} on any finite dimensional vector space $V$ is the equivalence class of an ordered basis, under ther relation:
		$$[e_1,...,e_n]\sim[f_1,...,f_n]A$$
		Where $A$ is an invertible matrix with positive determinant.
	\end{Def}
	Clearly, there are only 2 possible orientations on a manifold, since every basis is related to any other basis by a matrix of $GL(n,\mathbb{R})$ (on a real vector space).
	\begin{Def}
		We call \textit{pointwise orientation} on a manifold a map $\mu$ that assigns to each point $p\in M$ an orientation for $T_pM$. A manifold $M$ with a continuous pointwise orientation  is called \textit{orientable}.
	\end{Def}
	An orientation for the tangent space is the equivalence class of an ordered basis: $[\{X^i\}]$. So, an orientation for a manifold is a map that assigns to each point a class of an ordered basis of the corresponding tangent space, in a continuous way:
	$$\mu(p)=[\{X^i_p\}]$$
	\begin{Obs}
		Since every vector space has only 2 orientations, and at each point the tangent space is indeed a vector space, any connected orientable manifold has only 2 possible orientations. The full proof of this claim can be found in [1] chap. 6 pag. 241-242.
	\end{Obs}
	\begin{Ex}
		The space $\mathbb{R}^n$ is orientable and an orientation is given by the global frame ${\partial\over \partial{r^i}}$ where $r^i$ are the standard coordinates.
	\end{Ex}
	\begin{Ex}
		Ler $R=\{(x,y)\in \mathbb{R}^2|x\in[0,1], y\in [-1,1[\}$
		The Möbius strip can be defined as the quotient: $R\over \sim$ where $(0,y)\sim(1,-y)$. We now show that this structure is not orientable.\\
		Suppose that the Möbius strip is orientable. Let $U$ be the interior of the rectangle: $U=\{(x,y)\in \mathbb{R}^2|x\in]0,1[, y\in [-1,1[\}$. Assume an orientation given by the basis $e_1,e_2$ of the tangent space in this rectangle. By continuity of the orientation, those vectors will be an orientation also for the points $(0,0)$ and $(1,0)$. However, this basis at $(1,0)$ is mapped to $e_1,-e_2$ at $(0,0)$. This proves the orientation is not continuous and generates an absurd. Thus, the Möbius strip is not orientable.
	\end{Ex}
	\begin{Prop}
		A pointwise orientation $\mu(p)=[\{X^i_p\}]$ on a manifold $M$ is continuous if and only if for any chart $(U,\phi)$ with coordinates $\{x^i\}$, the function:
		$$dx^1\wedge...\wedge dx^n(X^1,...,X^n)$$
		is positive everywhere.
	\end{Prop}
	\begin{proof}
		Suppose the orientation is continuous. This means that for each point there exists an open set $V$ inside which the orientation is represented by a continuous frame $[\{X^i\}]$. Choose then a chart $(U,\phi)$ contained in $V$ and expand the frame like:
		$$X^i=a^{ij}{\partial\over \partial x^j}$$
		Then, by feeding this to the $n$-form above:
		$$dx^1\wedge...\wedge dx^n(X^1,...,X^n)=det(a^{ij})dx^1\wedge...\wedge dx^n\bigg({\partial\over \partial x^1},...,{\partial\over \partial x^n}\bigg)=det(a^{ij})$$
		and the determinant is always non vanishing since it is a change of basis determinant. However, it can be either greater or lower than 0. By continuity of the orientation, if the determinant is $>0$ at a point $p\in U$, it is also positive everywhere else. If instead the determinant is negative, we will simply choose $\tilde{x}^1=-x^1$ and gain an additional $(-)$ sign.\\
		Suppose now that the function $dx^1\wedge...\wedge dx^n(X^1,...,X^n)$ is everywhere positive inside a given chart $(U,\phi)$. In those coordinates, the field can be expanded like before and we obtain again the same result:
		$$dx^1\wedge...\wedge dx^n(X^1,...,X^n)=det(a^{ij})$$
		If the left-hand side of this equation is always positive inside $U$, so it is the right hand side. From this we can say that $[\{X^i\}]$ and $[\{{\partial\over\partial x^i}\}]$ describe the same orientation and so $\mu$ is continuous.
	\end{proof}
	\begin{Theo}
		A smooth manifold $M$ is orientable if and only if there exists a non vanishing smooth $n$-form on it.
	\end{Theo}
	\begin{proof}
		To see the proof one can look at [1] chap. 6 pag. 243.
	\end{proof}
	We now define a very important form on metric manifolds, called volume form. This will be crucial in the definition of the Hodge operator.
	\begin{Def}
		Let $(M,g)$ be a smooth metric manifold manifold and $(U,\phi)$ a chart with coordinates $\partial\over \partial x^i$. We define the volume form as:
		$$\omega=\sqrt{|det(g^{ij})|}dx^1\wedge...\wedge dx^n$$
	\end{Def}
	This is clearly a nowhere vanishing form. It can be shown that, by applying the Gram–Schmidt process, the volume form takes the simpler form:
	$$\omega=de^1\wedge...\wedge de^n$$
	where $e^i$ is the orthonormal basis with respect to $g$ inside $U$. The proof is just a straightforward calculation and can be found in [2] chap. 3 pag. 134.
	\section{Exterior derivative and Hodge operator}\label{Sec_1.7}
	In this section we will introduce the notion of exterior derivative and hodge operator. More informations on those topic can be found in [3] chap. 7 pag 401-413.
	\begin{Def}
		Let $M$ be a smooth manifold and $(U,\phi)$ a chart. We define the \textit{exterior derivative} of a $k$-form $\omega=a_Idx^I$ as the $(k+1)$-form:
		$$d\omega=da_I\wedge dx^I$$
		where $da_I$ is the differential of the function $a_I$.
	\end{Def}
	Now we look at some important properties of the exterior derivative.
	\begin{Prop}
		Let $M$ be a smooth manifold. Then the following properties hold:
		\begin{itemize}
			\item the exterior derivative is an \textit{anti-derivation of degree 1}: $$d(\omega\wedge\beta)=d\omega\wedge\beta+(-)^{deg(\omega)}\omega\wedge\beta$$
			where $\omega\in \bigwedge^k TM^*$, $\beta\in \bigwedge^l TM^*$ and $deg(\omega)=k$.
			\item $d^2=0$;
		\end{itemize}
	\end{Prop}
	\begin{proof}
		The first property is just a consequence of the algebra of the exterior power. In particular, let $\omega=a_Idx^I$, $\beta=b_Jdx^J$ in our chart. Recalling that the differential of a $1$-form is linear and respects Leibniz:
		$$d(a_Idx^I\wedge b_Jdx^J)=d(a_Ib_Jdx^I\wedge dx^J)=d(a_Ib_J)\wedge dx^I\wedge dx^J=$$
		$$=da_I\wedge dx^I\wedge b_Jdx^J+a_Idb_J\wedge dx^I\wedge b_Jdx^J=d\omega\wedge\beta+(-)^{deg\omega}\omega\wedge db_J\wedge dx^J$$
		As for the seconf property, it is a consequence of the symmetry of the Hessian matrix.\\
		Recall that the differential of a function is:
		$$df=\partial_ifdx^i$$
		Taking again the differential we get second derivatives:
		$$d^2f=\partial_j\partial_i f dx^j\wedge dx^i$$
		However, while the second partial derivation is symmetric, the wedge product is skew-symmetric. This gives 0 as a result.
	\end{proof}
	The above properties of the exterior derivative con be shown to define it in a unique way on any smooth manifold.
	\begin{Theo}
		On any smooth manifold $M$ there exists a unique exterior derivative $d$ such that:
		\begin{itemize}
			\item $d$ is an anti-derivation of degree 1;
			\item $d^2=0$;
			\item if $f$ is a smooth function and $X$ is a vector field, then:
			$$df(X)=Xf$$
		\end{itemize}
	\end{Theo}
	\begin{proof}
		The reader can find the full proof in [1] chap. 5 pag. 210-214.
	\end{proof}
	\begin{Def}
		A $k$-form $\omega$ is said to be \textit{closed} if $d\omega=0$ and \textit{exact} of $\omega=d\alpha$ for $\alpha\in\Omega^{k-1}(M)$.
	\end{Def}
	\begin{Obs}
		It is clear that any exact form is closed. However, in general, not all closed forms are exact.
	\end{Obs}
	We now look at the definition of the Hodge operator and see some properties of it. This operator will be useful in writing the Maxwell equations from a differential geometry point of view.
	\begin{Obs}
		The inner product on $M$ induces an inner product on forms. Namely, at any point $p\in M$ we have:
		$$\braket{\alpha^1_p\wedge...\wedge \alpha^k_p,\beta^1_p\wedge...\wedge\beta^k_p}=det(\braket{\alpha^i_p,\beta^j_p}_{i,j\in[1,k]})$$
		In particular, at any $p\in M$, we have a smooth inner product of vectors. Since $g$ is a bilinear, symmetric and non degenerate, it induces an isomorphism on the cotangent space at any point. The inner prduct of 1-forms is defined as being equal to the inner product of the corresponding vectors. More on this construction will be said in section \ref{Sec_3.3}.
		The extension of this product to $k$-forms is clearly symmetric. The bilinearity follows from the multilinearity of the determinant for sums of rows and columns. As for the non degeneracy, it is a consequence of the non degeneracy of $\braket{,}$. Lastly, since $\braket{,}$ is smooth, this induced product is also smooth.
	\end{Obs}
	\begin{Def}
		Let $(M,g)$ be a pseudo Riemannian orientable manifold of dimension $n$. Let $\omega$ denote the volume form. We define the \textit{Hodge operator} acting on $k$-forms as:
		$$\star:\bigwedge^kTM^*\rightarrow \bigwedge^{n-k}TM^*$$
		such that at any $p\in M$, for every $\alpha_p,\beta_p\in\bigwedge^kT_pM^*$ we have:
		$$\alpha_p\wedge(\star\beta_p)=\braket{\alpha_p,\beta_p}\omega_p$$
	\end{Def}
	This operator is clearly bilinear since the inner product on forms is. In the next proposition we will see that it also depends smoothly on points.
	\begin{Prop}\label{Prop_1.7.2}
		Let $(M,g)$ an oriented pseudo-Riemannian manifold and $\omega$ its volume form. Then the Hodge operator satisfies, for any $p\in M$, the following properties:
		\begin{itemize}
			\item $\star 1=\omega;$
			\item $\star\omega=sgn(g)$ where $sgn(g)$ is $(-)^q$ and $q$ is the number of -1 in the signature of $g$;
			\item let $\{dx^i_p\}$ be a generic basis for $T_pM^*$, not necessarily orthonormal, let $[dx^i]$ be it's orientation and let $I,J$ be ordered subsets of $\{1,2,3,...,n\}$ of $k$ elements and $J'$ be the ordered complement of $J$, then:
			$$\star dx_p^I=[dx^i]\sqrt{|det(g_p)|}\sum_J[JJ']\braket{dx_p^I,dx_p^J}dx_p^{J'}$$
			where $[JJ']$ is the sign of the permutation needed to put the set $JJ'$ in order;
			\item $\star^2=sgn(g)(-)^{k(n-k)}$
		\end{itemize}
	\end{Prop}
	\begin{proof}
		In this proof we will omit the subscript $p$ specifying the point to simplify the notation. 
		By definition of the Hodge operator:
		$$\alpha_p\wedge(\star\alpha_p)=\braket{\alpha_p,\alpha_p}\omega_p$$
		From this, the first property is proved since: $\braket{1,1}=1$.\\
		The second one is a clear consequence of:
		$$\braket{\omega,\omega}=sgn(g)$$
		to prove the third property, consider the following: one can always expand $\star dx^I_p=\sum_Jc_p^{IJ}dx_p^{J'}$ since it is a $n-k$ form. The sum is on every ordered subset $J$ of $\{1,...,n\}$ of length $k$. Taking now the wedge product from both sides with $\sum_J dx^J$ we get:
		$$\sum_Jdx_p^J\wedge(\star dx_p^I)=\sum_Jdx_p^J\wedge\sum_Jc_p^{IJ}dx_p^{J'}$$
		Since $J$ are all different and of same length, each of their complements $J'$ will contain at least one element in common with every possible $J$, except the one they are complementary to. Knowing that the wedge product between two repeated elements is 0, we can cancel off one of the two sums over $J$ and find:
		$$\sum_Jdx_p^J\wedge(\star dx_p^I)=\sum_Jc_p^{IJ}dx_p^J\wedge dx_p^{J'}=\sum_J\braket{dx_p^J,dx_p^I}\omega_p$$
		for each $J$, we can write $\omega_p=\sqrt{|det(g_p)|}[JJ']dx_p^J\wedge dx_p^{J'}$ where $[JJ']$ is the sign of the permutation needed to put the set $JJ'$ in order. This clearly implies:
		$$c_p^{IJ}=[dx_p^i]\sqrt{|det(g_p)|}[JJ']\braket{dx_p^J,dx_p^I}$$
		Putting everything together we find:
		$$\star dx_p^I=[dx^i]\sqrt{|det(g_p)|}\sum_J[JJ']\braket{dx_p^I,dx_p^J}dx_p^{J'}$$
		This formula shows that $\star$ depends smoothly on the points of the manifold.
		As for the last property, we can prove it for an orthonormal basis $\{de_p^i\}$. By construction, if $I$ is a subset of length $k$ of $\{1,2,...,n\}$, $\star de_p^I=[de_p^i]\braket{de_p^I,de_p^I}[IJ'] de_p^{J'}$ where $J'$ is the ordered complement of $I$. In fact, being an orthonormal basis, the only non 0 term in the above sum is the one where $I=J$. Consider then the following equations:
		$$\star de_p^I\wedge (\star^2 de_p^I)=\braket{\star de_p^I,\star de_p^I}\omega_p\hbox{ and: } de_p^I\wedge (\star de_p^I)=\braket{de_p^I,de_p^I}\omega_p$$
		By substituting the previous expression for $\star de^I$, we see that $\braket{\star de_p^I,\star de_p^I}=\braket{de_p^{J'}, de_p^{J'}}$. Also, by the alternating property of the exterior algebra, exchanging the forms amounts to an additional $(-)$ sign: $\star de_p^I\wedge (\star^2 de_p^I)=\star^2 de_p^I\wedge (\star de_p^I)\cdot (-)^{k(n-k)}$. Clearly, combining the two previous expressions, one finds:
		$$\star^2 de^I=(-)^{k(n-k)}\braket{de_p^{J'}, de_p^{J'}}\braket{de_p^{I}, de_p^{I}}de^I$$
		And since the basis is orthonormal, one finds $\braket{de_p^{J'}, de_p^{J'}}\braket{de_p^{I}, de_p^{I}}=sgn(g)$. This proves the last property.
	\end{proof}
	\begin{Ex}
		Let $\mathbb{R}^{1,3}$ be Minkowski space with the metric tensor $\eta=diag(-,+,+,+)$. Let $\{dt,dx,dy,dz\}$ be an orthonormal basis for the cotangent space at any point and consider the volume form $\omega=dt\wedge dx\wedge dy\wedge dz$. We can explicitly calculate:
		$$\star dx=[x,t,y,z]\braket{dx,dx}dt\wedge dy\wedge dz$$
		And clearly the permutation $[x,t,y,z]$ gives (-) as a result. Thus:
		$$\star dx=-dt\wedge dy\wedge dz$$
		In fact, one clearly sees that:
		$$dx\wedge (\star dx)=-dx\wedge dt\wedge dy\wedge dz=\omega$$
	\end{Ex}
	We now use the Hodge operator to construct another linear mapping, which will behave like the adjoint operator of the exterior derivative $d$.
	\begin{Def}
		We define the \textit{co-differential} acting on $k$ forms as:
		$$d^*=(-)^{k(n-k)+1}sgn(g)\star \circ  d\circ\star$$
	\end{Def}
	This operator has some interesting properties:
	\begin{Prop}
		The co-differential satisfies the following properties:
		\begin{itemize}
			\item $d^{*2}=0$;
			\item $\braket{\alpha,d\beta}=\braket{d^*\alpha,\beta}$;
		\end{itemize}
	\end{Prop}
	\begin{proof}
		The first claim is obvious since:
		$$d^{*2}\propto\star \circ  d\circ\star\circ\star \circ  d\circ\star=\star \circ  d\circ\star^2\circ  d\circ\star\propto\star\circ d^2\circ\star=0$$
		As for the second claim, it requires some knowledge about integration on manifolds. Since this work does not cover this topic, the reader is advised to look at [3] chap. 7 pag. 411 for the full proof.
	\end{proof}
	Before concluding this section, we give an incredibly important result known as the \textit{Poincaré Lemma}.
	\begin{Lm}\label{P.L.}
		On any open ball of $\mathbb{R}^n$ every closed form is exact.
	\end{Lm}
	The proof of this lemma goes beyond the scope of this work. The reader can find it in [1] chap. 7 pag 300.\\
	\\
	Since any smooth manifold is locally homeomorphic to $\mathbb{R}^n$, it means that, at least locally, every closed form can be expressed as an exact form.
	\chapter{Lie groups and Lie algebras}
	In this chapter we introduce the concepts of Lie group and Lie algebra, with some important examples. We will also study representations of those mathematical structures. In the framework of gauge theories, Lie groups are a fundamental object since they will represent the symmetry groups of our system. More information on these topics can be found in:
	[1] chap. 4; [3] chap. 2, 3; [5] chap. 2, 3; [6] chap. 1, 2; [7] part 1, 2.
	\section{Topological groups}
	In this section we introduce topological groups, group actions and quotients, with some examples. More information can be found in [1] chap. 1 pag. 66, [4] chap. 5 pag. 35,36,37. 
	\begin{Def}
		We call \textit{group} a set $G$ endowed with a binary operation $\cdot:G\times G\rightarrow G$ called \textit{multiplication} such that it has the following properties:
		\begin{itemize}
			\item Associativity: $a\cdot (b\cdot c)=(a\cdot b)\cdot c$ $\forall a,b,c\in G$;
			\item Existence of neutral element: $\exists e\in G$ such that $\forall a\in G$ it holds $a\cdot e=e\cdot a=a$;
			\item Existence of inverse: $\forall a\in G$ $\exists a^{-1}$ such that $a\cdot a^{-1}=a^{-1}\cdot a=e$.
		\end{itemize}
		A group with a commutative multiplication so that $ab= b\cdot a$ for every $a,b\in G$, is called \textit{commutative} or \textit{abelian}.
	\end{Def}
	From now on, the multiplication will also be indicated by omitting the symbol $\cdot$ to simplify the notation.
	\begin{Def}
		A \textit{topological group} $G$ is a topological space with the group structure, such that the multiplication $\cdot:G\times G\rightarrow G$ and the inversion $i:G\rightarrow G$ defined as $i(g)=g^{-1}$ are both continuous.
	\end{Def}
	\begin{Ex}
		$\mathbb{R}^n$ with the euclidean topology is a topological abelian group with the usual sum.
	\end{Ex}
	We now define the concept of group action on a generic set $X$.
	\begin{Def}
		Consider a group $G$ and a set $X$. We say that $G$ \textit{acts} (on the left) on $X$ or that $X$ is a (left) \textit{$G$ set} if there exists a map, called (left) \textit{action} $a:G\times X\rightarrow X$ such that:
		\begin{itemize}
			\item 	$a(e,x)=x$ for all $x\in X$;
			\item $a(gh,x)=a(g,a(h,x))$ for all $g,h\in G$ and $x\in X$.
		\end{itemize} 
		If $X$ is also a topological space and the map $x\longmapsto a(g,x)$ is continuous for every $g\in G$, then $X$ is called \textit{$G$ space}.
		We will write $a(g,x)$ also as $g\cdot x$.
	\end{Def}
	\begin{Obs}
		We defined the concept of left action, however we could have also defined in the same way a right action: $b:X\times G\rightarrow X$ such that $b(x,e)=x$ and $b(x,gh)=b(b(x,g),h)$. We can immediately show that, given a left action, there is an induced right action: in fact: $a:G\times X\rightarrow X$ defined as $(g,x)\longmapsto a(g,x)$, we can build a map $b:X\times G\rightarrow X$ defined as $(x,g)\longmapsto a(g^{-1},x)=b(x,g)$. 
		It is not difficult to show that this new action respects the needed properties:
		$$b(x,e)=a(e,x)=x$$
		In fact $e^{-1}=e$ since $G$ is a group. Recalling that $(gh)^{-1}=h^{-1}g^{-1}$, we have the chain of equalities:
		$$b(x,gh)=a((gh)^{-1},x)=a(h^{-1}g^{-1},x)=$$$$=a(h^{-1},a(g^{-1},x))=a(h^{-1},b(x,g))=b(b(x,g),h)$$
		This shows that the right map $(x,g)\longmapsto a(g^{-1},x)$ is well defined.
	\end{Obs}
	We now define the concepts of orbit and stabilizer of a G set.
	\begin{Def}
		Let $X$ be a $G$ set and $x\in X$. the set $O_x=\{a(g,x); \, \forall g\in G\}$ is called \textit{orbit} of $x\in X$. We define the $\textit{stabilizer}$ of $x\in X$ the set $G_x=\{g\in G|a(g,x)=x\}$. Sometimes we refer to the stabilizer of a point $x$ also with $Stab_x$.
	\end{Def}
	\begin{Obs} \label{Obs: 2.1}
		One can easily verify that two orbits are either equal or disjoint. To see this, take $O_x$ and $O_y$ two orbits, and a point $z\in O_x\cap O_y$, we can write:
		$$z=a(g,x)=a(g',y)$$
		Then $x=a(g^{-1}g',y)$, so the orbits coincide.
		From this it follows that $X$ is a disjoint union of the orbits of its points: $X=\bigsqcup O_x$.
	\end{Obs}
	\begin{Def}
		The action of a group $G$ on a set $X$ is called \textit{transitive} if there is just only one orbit. This means that for $x,y\in X$ it exists $g\in G$ such that $x=a(g,y)$.
	\end{Def}
	This thesis will mainly study structures in which groups act on manifolds. We thus give the following definition.
	\begin{Def}
		A manifold $M$ with a Lie group $G$ (left) right-acting on it is called a \textit{G-manifold}. We also say that a map $f:M\rightarrow N$ between $G-manifolds$ is equivariant if $f(x\cdot g)=f(x)\cdot g$.
	\end{Def}		
	\section{Lie groups}
	In this section we give the definition of Lie groups and subgroups, with some important results. More informations can be found in [1] pag. 164-174 chap. 4.
	\begin{Def}
		A \textit{Lie group} $G$ is a smooth manifold with a group structure, such that the operations $\cdot:G\times G\rightarrow G$ and inversion $i:G\rightarrow G$ are $C^\infty$.
		A \textit{subgroup} $H$ of a Lie group $G$ is a immersed submanifold of $G$ which is still a group under the induce operations from $G$.
	\end{Def}
	\begin{Ex}
		Consider $GL_n(\mathbb{R})$, which was proven to be a smooth manifold. Given the row-column multiplication and the inverse $i$ which associates to a matrix its inverse, those make $GL_n(\mathbb{R})$ a group. Those operations can also be shown to be $C^\infty$ and so $GL_n(\mathbb{R})$ is a Lie group. To see the full proof consult [1] chap. 1 pag. 66.
	\end{Ex}
	\begin{Ex}
		Consider $SL_n(\mathbb{R})=\{A\in M_{n\times n}| \det(A)=1\}$. since $SL_n(\mathbb{R})=\det^{-1}(1)$ we have that $SL_n(\mathbb{R})$ is a closed subgroup of $GL_n(\mathbb{R})$. It is also possible to show that ([1] pag. 105 cap. 3, Theorem 9.9) $SL_n(\mathbb{R})$ is a regular submanifold $GL_n(\mathbb{R})$ of dimension $n^2-1$. Since the operations of multiplication and inversion are still $C^\infty$ on $SL_n(\mathbb{R})$, we can say that this is a Lie subgroup of $GL_n(\mathbb{R})$.\\
		\\
		To find more informations on this example the reader can see [1] pag. 105-107, 125, 165 chap. 3 and 4.\\
		\\
		It is possible to show that (indicating with $^+$ the adjoint and with $T$ the transposition):\\\\
		$O_n(\mathbb{R})=\{A\in M_{n\times n}(\mathbb{R})\, |\, A^TA=AA^T=\mathbb{I}\};$\\\\
		$SO_n(\mathbb{R})=\{A\in M_{n\times n}(\mathbb{R})\, |\, A^TA=AA^T=\mathbb{I};\, \det(A)=1\};$\\\\
		$U_n(\mathbb{R})=\{A\in M_{n\times n}(\mathbb{R})\, |\, A^+A=AA^+=\mathbb{I}\};$\\\\
		$SU_n(\mathbb{R})=\{A\in M_{n\times n}(\mathbb{R})\, |\, A^+A=AA^+=\mathbb{I};\,\det(A)=1\};$\\\\
		are all real Lie groups. 
		For the full proofs the reader is advised to check [5] pag. 41-51 chap. 2.
	\end{Ex}
	\begin{Def}
		We call \textit{homomorphism between Lie groups} $G$ and $H$ a smooth map $f:G\rightarrow H$ such that $f(gh)=f(g)f(h)$.
	\end{Def}
	We note that an homomorphism between groups sends the identity into the identity. In fact $f(eg)=f(e)f(g)$ ed $f(g)=f(eg)$.\\\\
	We indicate with $l_g:G\rightarrow G$ where $l_g(x)=gx$ is the left multiplication on $G$ Lie group.
	\begin{Obs}
		$l_g$ is $C^\infty$ since it corresponds to the multiplication of a Lie group. Since this has a clear smooth inverse $l_{g^{-1}}$, it is a diffeomorphism.
	\end{Obs}
	\section{Lie algebras}
	In this section we define Lie algebras and left invariant vector fields, with some examples. For more details see [1] chap. 4, [5] chap. 2 pag. 46.
	\begin{Def}
		We call \textit{Lie algebra} a vector space $\mathfrak{g}$ endowed with a binary operation called Lie bracket $[,]:\mathfrak{g}\times\mathfrak{g}\rightarrow\mathfrak{g}$ such that it respects:
		\begin{itemize}
			\item Skew-symmetry: $[x,y]=-[y,x]$, $\forall x,y\in\mathfrak{g}$;
			\item Jacobi identity: $[x,[y,z]]+[y,[z,x]]+[z,[x,y]]=0$, $\forall x,y,z\in \mathfrak{g}$ 
		\end{itemize}
		We call \textit{Lie subalgebra} of $\mathfrak{g}$ a subspace $\mathfrak{h}\subset\mathfrak{g}$ closed with respect to $[,]$.\\
		We define an \textit{homomorphism between lie algebras} as a linear map $f:\mathfrak{g}\rightarrow\mathfrak{h}$ such that $f([x,y])=[f(x),f(y)]$ for all $x,y\in\mathfrak{g}$.
	\end{Def}
	In general, we will only consider finite-dimensional Lie algebras.
	\begin{Ex} \label{Obs: bracket Mnn}
		Consider $M_{n\times n}(\mathbb{R})$ vector space of $n\times n$ matrices on $\mathbb{R}$. We define a bracket in the follwoing way: given $A,B\in M_{n\times n}(\mathbb{R})$, $[A,B]:= AB-BA$. Clearly, this definition makes $[,]$ bilinear and skew-symmetric. As for the Jacobi identity, it is a straightforward calculation to verify it. This makes $M_{n\times n}(\mathbb{R})$ a Lie algebra. 
	\end{Ex}
	\begin{Ex}
		Let $V$ be a generic real vector space. If we define $[,]:V\times V\rightarrow V$ like $[x,y]=0$, $V$ clearly becomes a Lie algebra, called \textit{trivial}.
	\end{Ex}
	\section{Lie algebras of Lie groups}
	In this section we define the concept of Lie algebra of a Lie group. We are going to see that it is possible to identify the Lie algebra of any Lie group with its tangent space at the identity. For more details the reader is advised to see [1] pag 178-182 cap. 4, [6] cap. 1 e [7] part 1 cap.3.
	\begin{Def}
		A vector field $X$ on a Lie group $G$ is called \textit{left invariant} if $dl_g(X)=X$ for all $g\in G$.
	\end{Def}
	A vector field is left invariant if it's translation induced by the left multiplication leaves it invariant: which means $dl_g(X_h)=X_{gh}$ for all $g,h\in G$.
	It follows that $X_g=dl_g(X_e)$, this means that a left invariant vector field is completely determined by its value at the identity. It is also possible to show that any left invariant vector field is smooth. ([1] pag. 181 cap.4).
	\begin{Obs} \label{Obs: 2.2}
		Given a Lie group $G$ and a vector $A\in T_eG$ tangent to the identity, we can define a vector field $X^A:G\rightarrow T_eG$ by setting $X^A_g=dl_g(A)$. This field is left-invariant since $$dl_g(X^A_h)=dl_g(dl_h(X^A_e))=dl_{gh}(X^A_e)=dl_{gh}(A)=X^A_{gh}$$
		We say that $X^A$ is generated by $A\in T_eG$ and call $Lie(G)$ the set of all $X^A$ generated from vectors tangent to the identity of $G$.	
	\end{Obs}
	Given two vector fields $X$ and $Y$, we define their Lie bracket in a point $p$ as:
	$$[X,Y]_pf=(X_pY-Y_pX)f$$
	It is possible to show that if $X$ and $Y$ are smooth vector fields on $M$, then also $[X,Y]$ is smooth on $M$. More details on the proof can be found in [1] pag. 157 cap. 4. The following result also holds:
	\begin{Prop}
		If $X$ and $Y$ are left invariant, then also $[X,Y]$, $cX$ and $X+Y$ are. Moreover, $[X,Y]$ respects the Jacobi identity. Thus, $Lie(G)$ is a Lie algebra.
	\end{Prop}
	\begin{proof}
		The proof of this proposition is just a straightforward calculation and can be found in [1] chap. 4 pag. 182,183.
	\end{proof}
	Thus, $Lie(G)$ is a Lie algebra called \textit{Lie algebra of $G$} and it is referred to as $\mathfrak{g}$.
	\begin{Theo}
		Given any Lie group $G$, there exists an isomorphism between $\mathfrak{g}$ and $T_eG$, so that $\mathfrak{g}\cong T_eG$.
	\end{Theo}
	\begin{proof}
		The proof of this theorem goes too far from the topic of this thesis. The interested reader can find it in [5] pag. 51 chap. 2.
	\end{proof}
	\begin{Obs}\label{Obs: bracket T}
		We can define a product $[,]$ on the tangent space at the identity in the following way: given two vectors $A,B\in T_eG$, we set $[A,B]=[X^A,X^B]_e$, where $X^A$ and $X^B$ are the left-invariant vector fields generated by the chosen vectors. It is possible to prove that $X^{[A,B]}=[X^A,X^B]$. 
		For the full proof see [1] pag. $183$ chap. $4$.
	\end{Obs}
	\begin{Def}
		We say that a Lie algebra $\mathfrak{g}$ is \textit{compact} if it is the Lie algebra of a compact Lie group.
	\end{Def}
	We now give some examples of tangent spaces to the identity for matrix groups, recalling that we can identify these spaces with the Lie algebra of the group itself.
	\begin{Ex}
		We call $\mathfrak{gl_n(\mathbb{R})}$ the Lie algebra of $GL_n(\mathbb{R})$.
		By observation \ref{Obs 1.1.3}, we have $T_\mathbb{I}GL_n(\mathbb{R})=T_\mathbb{I}M_{n\times n}=M_{n\times n}$. From this, $$\mathfrak{gl_n(\mathbb{R})}\cong M_{n\times n}= T_\mathbb{I}GL_n(\mathbb{R})$$
		It is also possible to prove that the bracket defined in observation \ref{Obs: bracket Mnn} coincides with the one of observation \ref{Obs: bracket T}.
	\end{Ex}
	\begin{Ex}
		We call $\mathfrak{sl_n(\mathbb{R})}$ the Lie algebra of $SL_n(\mathbb{R})=\{A\in M_{n\times n}|\det(A)=1\}$.
		Consider a curve $\gamma:]-\epsilon,\epsilon[\rightarrow M_{n\times n}$ such that $\gamma(t)=\mathbb{I}+tB+O(t^2)$, it is necessary that $\det(\mathbb{I}+tB+O(t^2))=1$. However, $\det(\mathbb{I}+tB+O(t^2))=1+\hbox{tr}(B)+O(t^2)$ from which it follows: $$T_\mathbb{I}SL_n(\mathbb{R})\cong \mathfrak{sl_n(\mathbb{R})}=\{A\in M_{n\times n}|\hbox{tr}A=0\}$$ 
	\end{Ex}
	\begin{Ex}
		Let $GL(V)=\{f:V\rightarrow V|\, \textnormal{$f$\,  is \, linear \, and \, invertible}\}$ where $V$ is a finite dimensional vector space. We indicate with $\mathfrak{gl}(V)$ the Lie algebra of $GL(V)$.  Then, having fixed a basse for $V$, we have the identification $GL(V)\cong GL_n(\mathbb{R})$. The tangent space to $GL_n(\mathbb{R})$ was proven to be $M_{n\times n}(\mathbb{R})\cong End(V)=\{f:V\rightarrow V|\textnormal{\, $f$\, is \, linear}\}$. From this follows that: 
		$$\mathfrak{gl}(V)\cong T_{id}GL(V)\cong End(V)$$ 
	\end{Ex}
	\begin{Obs}
		Consider a Lie algebra $\mathfrak{g}$ with a basis of generators indexed like $\{T^a\}$. Those generators will have some commutation relations of the form:
		$$[T^a,T^b]=f^{ab}_{\hspace{9pt}c}T^c$$
		where we have used Einstein's notation for the sum of contracted indices. We call the numbers $f^{ab}_{\hspace{9pt}c}$ \textit{structure constants}. Those completely define the commutators between all elements of the Lie algebra, since any element is a linear combination of the generators.
	\end{Obs}
	\begin{Obs}
		Since the left translation $l_g$ is a diffeomorphism, its differential is an isomorphism of vector spaces:
		$$dl_g:T_eG\rightarrow T_gG$$
		Thus, we can always express a tangent vector at $g\in G$ as  a left invariant vector field at that point.
	\end{Obs}
	\section{The exponential map}
	In this section we will give the definition of exponential map and look at some properties of it. We will focus on matrix Lie groups since those will be the main object of gauge theories.
	To find more details the reader is advised to check [7] part 1, chap 2-2.4 and 3.7.
	\begin{Def}
		Given $G$ Lie group and $X\in\mathfrak{g}$, the \textit{exponential map} \\$exp:\mathfrak{g}\rightarrow G$ is defined as $exp(X)=\Phi_X(1,e)$ the flux of the field $X$.
	\end{Def}
	We will also refer to $exp(X)$ with $e^X$.
	\begin{Def}
		We call \textit{exponential} of a matrix $X\in M_{n\times n}$ the series \\$e^X=\sum_{n=0}^{\infty}{X^n\over n! }$ where $X^n$ is the $n^{th}$ product of $X$ with itself.
	\end{Def}
	It can be shown that this series converges for all $X\in M_{n\times n}$ and that it is a continuous function of $X$. Moreover, the following holds:
	\begin{itemize}
		\item $e^0=I$;
		\item $(e^X)^+=e^{X^+}$ where $^+$ indicates the adjoint;
		\item $e^X$ is invertible and $(e^X)^{-1}=e^{-X}$;
		\item $e^{(a+b)X}=e^{aX}e^{bX}$ for all $a,b\in\mathbb{R}$;
		\item if $XY-YX=0$ then $e^{X+Y}=e^Xe^Y=e^Ye^X$ for all $X,Y$ matrices;\\
		\item if $C$ is an invertible matrix, then $e^{CXC^{-1}}=Ce^XC^{-1}$
	\end{itemize} 
	However the most important results are: ${d\over dt}e^{tX}\bigg{\rvert}_0=X$ and ${d\over dt}e^{tX}\bigg{\rvert}_{t'}=Xe^{t'X}$.\\
	In fact, this ensures that the exponential map coincides with the exponential of a matrix. For the proofs see [t] pag. 38-41 chap.2.\\
	We now look at an important proposition, which will allow us to establish a relation between homomorphisms of Lie algebras and Lie groups.
	\begin{Prop} \label{Prop: 2.4.1}
		Let $G$ and $H$ be matrix Lie groups and $\mathfrak{g}$, $\mathfrak{h}$ their algebras. let $\rho:G\rightarrow H$ a Lie group homomorphism. Then there is a unique linear map $q:\mathfrak{g}\rightarrow\mathfrak{h}$ such that:
		\begin{itemize}
			\item $\rho(e^X)=e^{q(X)}$; 
			\item $q(AXA^{-1})=\rho(A)q(X)\rho(A)^{-1}$;
			\item $q([X,Y])=[q(X),q(Y)]$, is a Lie algebra homomorphism;
			\item $q(X)={d\over dt}\bigg{\rvert}_0\rho(e^{tX})$.
		\end{itemize}
	\end{Prop}
	\begin{proof}
		For the full proof see [47] pag. 67 chap. 3.
	\end{proof}
	Thus, given a homomorphism between Lie groups it is always possible to construct a homomorphism between their Lie algebras.
	\section{Representations of Lie algebras and groups}
	In this section we define the concept of representations of Lie groups and algebras and look at some properties of them. We will also see how those two objects can be related to one another. For more details see [7] part 1, chap 4.
	\begin{Def}
		Given a group $G$ and a vector space $V$ of finite dimension, we call \textit{representation} a homomorphism between groups $\rho:G\rightarrow GL(V)$. In other words, we ask for $\rho(gh)=\rho(g)\circ\rho(h)$ for every $g,h\in G$.\\
		If $G$ is a Lie group, we define a representation as a homomorphism between Lie groups $\rho:G\rightarrow GL(V)$. With \textit{dimension of the representation} we refer to the dimension of the vector space $V$.
	\end{Def}Let us now see some easy examples:
	\begin{Ex}
		Let $G$ be a matrix Lie group. We define $\rho: G\rightarrow GL_n(\mathbb{C})$ such that $\rho(X)=I$, then $\rho$ is a representation, called \textit{trivial}. 
	\end{Ex}
	\begin{Ex}
		Let $G$ be a matrix Lie group. By definition $G\subset GL_n(\mathbb{C})$. We define $\rho: G\rightarrow GL_n(\mathbb{C})$ as $\rho(X)=X$, then $\rho$ is a representation, called \textit{standard}. 
	\end{Ex}
	\begin{Def}
		Given a Lie group representation $\rho$ acting on $V$, a subspace $W\subset V$ is called \textit{invariant} if $\rho(g)w\in W$ for all $w\in W$.
		A representation in which all the invariant subspaces are trivial: $\{0\}$ and $V$; is called \textit{irreducible}.
	\end{Def}
	Let us now look at the definition of Lie algebra representation. Later on, we shall see that the representation of a Lie group are strictly related to the ones of its Lie algebra.
	\begin{Def}
		Let $\mathfrak{g}$ be a Lie algebra. A representation of $\mathfrak{g}$ is a morphism of Lie algebras $\pi:\mathfrak{g}\rightarrow End(V)$; where $V$ is a finite dimensional vector space. Thus $\pi([X,Y])=[\pi(X),\pi(Y)]$.
	\end{Def}
	We now look at a theorem which establishes a strong connection between those representations.
	\begin{Theo}
		Let $G$ be a matrix Lie group with $\mathfrak{g}$ Lie algebra of $G$ and let $\rho:G\rightarrow GL(V)$ be a representation. Then, there exists a unique induced representation on $\mathfrak{g}$, namely $d\rho:\mathfrak{g}\rightarrow \mathfrak{gl}(V)\equiv End(V)$ such that $\rho(e^{tX})=e^{d\rho(X)}$ and $d\rho(X)={d\over dt}\rho(e^{tX})|_0$.\
		Moreover: $d\rho(AXA^{-1})=\rho(A)d\rho(X)\rho(A)^{-1}$.	
	\end{Theo}
	\begin{proof} [Idea of proof]
		After fixing a basis, we can identify $GL(V)$ with $GL_n(\mathbb{C})$ (or $\mathbb{R}$), then, if we assume the hypothesis of the theorem, $\rho$ becomes a homomorphism between matrix Lie groups. By proposition \ref{Prop: 2.4.1} there exists a unique map $\varphi:\mathfrak{g}\rightarrow End(V)$ which coincides with $d\rho$ and respects all of the needed properties.
	\end{proof}
	\begin{Theo}\label{Theo: 3.1}
		Let $G$ a connected matrix Lie group, $\mathfrak{g}$ its Lie algebra, $\rho:G\rightarrow GL(V)$ a representation of $G$ and $d\rho:\mathfrak{g}\rightarrow \mathfrak{gl}(V)$ the induced representation. Then $\rho$ is irreducible if and only if $d\rho$ is.
	\end{Theo}
	For the full proof see [7] pag. 87 chap. 4.\\
	\\
	Thus, given a connected matrix Lie group and a representation, not only it is always possible to find a representation of its Lie algebra, but the irreducibility of one implies the irreducibility of the other.
	\begin{Obs}
		One could show that, for a simply connected matrix Lie group, there exists a bijection between representations of the group and of the associated algebra. More details in [7] pag. 106 cap. 4.
	\end{Obs}
	We now construct a very important representation: the adjoint representation. This will be the used very frequently in gauge theories since it controls the transformations of the gauge bosons.
	\begin{Def}
		Let $G$ be a matrix Lie group with Lie algebra $\mathfrak{g}$. The \textit{adjoint representation} is the map:
		$$\hbox{Ad}:G\rightarrow GL(\mathfrak{g})\hbox{ acting like } \hbox{Ad}_g(X)=gXg^{-1}$$
	\end{Def}
	For every $g\in G$, the map $\hbox{Ad}_g$ is clearly a homomorphism. The corresponding induced adjoint representation on the Lie algebra is:
	$$\hbox{ad}=d(\hbox{Ad}):\mathfrak{g}\rightarrow\mathfrak{gl(g)}$$
	and the action of this map is:
	$$\hbox{ad}_X(Y)=[X,Y]$$
	\begin{Obs}
		There is another way to look at the adjoint representation. We can define the \textit{conjugation} operator on a Lie group by composing the right and the left translations:
		$$c_g=l_g\circ r_g^{-1}=r_g^{-1}\circ l_g$$
		This is a map:
		$$c_g:G\rightarrow G\hbox{ acting like } c_g(x)=gxg^{-1}$$
		From this it is clear that the adjoint representation of a matrix lie group $G$ is just the differential of the conjugation operator.
		$$\hbox{Ad}_g=dc_g$$
	\end{Obs}
	\begin{Obs}
		Obviously, the adjoint representation of a Lie group of dimension $n$ is a representation of dimension $n$.
	\end{Obs}
	\section{Simple and semisimple Lie algebras}
	In this section we will introduce the notions of simple and semisimple Lie algebras. We will also look at some properties and a criterion to establish if a Lie algebra is simple. Semisimple Lie algebras will be extremely important for the construction of the lagrangians since they will allow us to use a positive definite scalar product. The reader can find more information on this topic on [3] chap. 2, [5] chap. 3 and 4, [7] part 1 pag. 60,61, part 2 pag. 167-172.
	\begin{Def}
		Let $\mathfrak{g}$ be a Lie algebra. We call \textit{ideal} of $\mathfrak{g}$ a subspace $\mathfrak{h}$ such that 
		$$[\mathfrak{g},\mathfrak{h}]\subset\mathfrak{h}$$
	\end{Def}
	Equivalently, the above condition can also be rewritten like:
	$$\hbox{ad}_\mathfrak{g}\mathfrak{h}\subset\mathfrak{h}$$
	\begin{Def}
		Let $\mathfrak{g}$ be a Lie algebra. We say that $\mathfrak{g}$ is:
		\begin{itemize}
			\item \textit{simple} if it is non abelian and has no non trivial ideals;
			\item \textit{semisimple} if it has no non 0 abelian ideals.
		\end{itemize}
	\end{Def}
	We now give two criterion to establish when a Lie algebra is simple.
	\begin{Theo}
		A Lie algebra $\mathfrak{g}$ is simple if and only if it is non abelian and its adjoint representation is irreducible.
	\end{Theo}
	\begin{proof}
		The proof is just a consequence of the definition of ideal.			
	\end{proof}
	\begin{Ex}
		Consider the Lie algebra $\mathfrak{sl}(2,\mathbb{R})=\{A\in M_{2\times 2}|\hbox{tr} A=0\}$. This has the following basis:
		$$E=
		\begin{pmatrix}
			0 && 1\\
			0 && 0
		\end{pmatrix};
		F=
		\begin{pmatrix}
			0 && 0\\
			1 && 0
		\end{pmatrix};
		H=
		\begin{pmatrix}
			1 && 0\\
			0 && -1
		\end{pmatrix}$$
		And the commutation relations between the generators are 
		$$[E,F]=H, [H,F]=-2F,[H,E]=2E$$
		The adjoint representation acts like:
		$$\hbox{ad}_{(aH+bE+cF)}(Y)=[aH+bE+cF,Y]$$
		The adjoint representation is of dimension equal to the generators of the group, so 3 in our case. One can show that the matrix corresponding to the element $X=aH+bE+cF$ is:
		$$\hbox{ad}_X=
		\begin{pmatrix}
			0 && -c && b \\
			-2b && 2a && 0 \\
			2c && 0 && -2a \\
		\end{pmatrix}$$
		And this has no non-trivial invariant subspaces. Thus, the adjoint representation is irreducible and the Lie algebra $\mathfrak{sl_n}(\mathbb{R})$ is simple.
	\end{Ex}
	\section{Scalar products on Lie algebras}
	In this section we will briefly study scalar products on Lie algebras, which will be fundamental in the further analysis on Gauge theories. For further references see [3] chap. 2.
	\begin{Def}
		Let $\braket{,}$ be a metric on a Lie group $G$. We say that the metric is:
		\begin{itemize}
			\item \textit{left-invariant} if $l^*_g\braket{,}=\braket{,}$;
			\item \textit{right-invariant} if $r^*_g\braket{,}=\braket{,}$;
			\item \textit{bi-invariant} if it is both right and left invariant.
		\end{itemize}
	\end{Def}
	It is clear that, given any metric $\braket{,}$ on $G$, we have an induced scalar product on $\mathfrak{g}$, given by the product of vector fields at the identity $\braket{,}_\mathfrak{g}=\braket{,}_e$. On the contrary, given a scalar product on the Lie algebra $\mathfrak{g}$, it is easy to construct:
	\begin{itemize}
		\item a left-invariant metric:
		$$\braket{X_g,Y_g}_g=\braket{dl_{g^{-1}}X_g,dl_{g^{-1}}Y_g}_\mathfrak{g}$$
		\item a right-invariant metric:
		$$\braket{X_g,Y_g}_g=\braket{dr_{g^{-1}}X_g,dr_{g^{-1}}Y_g}_\mathfrak{g}$$
	\end{itemize}
	Where $X_g,Y_g\in T_gG$ are tangent vectors at $g$.
	\begin{Prop}
		Let $\braket{,}$ be a left-invariant metric on a Lie group $G$. Then $\braket{,}$ is bi-invariant if and only if the scalar product defined on $\mathfrak{g}$ is $\hbox{Ad}$-invariant:
		$$\braket{X_e,Y_e}=\braket{\hbox{Ad}_g(X)_e,\hbox{Ad}_g(Y_e)}$$
		For every $X_e,Y_e\in\mathfrak{g}$ and $g\in G$.
	\end{Prop}
	\begin{proof}
		Let $X_g,Y_g\in T_gG$ be vectors tangent to $g\in G$. Suppose $\braket{,}$ is a left-invariant metric on $G$. Then:
		$$r_h^*\braket{X_g,Y_g}=\braket{dl_{(gh)^{-1}}\circ dr_h(X_g),dl_{(gh)^{-1}}\circ dr_h(Y_g)}=$$
		$$=\braket{\hbox{Ad}_{h^{-1}}\circ dl_{g^{-1}}(X_g),\hbox{Ad}_{h^{-1}}\circ dl_{g^{-1}}(Y_g)}$$
		This is clearly equal to $\braket{X_g,Y_g}=\braket{dl_{g^{-1}}(X_g),dl_{g^{-1}}(Y_g)}$ if and only if the action of the adjoint leaves the product unchanged.
	\end{proof}
	\begin{Theo} \label{Scal_prod_theo_1}
		On every compact Lie group $G$ there exists a bi-invariant positive definite metric.
	\end{Theo}
	\begin{proof}
		The proof of this result requires tools which are not explained in this thesis and can be found in [3] chap. 2 pag. 106-108.
	\end{proof}
	We now introduce the concept of Killing form. Our aim is to find an inner product on $\mathfrak{g}$ which is non degenerate and positive definite. We will see that those requirements will be satisfied by the Killing form of a compact Lie algebra. This product, once found, will be used to define the Lagrangians.
	\begin{Def}
		Let $\mathfrak{g}$ be a finite dimensional Lie algebra on $\mathbb{K}=\mathbb{R},\mathbb{C}$. We define the \textit{Killing form} of $\mathfrak{g}$ as the map:
		$$B_\mathfrak{g}:\mathfrak{g}\times \mathfrak{g}\rightarrow\mathbb{K}\hbox{ acting like } B_\mathfrak{g}(X,Y)=tr(\hbox{ad}_X\circ \hbox{ad}_Y)$$
	\end{Def}
	\begin{Obs}
		By definition, this map is clearly bilinear and symmetric. Note that in the case $\mathbb{K}=\mathbb{C}$ the Killing form is still symmetric and not hermitian.
	\end{Obs} 
	\begin{Prop}\label{Prop_2.8.2}
		The Killing form is invariant under the action of Lie algebra automorphism. In prticular, the Killing form is bi-invariant.
	\end{Prop}
	\begin{proof}
		Let $\sigma:\mathfrak{g}\rightarrow\mathfrak{g}$ be an automorphism. Then:
		$$\hbox{ad}_{\sigma X}(Y)=[\sigma X,Y]=\sigma([X,\sigma^{-1}Y])=\sigma\circ \hbox{ad}_X\circ \sigma^{-1}(Y)$$
		So that $\hbox{ad}_{\sigma X}=\sigma\circ \hbox{ad}_X\circ \sigma^{-1}$. The immediatly implies:
		$$B_\mathfrak{g}(X,Y)=B_\mathfrak{g}(\sigma X,\sigma Y)$$
		This completes the proof.
	\end{proof}
	We now look at some important results regarding the Killing form on semisimple Lie algebras.
	\begin{Theo}\label{Theo_2.8.2}
		Let $\mathfrak{g}$ be a compact Lie algebra. Then the Killing form is negative definite and non degenerate if and only if $\mathfrak{g}$ is semisimple.
	\end{Theo}
	\begin{proof}
		The proof of this theorem can be found in [3] chap. 2, pag. 114,115.
	\end{proof}
	\begin{Obs}
		Let $\mathfrak{g}$ be a compact semisimple Lie algebra. Then, we know by the previous theorem, that the Killing form is negative definite. By symmetry of this inner product, we can choose an orthonormal basis $\{T^a\}$ of vectors such that:
		$$B_{\mathfrak{g}}(T^a,T^b)=-\delta^{ab}$$
	\end{Obs}
	\begin{Prop}
		The structure constants of a compact semisimple Lie algebra for an orthonormal basis of $B_\mathfrak{g}$ are totally antisymmetric.
	\end{Prop}
	\begin{proof}
		Let $\{T^a\}$ be an orthonormal basis for the Killing form. Then, since $\{T^a\}$ are generators, $[T^a,T^b]=f^{ab}_{\hspace{9pt}c}T^c$ and the antisymmetry in the first two indices is a clear consequence of the antisymmetry of $[,]$. Now it remains to show that also the other index commutes. It is known that:
		$$\hbox{ad}_{\hbox{ad}_X}(Y)=\hbox{ad}_X\circ \hbox{ad}_Y-\hbox{ad}_y\circ \hbox{ad}_X$$
		From this property it directly follows that:
		$$B_\mathfrak{g}(\hbox{ad}_X(Y),Z)=\hbox{tr}(\hbox{ad}_{\hbox{ad}_X(Y)}\circ \hbox{ad}_Z)=\hbox{tr}(\hbox{ad}_X\circ \hbox{ad}_Y\circ \hbox{ad}_Z-\hbox{ad}_Y\circ\hbox{ad}_X\circ \hbox{ad}_Z)$$
		By linearity and ciclicity of the trace we find:
		$$B_\mathfrak{g}(\hbox{ad}_X(Y),Z)=\hbox{tr}(\hbox{ad}_Y\circ \hbox{ad}_Z\circ \hbox{ad}_X-\hbox{ad}_Y\circ\hbox{ad}_X\circ \hbox{ad}_Z)=-B_\mathfrak{g}(Y,\hbox{ad}_X(Z))$$
		Using this property we evalue:
		$$B_\mathfrak{g}([T^a,T^b],T^c)=B_\mathfrak{g}(\hbox{ad}_{T^a}(T^b),T^c)=f^{ab}_{\hspace{9pt}d}B_\mathfrak{g}(T^d,T^c)=$$
		$$=-B_\mathfrak{g}(T^b,\hbox{ad}_{T^a}(T^c))=-B_\mathfrak{g}(T^b,[T^a,T^c])=-f^{ac}_{\hspace{9pt}d}B_\mathfrak{g}(T^c,T^d)$$
		By orthonormality of the chosen basis we finally find:
		$$f^{ab}_{\hspace{9pt}d}\delta^{dc}=-f^{ac}_{\hspace{9pt}d}\delta^{bd}$$
		$$f^{abc}=-f^{acb}$$
		This completes the proof.
	\end{proof}
	\chapter{Bundles on smooth manifolds}
	In this chapter we will introduce fiber bundles, vector bundles and principal bundles, with some examples. We will see some basic results which will allow us to study connections in the following chapters. To find more informations about those topics one may check [2] chap. 1 pag. 49-59, chap. 2 pag. 71-86, chap. 3 pag. 95-110, chap. 5; [3] chap. 4 pag. 193-223.
	\section{Fiber Bundles}
	In this section we will introduce fiber and vector bundles and give some important examples. For more details on fiber bundles one can look at [2] chap. 6 pag. 242, [3] chap. 4 pag. 193-195.
	\begin{Def}
		Let $M, E$ and $F$ be manifolds. Let $\pi: E\rightarrow M$ be a map such that:
		\begin{itemize}
			\item $\pi$ is smooth, surjective and continuous;
			\item given any open set $U\subset M$ we can find a diffeomorphism $\varphi: \pi^{-1}(U)\rightarrow U\times F$ called \textit{local trivialization} such that the following diagram commutes:
		\end{itemize}
		\[
		\begin{tikzcd}
			\pi^{-1}(U) \arrow{r}{\varphi} \arrow[swap]{dr}{\pi} & U\times F \arrow{d}{Proj}\\
			& U 
		\end{tikzcd}
		\]
		Where $Proj: U\times F\rightarrow U$ is the standard projection. $Proj(x_M,x_F)=x_M$.\\
		We call $(E,M,\pi,F)$ a \textit{fiber bundle}, $M$ \textit{base space}, $E$ \textit{total space} and $F$ \textit{fiber}.
	\end{Def}
	\begin{Ex}\label{Ex_1.1}
		Let $M$, $F$ be manifolds. We will indicate with $p_M$ points in $M$ and with $p_F$ points in $F$.\\ 
		We set $E=M\times F$ and construct a projection in the obvious way: 
		$\pi:M\times F\rightarrow M$ like $\pi(p_M,p_F)=x_M$. This map is obviously continuous and surjective since it is a projection. Moreover, it is also smooth since given two charts $(U_M,\phi_M:U_M\rightarrow \mathbb{R}^n)$ on $M$ and $(U_F,\phi_F:U_F\rightarrow \mathbb{R}^k)$ on $F$, we have:
		$$\phi_M\circ \pi\circ (\phi_M^{-1}\times \phi_F^{-1}):\mathbb{R}^n\times\mathbb{R}^k\rightarrow\mathbb{R}^n,\hbox{ }\phi_M\circ \pi\circ (\phi_M^{-1}\times \phi_F^{-1})(x^i_M,x^j_F)=x^i_M$$
		Now it remains to construct local trivializations. Considering any chart on $M$ like $(U_M,\phi_M:U_M\rightarrow \mathbb{R}^n)$ we set
		$$\varphi:\pi^{-1}(U)\rightarrow U\times F$$
		such that $\varphi(p_M,p_F)=(p_M,p_F)$ acts like the identity map. This map is clearly diffeomorphic.
		This gives $M\times F$ the structure of a fiber bundle.
	\end{Ex}
	\begin{Ex} \label{Ex_1.2}
		An example of the product bundle is the cylinder. It can be constructed as a fiber bundle over the circle $S^1=\{z\in \mathbb{C}|z=e^{i\theta}, \theta\in[0,2\pi]\}$.\\
		We define the cylinder as $C=S^1\times \mathbb{R}$, recalling that both $S^1$ and $\mathbb{R}$ are manifolds. We now repeat the same construction we did in the previous example: we define the projection as: $\pi:C\rightarrow S^1$ that sends $\pi(z,t)\rightarrow z$, where $z\in S^1$ and $t\in\mathbb{R}$. By the same arguments as above, this map is smooth and surjective.\\As for the local trivializations, we can take any subset of the circle $U$ and trivialize the bundle with the identity:
		$\varphi:\pi^{-1}(U)\rightarrow U\times \mathbb{R}=\mathbb{I}_d$.\\
		In this case the fiber of the cylinder is a line. Intuitively, each point has associated a line and the totality of those lines composes the cylinder.\\
		\begin{figure}[H]
			\begin{center}
				\begin{tikzpicture}
					%cilinder and circle
					\draw (0,0) ellipse (1.5 and 0.5) node at (2,0) {$S^1$};
					\draw (0,2) ellipse (1.5 and 0.5);
					\draw (0,4) ellipse (1.5 and 0.5);
					\draw (-1.5,2)--(-1.5,4);
					\draw (+1.5,2)--(+1.5,4);
					\draw node at (-2.5,3) {C};
					%lines
					\draw[->] (-2,+2)--(-2,0) node[midway,left] {$\pi$};
					\draw (-1,2.372677996)--(-1,4.372677996) node at (-0.5,3) {]a,b[};
					\draw (+0.5,1.51)--(+0.5,3.51) node at (+1,2.7) {]a,b[};
				\end{tikzpicture}
				\caption{The cylinder as a fiber bundle}
			\end{center}	
		\end{figure}
	\end{Ex}
	\begin{Ex}\label{Ex_3.1.3}
		Let $S^1=U(1)$, $S^2=\{(z,x)\in\mathbb{C}\times\mathbb{R}||z|^2+x^2=1\}$ and $S^3=\{(z_1,z_2)\in\mathbb{C}^2||z_1|^2+|z_2|^2=1\}$. We define a projection:
		$$\pi:S^3\rightarrow S^2\hbox{ acting like: }\pi(z_1,z_2)=(2z_1z_2^*,|z_1|^2-|z_2|^2)$$
		This map is obviously well defined since:
		$$4|z_1|^2|z_2|^2+(|z_1|^2-|z_2|^2)^2)=1$$
		Note that, chosing a point $(z,x)$ on $S^2$, the pre-image of the projection is defined up to a phase:
		$$\pi(z_1,z_2)=\pi(\lambda z_1,\lambda z_2)\hbox{ where }|\lambda|^2=1$$
		In particular, all of the possible phases are elements of $S^1=U(1)$ so that a point in $S^2$ can be idenitified with a point in $S^2$ times a circle $S^1$. This suggests to construct a fiber bundle on $S^2$ with fiber $S^1$ and total space $S^3$. Note that this fiber bundle cannot be trivial since $S^2\times S^1\neq S^3$. To see this it suffices to see that the fundamental group of $S^2\times S^1$ is $\mathbb{Z}$ and the one of $S^3$ is trivial.
		It is known that $S^n$ is a smooth manifold and so the projection $\pi$ defined earlier is a smooth map. One can construct a local trivialization in the following way: take two open subsets $U_{1,2}\subset S^2$ defined like:
		$$U_1=S^2-(0+i0,1);U_2=S^2-(0+i0,-1)$$
		Those subsets cover every point in $S^2$. Then it is clear that the pre-images of those sets are:
		$$\pi^{-1}(U_1)=\{(z_1,z_2)|z_2\neq 0\}\hbox{ and }\pi^{-1}(U_1)=\{(z_1,z_2)|z_1\neq 0\}$$
		We now define two local trivializations:
		$$\varphi_1:\pi^{-1}(U_1)\rightarrow S^2\times S^1\hbox{ such that }\varphi(z_1,z_2)=(\pi(z_1,z_2),{z_2\over |z_2|})$$
		$$\varphi_2:\pi^{-1}(U_2)\rightarrow S^2\times S^1\hbox{ such that }\varphi(z_1,z_2)=(\pi(z_1,z_2),{z_1\over |z_1|})$$
		Those maps are smooth and one can define their inverses as:
		$$\varphi_1^{-1}((z,x),\lambda)=(z_1,z_2)\cdot\lambda{|z_2|\over z_2}$$
		$$\varphi_2^{-1}((z,x),\lambda)=(z_1,z_2)\cdot\lambda{|z_1|\over z_1}$$
		Where $(z_1,z_2)$ are two arbitrary elements of the pre-image of $(z,x)$. One can easily check that the previous definition do not depend on the choice of the element in the pre-image. Moreover, those two local trivializations clearly agree in $U_1\cap U_2$. This defines a fiber bundle called \textit{Hopf bundle}.
	\end{Ex}
	\section{Vector bundles}
	We now introduce the notion of vector bundles and sub-bundles, which will be of great importance. A vector bundle can be thought of as a fiber bundle, but with a vector space as a fiber. More details on vector bundles are to be found in [2] chap. 1 pag. 49-59, [3] chap. 4 pag. 225.
	\begin{Def}\label{Def_5.2}
		Let $E,M$ be manifolds. Let $\pi:E\rightarrow M$ be a smooth and surjective map such that:
		\begin{itemize}
			\item for every $p\in M$, the set $E_p=\pi^{-1}(p)$ is a vector space of dimension $k$;
			\item for every point $p\in M$ there is an open set $U\in M$ containing $p$ and a diffeomorphic fiber-preserving local trivialization $\varphi:\pi^{-1}(U)\rightarrow U\times \mathbb{R}^k$ that reduces to a linear isomorphism on each fiber:
			$$\varphi_p:E_p\rightarrow U\times \mathbb{R}^k$$
		\end{itemize}
		We call $(E,M,\pi,\mathbb{R}^k)$ a \textit{smooth vector bundle of rank $k$}.
	\end{Def}
	Sometimes, when it is not necessary to specify other structures, we will refer to a vector bundle $(E,M,\pi,\mathbb{R}^k)$ just with $E$.
	\begin{Ex}
		Consider a manifold $M$ of dimension $n$ and it's tangent bundle $TM$ with the induced $2n$-manifold structure from $M$. We have a projection $\pi:TM\rightarrow M$ defined as $\pi(p,v_p)=p$ which is continuous and surjective. 
		\\
		Before constructing the local trivializations, we need to show that this projection is smooth. This is done via charts. Consider a chart $(U,\phi)$ on $M$ and the corresponding $(TU,\Phi:TU\rightarrow \phi(U)\times \mathbb{R}^n)$. Then we can look at the composition:
		$$\phi\circ \pi\circ \Phi^{-1}:\phi(U)\times \mathbb{R}^n\rightarrow \mathbb{R}^n \hbox{ such that }$$ 
		$$\phi(\pi(\Phi^{-1}(x^i,a^i)))=\phi(\pi(p,v_p))=\phi(p)=x^i$$
		This is clearly smooth since $\phi$ and $\Phi$ are. This proves $\pi$ is smooth as well.\\
		\\
		Now we need to construct local trivializations. We wish for a diffeomorphism $\varphi:\pi^{-1}(U)\rightarrow U\times F$ where the fiber at each point is the tangent space $T_pM$. We also wish for this map to reduce, on each fiber, to a linear isomorphism.\\
		Given any chart on $TM$ like $(TU,\Phi)$, we take as our candidate for the local trivialization the map: 
		$$\varphi:(\phi^{-1}\times \mathbb{I}_{\mathbb{R}^n})\circ \Phi:TU\rightarrow U\times \mathbb{R}^n$$
		In particular we have the following succession:
		$$\varphi:TU\xrightarrow{\text{$\Phi$}} \phi(U)\times \mathbb{R}^n\xrightarrow{\text{$\phi^{-1}\times \mathbb{I}_{\mathbb{R}^n}$}}U\times \mathbb{R}^n$$
		This map is clearly an homeomorphism and a diffeomorphism since both the charts on manifolds and the identity are. As for the linear isomorphism reduction, consider the map:
		$\varphi_p$ where we fix the first argument $p$. The induced map is the following:
		$$\varphi_p(v_p)=\varphi(p,v_p)=(p,v^i_p\partial_i|_p)$$
		This is clearly a linear isomorphism between $\mathbb{R}^n$ and itself. This proves $(TM,M,\pi,\mathbb{R}^{2n})$ is a smooth vector bundle.
	\end{Ex}
	\begin{Def}
		Let $(E,M,\pi_E,\mathbb{R}^n)$ and $(F,N,\pi_F,\mathbb{R}^k)$ two smooth vector bundles. A \textit{smooth bundle homomorphism} is a pair of smooth maps $(f:E\rightarrow F,\bar{f}:M\rightarrow N)$ such that:
		\begin{itemize}
			\item The following diagram commutes:
			\[
			\begin{tikzcd}
				E \arrow{r}{f} \arrow{d}{\pi_E} & F \arrow{d}{\pi_F}\\
				M \arrow{r}{\bar{f}}&N 
			\end{tikzcd}
			\]
			\item on each fiber, $f$ resitricts to a linear map for each $p\in M$:
			$$f_p:E_p\rightarrow
			F_{\bar{f}(p)}$$
		\end{itemize} 
		If there exists another bundle homomorphism $(g:F\rightarrow E,\bar{g}:N\rightarrow M)$ such that
		$f\circ g=\mathbb{I}_E$ and $g\circ f=\mathbb{I}_F$, then we call the bundle homomorphism a \textit{bundle isomorphism}.
	\end{Def}
	\begin{Def}\label{Def_5.3}
		A vector bundle is said to be trivial if it is isomorphic to $M\times F$ the product bundle.
	\end{Def}
	It clearly follows that the product bundle is trivial and so it is also called the \textit{trivial bundle}.
	\begin{Def}
		Let $(E,M,\pi_E,\mathbb{R}^k)$ and $(H,M,\pi_H,\mathbb{R}^r)$ be vector bundles on $M$, with $r\leq k$. Then we say that $H$ is a \textit{smooth vector sub-bundle} of $E$ if:
		\begin{itemize}
			\item $H$ is a regular submanifold of $E$;
			\item the inclusion map $i:H\rightarrow E$ is a bundle homomorphism.
		\end{itemize}
	\end{Def}
	We now give a criterion to establish if a family of vector spaces over a manifold can be given the structure of a vector sub-bundle.
	\begin{Prop}
		Let $(E,M,\pi_E,\mathbb{R}^r)$ be a vector bundle of rank $r$ and $H=\bigsqcup_{p\in M}H_p$ a subset of $E$ such that, at any $p\in M$, $H_p$ is a $k$-dimensional vector subspace of the fiber $E_p$. If for every $p\in M$, there exist a neighborhood $U$ of $p$ and $m \geq k$ smooth sections $s_1,...,s_m$ of $E$ over $U$ that span $H_q$ at every point $q \in U$, then $F$ is a smooth subbundle of E.
	\end{Prop}
	\begin{proof}
		The proof of this proposition can be found in [2] chap. 4 pag 175-176.
	\end{proof}
	In general, one can combine different vector bundles to obtain other vector bundles. Consider the following result:
	\begin{Prop}
		Let $E,F$ be two vector bundles over $M$. Then the following sets can be given the structure of vector bundles:
		$$E\oplus F,\hspace{10pt}E\otimes F,\hspace{10pt}\bigwedge^k E,\hspace{10pt}Hom(E, F)$$.
	\end{Prop}
	\begin{proof}
		The proof can be found in [2] chap. 4 pag 180-184.
	\end{proof}
	\section{The tangent-cotangent isomorphism}\label{Sec_3.3}
		In this section we will show how to use a metric to establish a correspondence between the tangent and the cotangent bundle. We will see how we can use the metric tensor to raise and lower the indices of vectors and covectors. More information on this can be found in: [9] chap. 13 pag. 341, 342.
		\begin{Obs}
		It is known that, given a symmetric non degenerate bilinear map $g=\braket{,}$ on a vector space $V$, there is a canonical isomorphism between $V$ and its dual $V^*$:
		$$v\rightarrow\braket{v,}$$
		Selecting the canonical basis $\{e_i\}$ for $V$, the map $g$ will be a matrix with entries $g_{ij}$ and its action on the components of $v\in V$ will be given by the standard rows-columns product:
		$$v\rightarrow
		v^*\hbox{ such that: }=g_{ij}v^i=v_j$$
		where we have indicated with $v^i$ the components of $v\in V$ and with $v_i$ the components of $v^*\in V^*$ its dual. This operation allows us, given a metric, to relate the components of a vector on $V$, to the ones of it's dual $V^*$ through an operation that we call \textit{lowering} of the indices.\\
		Clearly, since we have an isomorphism, we will also have an inverse mapping, $g^{-1}$, whith components $g^{ij}$. This allows for the opposite operation: the \textit{raising} of indices of the components of any vector.\\
		\\
		In a metric manifold, we can smoothly raise and lower indices of vectors through the metric tensor $g$ in the same way.
	\end{Obs}
	\begin{Prop}
		Any metric on a smooth manifold provides a natural smooth isomoprhism between the tangent and the cotangent spaces.
	\end{Prop}
	\begin{proof}
		For each $p\in M$, given any $X\in T_pM$, we can define a unique co-vector: 
		$$X^*_p=g_p(X_p,\cdot)$$
		The map $X\rightarrow X^*$ is a bundle homomorphism since: $g(X,\cdot )(Y)=g(X,Y)$ at any $p\in M$.
		In any coordinates, expanding $X=X^i\partial_i;X^*=X^*_jdx^j$, the components of $X^*_p$ will be given by the lowering action of the metric:
		$$X^*_{p,j}=g_{p,ij}X_p^i$$
		We can make this bundle homomoprhism into a bundle isomorphism by considering the inverse. At any point, the map $g_p$ is bilinear, non degenerate and symmetric and so it is invertible. In any coordinate, we have the following inverse mapping:
		$$\omega_p\rightarrow\omega^i\partial_i\hbox{ for any }\omega=\omega_jdx^j{\in T_pM^*}$$
		where $\omega^i=g^{ij}\omega_j$ and $g^{ij}$ is the inverse of $g_{ij}$. By definition of the metric tensor, this mapping is smooth as well. 
	\end{proof}
	\begin{Obs}
		Note that, for a vector space $V$ endowed with a non degenerate bilinear symmetric map $g=\braket{,}$, we can define an induced map with the same properties on the dual space. For $v^*,w^*\in V^*$, define:
		$$g^*(v^*,w^*)=g(v,w)$$
		In coordinates, this means:
		$$g(v_ie^i,w_je^j)=g(v^ie_i,w^je_j)=g^{ij}v_iw_j=g_{ij}v^iw^j$$
		In a smooth manifold, this construction induces a smooth inner product on 1-forms.
	\end{Obs}
	\section{Sections and frames}
	In this section we introduce the notions of sections and frames of a vector bundle. We will also find a criterion to check if a smooth vector bundle is trivial. More details can be found in [2] chap.1. pag. 51,52.
	\begin{Def}\label{Def_5.5}
		Let $(E,M,\pi,\mathbb{R}^n)$ be a vector bundle. We say that a map \\$s:U\subset M\rightarrow E$ is a \textit{section} if $\pi\circ s=\mathbb{I}_d$. We denote the space of all smooth sections with $\Gamma(E)$.
		%We say that a section is \textit{global} if it is defined over the entire manifold.
	\end{Def}
	\begin{Ex}
		Consider the tangent bundle $TM$ of a manifold $M$. A section is a map that associates to each point a tangent vector. $s:M\rightarrow TM$. Note that every vector field is a section of the tangent bundle, since it acts in the same way: it takes a point and associates a vector to it. Thus, smooth vector fields are smooth sections of the tangent bundle and vice-versa. We call the space $\Gamma(TM)=\mathfrak{X}(M)$.
	\end{Ex}
	\begin{Ex}
		Consider the smooth vector bundle $\bigwedge^kTM^*$. Then, a section of this bundle is a map that associates to each point of $M$ an element in the total space $\bigwedge^kTM^*$. This means that the sections are exactly the $k$-forms. Thus, $\Omega^k(M)=\Gamma(\bigwedge^kTM^*)$.
	\end{Ex}
	\begin{Def}\label{Def_5.6}
		Let $(E,M,\pi,\mathbb{R}^k)$ be a vector bundle. We define a \textit{frame} on an open set $U\subset M$ as a collection of sections $\{s_i\}$ such that $\{s_i(p)\}$ form a basis for the fiber $F$ at each $p$. 
	\end{Def}
	\begin{Prop}\label{Prop_2.3.1}
		A smooth vector bundle is trivial if and only if it has a smooth frame.
	\end{Prop}
	\begin{proof}
		Let $E$ be trivial. This means there exists a diffeomorphic trivialization $\varphi:E\rightarrow M\times \mathbb{R}^k$. Let now $\{e_i\}$ be a basis for $\mathbb{R}^k$. Then, $\{(p,e_i)\}$ forms a basis of $\{p\}\times \mathbb{R}^k$.
		\\\\
		Now we check the contrary: suppose we have a smooth frame for $E$, indexed with $\{e_i\}$. This means that every point $e\in E$ can be expressed as a linear combination:
		$$e=a^ie_i$$
		Consider the mapping $\phi:E\rightarrow M\times\mathbb{R}^k$ acting like:
		$$\phi(e)=(\pi(e),a^1,...,a^k)$$
		This has a clear inverse: $\phi^{-1}(\pi(e),a^1,...,a^k)=a^ie_i$.
		It is clear that at each point $p$ this map reduces to a linear isomorphism.
	\end{proof}
	\begin{Ex}
		Consider the smooth manifold $\mathbb{R}^2$ and its tangent bundle $T\mathbb{R}^2$. At any point $(x,y)\in\mathbb{R}^2$ we can find a frame: $\partial_x,\partial_y$. This is a global smooth frame. By proposition $\label{prop_2.3.1}$, the tangent bundle of $\mathbb{R}^2$ is thus trivial.
	\end{Ex}
	\begin{Ex}
		Consider a generic Lie group $G$. We can find a frame for the vector fields at the identity by choosing a basis for the Lie algebra $\{T^a\}$ and considering the corresponding left-invariant vector fields $\{\tilde{T^a}\}$. Consider then the left-multiplication $l_g:G\rightarrow G$ on $G$. This is a diffeomorphism and so it induces an isomorphism of tangent spaces:
		$$dl_g:T_eG\rightarrow T_gG$$
		Any tangent vector at any $g\in G$ can be expressed as a linear combination of
		$$dl_g\tilde{T^a}$$
		This means that choosing a frame at the identity corresponds to choosing a global frame. Thus, every Lie group has a trivial tangent bundle. 
	\end{Ex}
	\begin{Obs}
		Since every bundle is locally trivial, from proposition \ref{Prop_2.3.1} we can say that choosing a local smooth frame corresponds to choosing a local trivialization.
	\end{Obs}
	\section{Vector valued forms}
	In this section we will introduce the concept of vector valued forms. Those will be extremely important in the construction of the Yang-Mills lagrangian, since the vector potential will be a vector valued form. More informations on this topic can be found in [2] chap. 4 pag. 186-197.
	\\\\
	One can think of vector valued forms as differential forms that take value on a generic vector space instead of $\mathbb{R}$. The construction is purely algebraic.
	\begin{Def}
		Let $W,V$ be vector spaces of finite dimension. We define a $V$-valued \textit{$k$-covector} on $W$ as an element of $Hom(\bigwedge^k W,V)$.
	\end{Def}
	It is known from linear algebra that we have the following isomorphism:
	$$Hom(\bigwedge^k W,V)\simeq \bigwedge^k W^*\otimes V$$
	A $V$-valued $k$-form on a manifold $M$ is a map that associates to each point a $V$-valued $k$-covector in $T_pM$. We can also think of $V$-valued $k$-form as section of some vector bundle: let $V$ be a finite dimensional vector space and $(E,M,\pi,\mathbb{R}^n)$ be a smooth vector bundle on $M$. Then, we can consider the product bundle $M\times V$, and define a new vector bundle $E\otimes (M\times V)$ which we will indicate with $E\otimes V$.
	\begin{Def}
		A $V$-valued $k$-form on $M$ is a section of the vector bundle $\bigwedge^k TM^*\otimes V$. The space of smooth $k$-forms is labeled as:
		$$\Omega^k(M,V)=\Gamma(\bigwedge^k TM^*\otimes V)$$
	\end{Def}
	Clearly, given a basis $\{v_i\}$ for $V$, we can express any $V$-valued $k$-form on $M$ as a linear combination:
	$$\omega=\omega^i\otimes v_i \hbox{ and } \omega^i\in \bigwedge^k TM^*$$
	where obviously we imply the summation on contracted indices. \\
	We can also define a product between $V$-valued $k$-forms. Let $V,W,U,Z$ be vector spaces. It is known from linear algebra that, if $\omega\in Hom(\bigwedge^kV,W)$ and $\eta\in Hom(\bigwedge^lV,U)$ are covectors and $\mu:W\times U\rightarrow Z$ is a multilinear map, we have a natural induced multilinear alternating map:
	$$\omega\cdot \eta(v_1,...,v_{k+l})=\sum_\sigma sgn(\sigma)\mu(\omega(v_{\sigma(1)}),...,v_{\sigma(k)},\eta(v_{\sigma(k+1)}),...,v_{\sigma(k+l)})$$
	We can use this map to define a generic product between forms:
	\begin{Prop}
		Let $\{v_i\}$ be vectors in $V$ and $\{w_i\}$ be vectors in $W$ and let $M$ be a smooth manifold. Let $\omega=\omega^i\otimes v_i\in\Omega^k(M,V)$ and $\eta=\eta^i\otimes w_i\in\Omega^l(M,W)$ be vector valued forms. Then, for any multilinear map $\mu:V\times W\rightarrow Z$ there is a naturally induced multilinear alternating map:
		$$\omega\cdot \eta=\omega^i\wedge\eta^j \mu(v_i,w_j)$$
	\end{Prop}
	\begin{proof}
		The proof is just a straightforward calculation and can be found in: [2] chap. 4 pag 189.
	\end{proof}
	\begin{Ex}\label{Ex_3.5.1}
		In the case of forms with value on a Lie algebra $\mathfrak{g}$, the multilinear map $\mu$ coincides with the commutator $[,]$, so that if $\{T_i\}$ are elements of $\mathfrak{g}$ and $\omega=\omega^i\otimes T_i\in\Omega^k(M,\mathfrak{g})$ and $\eta=\eta^i\otimes T_i\in\Omega^l(M,\mathfrak{g})$
		$$[\omega,\eta]=\omega^i\wedge\eta^j[T_i,T_j]$$
		Clearly, by the properties of the exterior power and of the commutator:
		$$[\omega,\eta]=(-)^{kl+1}[\eta,\omega]$$
	\end{Ex}
	\begin{Obs}
		We can extend the notion of exterior derivative to vector valued forms. In particular, given an expansion $\omega=\omega^i\otimes v_i$ where $\{v_i\}$ is a basis for $V$, we define:
		$$d\omega=d\omega^i\otimes v_i$$
	\end{Obs}
	\begin{Prop}
		The exterior derivative on vector valued forms is an antiderivation of degree 1.
	\end{Prop}
	\begin{proof}
		Let $\{v_i\}$ be a basis for $V$ and $\{w_i\}$ be a basis for $W$. As usual, let $\omega=\omega^i\otimes v_i$ be a $k$-form and $\eta=\eta^i\otimes w_i$ be a $l$-form. Then:
		$$d(\omega\cdot \eta)=d(\omega^i\wedge\eta^j\mu(v_i,w_j))=d(\omega^i\wedge\eta^j)\mu(v_i,w_j)$$
		By the fact that the exterior derivative is an antiderivation of degree 1 for the exterior product, it immediatly follows:
		$$d(\omega\cdot \eta)=d\omega\cdot \eta+(-)^{deg(\omega)}\omega\cdot d\eta$$
	\end{proof}
	We can also extend the definition of pullback to vector valued forms, in the natural obvious way:
	\begin{Def}
		Let $\alpha\in\Omega^k(M,V)$ be a $V$-valued form on $M$ and $f:N\rightarrow M$ be a smooth map between manifolds. We define the \textit{pullback} of $\alpha$ as:
		$$f^*\alpha_p(u_1,...,u_k)=\alpha_p(dfu_1,...,dfu_k)$$
		where $u_i\in T_pN$.
	\end{Def}
	There are some straightforward properties of the pullback:
	\begin{Prop}\label{Prop_3.4.3}
		Let $f:N\rightarrow M$ be any smooth map between manifolds and $\mu:V\times W\rightarrow Z$ be any multilinear map between vector spaces. Then, if $\alpha\in\Omega^k(M,V),\beta\in\Omega^l(M,W)$, we have:
		\begin{itemize}
			\item $f^*\alpha=f^*\alpha^i\otimes v_i$;
			\item $f^*(\alpha\cdot \beta)=(f^*\alpha)\cdot (f^*\beta)$; the pullback commutes with the product;
			\item the pullback commutes with the exterior derivative:
			$$f^*d\alpha=df^*\alpha$$
		\end{itemize}
	\end{Prop}
	\begin{proof}
		To prove the first claim we simply expand:
		$$f^*\alpha_p(u_1,...,u_k)=f^*(\alpha^i\otimes v_i)(u_1,...,u_k)=(\alpha^i\otimes v_i)(dfu_1,...,dfu_k)=\alpha^i(dfu_1,...,dfu_k)\otimes v_i$$
		Where $u_1\in T_pN$. To prove the second claim we recall that, having chosen some basis $\{v_i\}$ for $V$ and $\{w_i\}$ for $W$:
		$$(\alpha\cdot\beta)=\alpha^i\wedge\beta^j\mu(v_i,w_j)$$
		Then it is clear that:
		$$f^*(\alpha\cdot\beta)=f^*(\alpha^i\wedge\beta^j)\mu(v_i,w_j)=(f^*\alpha^i\wedge f^*\beta^j)\mu(v_i,w_j)$$
		The last property follows immediately from the fact that for real valued forms the pullback commutes with the exterior derivative.
	\end{proof}
	\begin{Ex}\label{Ex_3.4.2}
		Let $G$ be a Lie group. We define the \textit{Maurer–Cartan form} $\theta$ as:
		$$\theta_g(X_g)=dl_{g^{-1}}X_g$$
		This is obviously a $\mathfrak{g}$ valued $1$-form. One can prove the following:
		$$r_g^*\theta=Ad(g^{-1})\theta$$
		This is a straightforward calculation:
		$$(r_g^*\theta)_{h}(X_{h})=\theta_{gh}(dr_gX_h)=dl_{(gh)^{-1}}dr_gX_h=Ad(g^{-1})dl_{h^{-1}}X_h$$
		Moreover, one can prove that:
		$$d\theta+{1\over 2}[\theta,\theta]=0$$
		To prove this, we can use the fact that:
		$$d\theta(X,Y)=X\theta(Y)-Y\theta(X)-\theta([X,Y])$$
		The proof of this claimed property can be found in [1] chap. 5 pag. 232, 233. Since $dl_g$ is an isomorphism, it is sufficient to prove the claim for left invariant vector fields at any point. Let thus $X_g,Y_g$ be left-invariant, then:
		$$d\theta_g(X_g,Y_g)=Y_g\theta_g(X)-X\theta_g(Y_g)-\theta([X_g,Y_g])$$
		The first two terms are 0 since the Maurer-Cartan form applied to a left-invariant vector field is a constant function of the points. As for the last term, due to the vector fields being left invariant, we have:
		$$\theta([X,Y])={1\over 2}[\theta,\theta](X,Y)$$
		This completes the proof.
	\end{Ex}
	Lastly, one can further generalize the notion of $k$-form by letting them take values on a vector bundle.
	\begin{Def}
		Let $(E,M,\pi, \mathbb{R}^n)$ be a vector bundle. We define an $E$-valued $k$-form as a section of $\bigwedge^kTM^*\otimes E$. We denote the space of smooth $E$-valued $k$-forms with $\Gamma(\bigwedge^kTM^*\otimes E)$.
	\end{Def}
	In particular, a form $\omega$ wjth value in a vector bundle $E$ associates to each point $p\in M$ an alternating map $\omega_p:T_pM\times...\times T_pM\rightarrow E_p$.
	\begin{Obs}
		By choosing a local frame $\{e_i:M\rightarrow E\}$, we can expand any form $\omega=\omega^i\otimes e_i$ where $\omega^i$ are real valued forms on the base manifold $M$.
	\end{Obs}
	\section{Principal Bundles}
	In this section we define the notion of principal bundles, look at some properties that they have and make some examples. The interested reader can find more details in [2] chap. 6
	\begin{Def}
		Let $(E,M,\pi,G)$ be a fiber bundle, where $G$ is a Lie group and $M$ and $E$ are manifolds. Let $\pi$ be smooth and the trivializations diffeomorphic; let $G$ be acting on $E$ such that:
		\begin{itemize}
			\item $G$ acts freely on $E$: $Stab(E)=\{e\in G\}$;
			\item Let $U\in M$ be a local trivializing set and $\phi:\pi^{-1}(U)\rightarrow U\times G$  the corresponding trivialization, then $\phi$ is equivariant: $\phi(x,h)\cdot g=\phi(x,hg)$
		\end{itemize} 
		Then we say that this fiber bundle is a \textit{principal $G$-bundle}.
	\end{Def}
	\begin{Ex}
		Consider the product bundle as usual, but let $E$ and $M$ be manifolds and $G$ be a Lie group. Then $E=M\times G$. We can give this the fiber bundle structure in the usual way: we first of all define a projection:
		$$\pi:M\times G\rightarrow M\hbox{ like }\pi(x,g)=x$$
		This map is clearly continuous and surjective. Also, it can be shown to be smooth through the use of charts (it is just a matter of substituting coordinates, since both $M$ and $G$ have the structure of a manifold). The local trivializations will just be the identity map, which is clearly diffeomorphic. Now, we need to ask for a $G$ action of $G$ on $M\times G$. There is an obvious choice:
		$$(x,g)\cdot h=(x,gh)$$
		Furthermore the identity map clearly preserves this action in an equivariant way.
	\end{Ex}
	\begin{Prop} \label{Prop_3.5.1}
		Let $(E,M,\pi,G)$ be a principal $G-$bundle, then $G$ acts transitively on each fiber. Also, for any group $G$, any right-equivariant map is a left translation.
	\end{Prop}
	\begin{proof}
		$G$ acts transitively on $\{p\} \times G$, obviously. Looking at the fiber diffeomorphism $\phi:\pi^{-1}\rightarrow U\times G$, it can be reduced to $\phi_p:E_p\rightarrow \{p\}\times G$, which is equivariant. Thus, on each fiber, $G$ acts transitively.\\
		\\
		As for the second proposition, it is just an easy result of group theory: if $f(gh)=f(g)h$, setting $g=e$ we get: $f(h)=f(e)h=l_{f(e)}h$.
	\end{proof}
	\begin{Obs} \label{Obs_3.5.1}
		By definition, in a principal bundle, the trivialization functions are equivariant. This means that:
		$$\pi\circ r=\pi\hspace{20 pt}d\pi\circ dr=d\pi$$
	\end{Obs}
	\begin{Ex}
		Let $G$ be a Lie group and $H$ be a normal closed subgroup of it. One can show that the quotient $G/H$ can be given a manifold structure.
		There is a natural projection $\pi:G\rightarrow G/H$ taking each element to its class: $g\rightarrow \{gH\}$. This can be given a principal $H$ bundle structure. The proof can be found in [8] chap. 3 pag. 120.
	\end{Ex}
	\begin{Ex}
		In example \ref{Ex_3.1.3} we proved that the Hopf bundle is a fiber bundle. It is also possible to endow this with a principal bundle structure. Recall that $S^1$ can be identified with the Lie group $U(1)$. This group acts on $S^3$ in the obvious way:
		$$\lambda(z_1,z_2)=(\lambda z_1,\lambda z_2)$$
		This action is evidently free. Moreover, the trivializations are equivariant under this action: suppose to take as open set $U_1=S^2-(0+i0,1)$, then:
		$$\varphi(\lambda z_1,\lambda z_2)=\bigg(\pi(\lambda z_1,\lambda
		z_2),\lambda{z_2\over |z_2|}\bigg)=\bigg(\pi(z_1,	z_2),\lambda{z_2\over |z_2|}\bigg)=\bigg(\pi(z_1,	z_2),{z_2\over |z_2|}\bigg)\cdot \lambda$$
		This makes the Hopf bundkle into a principal $U(1)$ bundle.
	\end{Ex}
	\begin{Def}
		We will call a section of a principal bundle $s:U\subset M\rightarrow E$ a \textit{local gauge}. If the section is global we will call it \textit{global gauge}.
	\end{Def}
	We now define yet another important concept: the transition functions.
	\begin{Def}
		Let $(P,M,\pi,G)$ be a principal bundle. Let $\phi_i:\pi^{-1}(U_i)\rightarrow U_i\times G$ and $\phi_j:\pi^{-1}(U_j)\rightarrow U_j\times G$ two local trivializations such that $U_i\cap U_j\neq\emptyset$. Then $\phi_i\circ\phi^{-1}_j:(U_i\cap U_j)\times G\rightarrow (U_j\cap U_i)\times G$ acts like: $\phi_i\circ\phi^{-1}_j(x,h)=(x,g_{ij}(x)h)$. We call \textit{transition function} between $\phi_i$ and $\phi_j$ the map $g_{ij}:U_i\cap U_j\rightarrow G$. 
	\end{Def}
	\begin{Obs}\label{Obs_3.5.2}
		The transition functions are clearly smooth functions of $x$ since for $h=e$
		$$\phi_i\circ\phi^{-1}_j(x,e)=(x,g_{ij}(x)e)$$
		Moreover, they satisfy the \textit{cocyle condition:}
		$$g_{ij}g_{jl}=g_{il}$$
	\end{Obs}
	From the cocycle condition one can easily show:
	\begin{Prop}
		The transition functions satisfy the following properties:
		\begin{itemize}
			\item $g_{ii}=e$
			\item $g_{ij}=g_{ji}^{-1}$
		\end{itemize}
	\end{Prop}
	\begin{proof}
		The proof can be found in [2] pag. 245.
	\end{proof}
	A key result is that, given a local gauge, we can construct a local trivialization on the open set it is defined on:
	\begin{Prop}\label{Prop_3.5.3}
		Let $(E,M,\pi,G)$ be a principal bundle. Let $s:U\subset M\rightarrow E$ be a smooth local gauge. Then, the map:
		$$\varphi^{-1}: U\times G\rightarrow P$$
		$$(x,p)\rightarrow s(x)\cdot p$$
		is a local trivialization.
	\end{Prop}
	\begin{proof}
		We now give an idea of the proof.
		We have to show that this map is a $G$-equivariant diffeomorphism. By smoothness of $s$ and of the right action, this map is smooth. The $G$-equivariance is clear:
		$$\varphi(x,hg)=s(x)\cdot hg=(s(x)\cdot h)\cdot g=\varphi(x,g)\cdot h$$
		Furthermore, one can prove the map is bijective and that the inverse is smooth. The full proof is in [3] chap. 4 pag. 210, 211.
	\end{proof}
	\begin{Obs}\label{Obs_3.5.3}
		Note that, given two sections $s_{1,2}:U\rightarrow P$ on the same open set $U$, we have two local trivializations:
		$$\phi^{-1}_{1,2}(x,g)=s_{1,2}(x)\cdot g$$
		then it is obvious that, by definition of transition function: 
		$$\phi_1\circ\phi_2^{-1}(x,g)=(x,g_{12}(x)\cdot g)=\phi_1(s_{2}(x)\cdot g)$$
		By applying on both sides $\phi_1^{-1}$ we find:
		$$s_2(x)\cdot g=s_1(x)\cdot g_{12}(x)\cdot g$$
		Clearly, we have the following relation between the two sections:
		$$s_2(x)=s_1(x)\cdot g_{12}(x)$$
	\end{Obs}
	We are now going to construct one of the most important principal bundles: the frame bundle. This is a principal bundle constructed from a pre-existing vector bundle.
	\section{The frame manifold}
	In this section we will introduce the notion of frame manifold of a vector space. We will show that the frame manifold is indeed a smooth manifold. For more information the reader is advised to look at: [2] chap. 6 pag. 241-252.
	\begin{Def}\label{Def_6.5}
		Consider a vector space $V$ of finite dimension. We define the frame manifold $Fr(V)$ as the set made of all ordered bases of $V$. This means that $Fr(V)$ contains all of the possible ways of choosing a basis for $V$.
	\end{Def}
	\begin{Obs}
		There is a natural action of the general linear group on $Fr(V)$. The idea is the following: since $Fr(V)$ contains all of the possible basis of $V$, we get that, having chosen a basis, we can re-order the elements of it and obtain the same basis. We can also linearly combine those elements to obtain another basis. Thus, all of the elements of $Fr(V)$ will be reached by matrices in $GL(n,\mathbb{R})$, where $n$ is the dimension of the vector space.\\
		\\
		In particular, identifying an element of $Fr(V)$ with a row vector $\vec{e}=(e_1,...,e_n)$, we have a right-action of the Lie group $GL(n,\mathbb{R})$ of the form:
		$$\vec{e}A=(e_1,...,e_n)\begin{pmatrix}
			a_{11}&a_{12}&...&a_{1n}\\
			a_{21}&a_{12}&...&a_{2n}\\
			.     &.     &.   &     \\
			.     &.     &.   &     \\
			a_{n1}&.     &...&a_{nn}\\
		\end{pmatrix}$$
		We also note that the stabilizer of any $\vec{e}\in Fr(V)$ is just $\mathbb{I}_n\in GL(n,\mathbb{R})$, so that the group acts freely on $Fr(V)$. Moreover, the orbit of any element is the whole space: 
		$$O(\vec{e}\in Fr(V))=Fr(V)$$
		This last result is obvious: given any two basis one can always find one as a linear combination of the other and so the two will be related by a matrix of $GL(n\mathbb{R})$.
		Now, by the orbit-stabilizer theorem, having fixed an element $\vec{e}\in Fr(V)$, we have a bijection:
		$$\phi_{\vec{e}}:{GL(n,\mathbb{R})\over Stab(\vec{e})}=GL(n,\mathbb{R})\rightarrow O(\vec{e})=Fr(V);\hbox{ such that } \phi_{\vec{e}}([A])=\vec{e}A$$
		The idea now is to put a manifold structure on $Fr(V)$ such that the above map becomes a diffeomorphism. Note also that the stabilizer of any element is just the identity group. Thus, not only the image, but also the domain of $\phi_{\vec{e}}$ will be independent of the choice of $\vec{e}$.
	\end{Obs}
	We now aim at giving $Fr(V)$ a manifold structure. To do this, we first need to show it is a topological manifold and then show that it possesses a smooth atlas.
	\begin{Theo}
		Given any vector space $V$, $Fr(V)$ is a topological manifold.
	\end{Theo}
	\begin{proof}
		Suppose $dim V=n$. Having fixed any element $\vec{e}\in Fr(V)$, we know that there is a bijection 
		$$\phi_{\vec{e}}:GL(n,\mathbb{R})\rightarrow Fr(V); \hbox{ acting like: }\vec{e}\mapsto\vec{e}A$$
		We use this bijection to induce a topology on $Fr(V)$ from the one of $GL(n,\mathbb{R})$. We say that a set $U\in Fr(V)$ is open if and only if $\phi^{-1}(U)$ is open in $GL(n,\mathbb{R})$. This automatically makes the map $\phi_{\vec{e}}$ homeomorphic and endows $Fr(V)$ with the same topological qualities as $GL(n,\mathbb{R})$. Moreover, since $GL(n,\mathbb{R})$ is a topological manifold, it is T2 and second countable and thus by this mapping also $Fr(V)$ is.
		It remains to make $Fr(V)$ locally euclidean. Consider any chart $(V,\Phi)$ on $GL(n,\mathbb{R})$. This induces a chart on $Fr(V)$ like follows:
		$$(\phi_{\vec{e}}(V),\Phi\circ\phi^{-1}_{\vec{e}})$$
		This completes the proof.
		Note also that if we were to choose any other element of $Fr(V)$ like $\vec{f}=\vec{e}A$, the map would still be still an homeomorphism since:
		$$\phi_{\vec{f}}(B)=\vec{f}B=\vec{e}AB=\phi_{\vec{e}}\circ l_A(B)$$
		And the left translation is a diffeomorphism.
	\end{proof}
	We now briefly study the smooth structure of the frame bundle. 
	\begin{Theo}
		Given any vector space $V$, $Fr(V)$ is a smooth manifold.
	\end{Theo}
	\begin{proof}
		We have already endowed $Fr(V)$ with the structure of a topological manifold. It only remains to show that two arbitrary charts are compatible. Consider two overlapping charts: $(U,\Phi_1\circ\phi^{-1}_{\vec{e}})$ and $(U,\Phi_2\circ\phi^{-1}_{\vec{e}})$ induced from $GL(n,\mathbb{R})$. Then we can write explicitly:
		$$\Phi_1\circ\phi^{-1}_{\vec{e}}\circ (\Phi_2\circ\phi^{-1}_{\vec{e}})^{-1}=
		\Phi_1\circ \Phi_2^{-1}$$ 
		This is clearly a diffeomorphism since the maps $\Phi_i$ are.
	\end{proof}
	\section{The frame bundle}
	In this section we will define the frame bundle of a smooth vector bundle. We will show that it is possible to use it to construct another bundle on the base manifold. More details are in [2] chap. 6 pag. 246 and [3] chap. 4 pag 223.\\
	\\
	Suppose to have a smooth vector bundle $(E,M,\pi,F)$. Our idea is now to associate to $M$ a smooth principal $GL(n,\mathbb{R})-bundle$. At any point $p\in M$, the fiber $F_p$ is a vector space. Thus, we can construct $Fr(F_p)$ the frame manifold of each fiber. Define the Frame Bundle as the following space:
	$$Fr(E)=\bigsqcup_{p\in M}Fr(F_p)$$
	The disjoint union of the fibers at each point. There is a natural projection that we can use:
	$$\pi_G:Fr(E)\rightarrow M\hbox{ such that } \pi_G(\vec{e_p}\in Fr(F_p))=p$$
	Now we need to find local trivializations. The idea is quite simple but somewhat elaborate.\\
	\\
	Suppose now to have a chart on $Fr(E)$ like $(U_{M},\psi:U_{Fr(E)}\rightarrow \mathbb{R}^n\times\mathbb{R}^k)$. Due to the existence of the projection $\pi_G$ we can induce a chart on $M$ by finding the corresponding open set on it $U_M$ and it's map $\phi:U_M\rightarrow \mathbb{R}^n$. We choose as trivialization the same map we had chosen to trivialize the tangent bundle:
	$$\varphi\equiv(\phi^{-1}\circ \mathbb{I}_{\mathbb{R}^k})\circ \psi: U_{Fr(E)}\xrightarrow{\psi}\phi(U_M)\times\mathbb{R}^k\xrightarrow{\phi^{-1}\times\mathbb{I}_{\mathbb{R}^k}}U_m\times\mathbb{R}^k$$
	This is clearly a diffeomorphism.\\
	\\
	Note that at every point there is an identification of $Fr(E_p)$ with the general linear group $GL(k,\mathbb{R})$. To make into a principal bundle we need to verify that $GL(k,\mathbb{R})$ acts freely on $Fr(E)$ and that the trivializations are equivariant under the action of the group.\\
	\\
	The action of $GL(k,\mathbb{R})$ on the fiber is just given by the product of matrices and so it is free. In particular, $GL(k,\mathbb{R})$ acts on $Fr(E)$ like this:
	$$(p\in M,\vec{e_p}\in Fr(E_p))\longrightarrow(p,\vec{e_p}A)$$
	This action is evidently free. Moreover, considering the chart $\varphi$, it is equivariant under this action. Thus, we have a principal $GL(k,\mathbb{R})$-bundle $(Fr(E),M,\pi_G,GL(k,\mathbb{R}))$.
	\begin{Obs}
		The frame bundle is nothing more than the union of all of the frame manifolds of the fibers. Once we have a smooth fiber bundle on $M$, constructing the frame bundle means associating another bundle to $M$. This new bundle will be a principal bundle and at any point the fiber will be $Fr(F)$ the frame manifold of the fiber, which is 1 to 1 with $GL(k,\mathbb{R})$.
	\end{Obs}
	\chapter{Connections on vector bundles}
	In this chapter we will introduce connections on vector bundles and the notion of parallel transport. For more details see [2] chap. 3 pag. 95-102, chap. 6 pag. 262-269.
	\section{Connections}
	In this section we will give the definition of a connection on a generic vector bundle and look at some examples and interesting properties. For more details the reader is advised to consult [2] chap. 6 pag. 262-269.
	\begin{Def}\label{Def_5.7}
		Let $(E,M,\pi,F)$ be a vector bundle. We call \textit{connection} a map $\nabla:\mathfrak{X}(M)\times\Gamma(E)\rightarrow\Gamma(E)$ such that:
		\begin{itemize}
			\item $\nabla$ is $\mathfrak{F}$-linear in $\mathfrak{X}(M)$ and $\mathbb{R}$ linear in $\Gamma(E)$: $\nabla_{fX}(\lambda s)=\lambda f\nabla_X s$ for $f:M\rightarrow \mathbb{R}, s\in\Gamma(E),\lambda\in\mathbb{R}$;
			\item $\nabla$ respects Leibniz: $\nabla_X (fs)=(Xf)s+f\nabla_X s$.
		\end{itemize}
		If $E=TM$ we say that the connection is affine.
	\end{Def}
	Note that since $X$ is a smooth vector field on $M$, we have that $Xf=df(X)$ is the differential of $f$ applied to $X$.
	\begin{Def}\label{Def_5.8}
		We say that a section $s$ is flat if $\nabla_X s=0$ for every $X\in \mathfrak{X}(M)$.
	\end{Def}
	\begin{Ex}
		Consider the smooth manifold $\mathbb{R}^n$ and it's tangent bundle. Then, a connection will be a map like:
		$$\nabla:\mathfrak{X}(M)\times\mathfrak{X}(M)\rightarrow\mathfrak{X}(M)$$
		We define $\nabla_XY$ to be the directional derivative: if $X_p$ is a vector at $p\in\mathbb{R}^n$ and $Y=Y^i\partial_i$ is a vector field, then:
		$$\nabla_XY=(X_pY^i)\partial_i$$
		This is an affine connection.
	\end{Ex}
	\begin{Obs}
		Given any connection $\nabla:\mathfrak{X}(M)\times\Gamma(E)\rightarrow\Gamma(E)$ on a vector bundle $(E,M,\pi,F)$ and a frame $\{e_i\}$ on a set $U\subset M$, we can expand any generic section into a linear combination:
		$$s=h^ie_i$$
		where the maps $h^i:U\rightarrow \mathbb{R}$ depend on the points. If we apply the connection to this section we find, by the defining properties:
		$$\nabla_Xs=(Xh^i)e_i+h^i\nabla_Xe_i$$
		The second term must be a section as well, so that $\nabla_Xe_i=\omega^j_i(X)e_j$. We call $\omega^j_i$ connection forms and the \textit{corresponding matrix} $[\omega^j_i]$ \textit{connection matrix}. 
	\end{Obs}
	\begin{Prop}
		If $(E,M,\pi,F)$ is a smooth vector bundle, it is always possible to find a connection on it.
	\end{Prop}
	\begin{proof}
		The proof of this proposition can be found in [2] chap.2 pag. 73.
	\end{proof}
	\section{The pullback bundle}
	In this section we will briefly define the pullback bundle and look at some of its properties. This construction will be useful to construct the notion of parallel transport along a given curve. To see more details the reader is advised to see [2] chap. 4, pag. 177-180.
	\begin{Def}
		Let $(E,M,\pi,F)$ be a vector bundle and $f:N\rightarrow M$ be a smooth map between manifolds. We define the \textit{pullback bundle} as the set:
		$$f^*E=\{(x_N,x_E)\in N\times E|f(x_N)=\pi(E)\}$$
	\end{Def}
	We can define a projection like: $\pi_N:f^*E\rightarrow N$ in the obvious way: $\pi_f(x_N,x_E)=x_N$. 
	\begin{Prop}
		Given any smooth vector bundle $(E,M,\pi,F)$ and $f:N\rightarrow M$ smooth map between manifolds, then $(f^*E,N,\pi_f,F)$ is a smooth vector bundle.
	\end{Prop}
	\begin{proof}
		The full proof of this proposition can be found at [2] chap.4 pag. 178.
	\end{proof}
	In general, we will denote with $\Gamma(f^*E)$ the space of all smooth sections of this new vector bundle.
	\begin{Ex}
		Consider a manifold $M$ and it's tangent bundle $(TM,M,\pi,\mathbb{R}^n)$. Consider then a smooth curve $\gamma:]a,b[\subset \mathbb{R}\rightarrow M$. This curve is a smooth map between manifolds and induces a new vector bundle, namely: $(\gamma^*TM,]a,b[,\pi_\gamma,\mathbb{R}^n)$. If now we look at the sections of this new bundle, they are maps like $s_\gamma:]a,b[\rightarrow\mathbb{R}^n$ which associate to each point of the curve a vector. Thus, smooth sections of this pullback bundle are just smooth vector fields along a curve.
	\end{Ex}
	\section{The parallel transport on vector bundles}
	In this section we are going to define the notion of parallel transportation along a smooth curve on a manifold. We will see that this definition relies strongly on the notion of connection. For more information the reader is advised to see [2] chap. 6 pag. 262-268.
	\begin{Theo}
		Let $(E,M,\pi,F)$ be a smooth vector bundle with a connection $\nabla$ on it. Let $\gamma:]a,b[\rightarrow M$ be a smooth curve on $M$ and consider the pullback bundle $\gamma^*E$ induced by it. Then, there is a unique linear map ${D\over dt}:\Gamma(\gamma^*E)\rightarrow \Gamma(\gamma^*E)$ called \textit{covariant derivative along $\gamma$}, such that:
		\begin{itemize}
			\item ${D\over dt}$ is linear;
			\item ${D\over dt}$ satisfies the Leibniz rule;
			\item if $s\in \Gamma(\gamma^*E)$ is induced from a global section $S\in\Gamma(E)$ restricted to $\gamma$, then ${D\over dt}s(t)=\nabla_{\gamma'(t)}S$.
		\end{itemize}
	\end{Theo}
	\begin{proof}
		Suppose that the covariant derivative exists. Then choose an open set with a frame $(U,{e_i})$, where $e_i:M\rightarrow
		E$ are sections. Then, any section along this curve can be expressed as a linear combination of the elements of the basis:
		$$s=s^ie_{i,\gamma}$$
		where we have indicated with $e_i,{\gamma}$ the restriction of the frame to the curve. This relation holds for every point $p\in U\cap\gamma$. Now we apply the defining properties of the covariant derivative:
		\begin{equation}\label{Cov_Der}
			{D\over dt}s={ds^i\over dt}e_i+s^i\nabla_{\gamma'(t)}e_i	
		\end{equation}
		This proves that if $D\over dt$ exists is unique. As for the existence, we can define the covariant derivative along a curve $\gamma(t)$ like we did in formula \ref{Cov_Der}. This definition is by uniqueness independent from the choice of the frame choice on $U$. This completes the proof.
	\end{proof}
	\begin{Def}
		We say that a section $s\in\Gamma(\gamma^*E)$ is parallel along $\gamma:]a,b[\rightarrow M$ if ${D\over dt}s=0$ for every $t$. We call $s(b)$ the parallel transport of $s(a)$ along $\gamma$.
	\end{Def}
	\begin{Prop}
		Let $(E,M,\pi,F)$ be a smooth vector bundle and $\gamma:[a,b]\rightarrow M$ a smooth curve. Then there is a unique isomorphism called \textit{parallel transport} of the form $\phi_{a,b}:F_{\gamma(a)}\rightarrow F_{\gamma(b)}$.
	\end{Prop}
	\begin{proof}
		Let $\{e_i(a)\}$ be a basis for $E_{\gamma(a)}$ and $s=s^ie_{i,\gamma}$ be a section along $\gamma$. Consider the following set of equations:
		$${D\over dt}{s^je_{j,\gamma}}=\dot{s}^je_j+s^j\nabla_{\gamma'(t)}e_j=[\dot{s}^j+s^j\omega^i_j(\gamma'(t))]e_{i,\gamma}=0$$
		This is a system of linear differential equations, which, once fixed an initial condition $s(a)=s^i(a)e_i(a)$ is given, has a unique solution in a neighborhood of $a$.  This means that, if $]a,b[$ is sufficiently small, there is a unique parallelly transported section $s(b)$ along $\gamma$. Using this result and fixing $s^j(a)=1$, we can say that there is a unique vector field for each index $j$, which is parallelly transported along $\gamma$ and such that it is a basis for $E$ at $\gamma(a)$. This induces a linear mapping:
		$$\phi_{a,b}:E_{\gamma(a)}\rightarrow E_{\gamma(b)}\hbox{ acting like: } s(a)\rightarrow s(b)$$
		From the fact that any section $s(t)$ is parallel along $\gamma(t)$ if and only if $s(-t)$ is parallel along $\gamma(-t)$, we can construct the inverse of this mapping:
		$$\phi^{-1}_{a,b}=\phi_{b,a}$$
		This proves the map is isomorphic.
	\end{proof}
	\begin{Obs}
		Recall that frames are elements of the frame bundle. From the previous result we have substantially shown that, given a smooth curve $\gamma$, a connection $\nabla$ on a vector bundle $(E,M,\pi,F)$ and an initial condition $\{e_i(a)\}\in Fr(E_\gamma(a))$, we can always find find a unique element of the frame bundle which is parallely transported along $\gamma$.
	\end{Obs}
	\begin{comment}
		We now look at the notion of parallel transport on the tangent bundle of a manifold.
		\section{The parallel transport on the tangent bundle}
		In this section we will use the previous results to study the notion of parallel transport on the tangent bundle of a smooth manifold. More information on this topic can be found in [2] chap.3 pag. 95-102.
	\end{comment}
	\chapter{Connections on principal bundles}
	In this chapter we will introduce the notion of connection on principal bundles, together with some properties of them. Those will be the main object of Gauge field theories. More informations can be found in [2] chap. 6 pag. 254-286; [3] chap. 4, 5.
	\section{Fundamental vector fields}
	In this section we will define the concept of fundamental vector field. We will look at some properties of it and at its integral curves. More details on this topic can be found in [2] chap. 6.\\
	\\
	Suppose to have a Lie group $G$ that right-acts smoothly on a manifold $M$. Then, consider any element $A\in\mathfrak{g}$ inside the Lie Algebra of $G$. Define:
	$$\bar{A}={d\over dt}\bigg|_{t=0}p\cdot e^{tA}={d\over dt}\bigg|_{t=0}\mu(p, e^{tA})\in T_pM$$
	where $\mu$ is the action of $G$ on the manifold.
	This, at any point is called \textit{fundamental vector field associated to $A$}.\\
	\\
	The idea is to take a point $p\in M$ and a smooth curve passing through it like $c_p(t)=p\cdot e^{tA}$. The fundamental vector field in $p$ is the initial tangent vector of this curve.
	\begin{Prop}\label{Prop_5.1.2}
		The fundamental vector field associated to $A$ is always smooth.
	\end{Prop}
	\begin{proof}
		This proof can be found in [2] chap. 6 pag. 247, 248.
	\end{proof}
	There is an alternative mathematical formulation for the fundamental vector field: define the map $j_p:G\rightarrow M$ like follows:
	$$j_p(g)=p\cdot g$$
	Its differential is defined as:
	$$dj_p(A\in\mathfrak{g})={d\over dt}\bigg|_{t=0}j_p(e^{tA})={d\over dt}\bigg|_{t=0}p\cdot e^{tA}$$
	\begin{Prop}\label{Prop_5.1.1}
		Let $G$ be a Lie group right-acting smoothly on a manifold $M$ and let $r_g(p)=p\cdot g$ the right translation. For $A\in\mathfrak{g}$, it's fundamental vector field $\bar{A}$ satisfies:
		$$dr_g(\bar{A})=\overline{Ad(g^{-1})(A)}$$
	\end{Prop}
	\begin{proof}
		Recall that the adjoint representation is given by the differential of the conjugation map:
		$$Ad(g)=dc_g:\mathfrak{g}\rightarrow\mathfrak{g};\hbox{ in particular }Ad(g)(A)=gAg^{-1}$$
		Now, consider $(r_g\circ j_p)(h)=p\cdot hg=p\cdot gg^{-1}hg=p\cdot g c_{g^{-1}}(h)=(j_{pg}\circ c_{g^{-1}})(h)$
		By the chain rule:
		$$dr_g(\bar{A}_p)=dr_gdj_p(A)=dj_{pg}dc_{g^{-1}}(A)=dj_{pg}(Ad(g^{-1})(A))=\overline{Ad(g^{-1})(A_{pg})}$$
	\end{proof}
	We now study the integral curves of the fundamental vector fields.
	\begin{Prop}
		The curve $c_p(t)=p\cdot e^{tA}$ is the integral curve of the fundamental vector field $\bar{A}$ passing through $p\in M$.
	\end{Prop}
	\begin{proof}
		It suffices to differentiate:
		$$c'_p(t)={d\over dt}\bigg|_{s=0}c_p(t+s)=
		{d\over dt}\bigg|_{s=0}p\cdot e^{tA}e^{sA}=\bar{A}_{pe^{tA}}=
		\bar{A}_{c_p(t)}$$
	\end{proof}
	\begin{Prop}
		The fundamental vector field $\bar{A}$ vanishes at $p$ if and only if $A$ is in the Lie algebra of $Stab(p)$.
	\end{Prop}
	\begin{proof}
		If $\bar{A}_p=0$ it means that the constant map $c_p(t)=p\in M$ is an integral curve of this vector field. By the previous proposition, we have the equivalence:
		$p=p\cdot e^{tA}$ from which we see that $A$ must be in the Lie algebra of $Stab(p)$ since $e^{tA}\in Stab(p)$.\\
		\\
		On the contrary, if $A\in Stab(p)$ then we ge that:
		$$\bar{A}_p={d\over dt}\bigg|_{t=0}p\cdot e^{tA}=
		{d\over dt}\bigg|_{t=0}p=0$$
	\end{proof}
	\begin{Obs}\label{Kerdj}
		As a corollary of the previous statement we can clearly say that $Ker(dj_p)$ at the identity is the stabilizer $Stab(p)$.
		$$Ker (dj_p)=\{A\in \mathfrak{g}|dj_p(A)=0\}=Stab(p)$$
	\end{Obs}
	\section{The vertical subbundle}
	In this section we will introduce the notion of vertical sub-bundle of the tangent bundle and see that the fundamental vector fields are precisely the vertical vector fields. More details on this topic can be found in [2] chap. 6 pag 250-252.
	\begin{Def}
		A \textit{distribution} on a manifold $M$ is a subbundle of the tangent bundle $TM$.
	\end{Def}
	Let $(E,M,\pi,G)$ be a principal $G$-bundle on a manifold $M$ and $x\in E$. By construction, the projection $\pi:E\rightarrow M$ and the differential $d\pi_{x}:T_{x}E\rightarrow T_{\pi(x)}M$ is surjective since the projection is. We now define the \textit{vertical tangent subspace} $\mathcal{V}_{x}\subset T_{x}E$ as the Kernel of the differential of the projection:
	$$\mathcal{V}_{x}\equiv Ker(d\pi_{x})$$
	\begin{Obs}
		By dimensional considerations, 
		$$dim (Ker(d\pi_{x}))+dim(Im(d\pi_{x}))=dim(T_{x}E)$$
		Moreover, the map $d\pi$ is surjective and so the image corresponds to the full space $T_{\pi(x)}M$, we get:
		$$dim (Ker(d\pi_{x}))=dim\mathcal({V}_{x})=dim(T_{x}E)-dim(T_{\pi_{x}}M)=dim (G)$$
	\end{Obs}
	\begin{Prop}\label{fund_vec_are_vert}
		For every element of $E$, the corresponding fundamental tangent vector is vertical.
	\end{Prop}
	\begin{proof}
		We just need to prove that the fundamental tangent vector is in the kernel of the differential. Recall that the fundamental vector field can be seen as the differential of the map $j_{x}:G\rightarrow E$ defined as 
		$$j_{x}(h)=x\cdot h $$
		Taking now the composition with the projection:
		$$(\pi\circ j_{x})(h)=\pi(x\cdot h)=\pi(x)$$
		This result is independent of the choice of $h\in G$, thus the composition $\pi\circ j_{x}$ is the constant map and so its differential is the null map.\\
		Recalling that $\bar{A}_{x}=dj_{x}(A)$ we clearly see that:
		$$d\pi_{x}(\bar{A}_{x})=
		(d\pi_{x}\circ dj_{x})(A)=
		d(\pi\circ j_{x})(A)=0$$
	\end{proof}
	From this proposition it follows a key result: the fundamental vector fields are precisely the vertical vector fields.
	\begin{Prop}
		Let $(E,M,\pi,G)$ be a principal $G-$bundle and $x\in E$. Then: 
		$$dj_{x,e}:\mathfrak{g}\rightarrow \mathcal{V}_{x}$$ is an isomorphism.
	\end{Prop}
	\begin{proof}
		By observation \ref{Kerdj} we know that the kernel of the differential at the identity is the stabilizer of $x$. However, for a principal $G-$bundle, the group acts freely on the manifold $E$ and so the stabilizer is trivial $Stab(x)=\{e\}$. Thus, $Ker (dj_{x,e})=0$ and the map is injective. By the previous proposition \ref{fund_vec_are_vert} the image of the differential lies in the vertical subspace. Moreover, $dim(\mathcal{V}_{x})=dim(G)$ and so the map must be an isomorphism of vector spaces.			
	\end{proof}
	\begin{Obs}
		As a consequence of the last result, all the vertical vectors are the fundamental vectors and all the fundamental vectors are the vertical vectors.\\
		Let $T^a$ be a basis for the Lie algebra $\mathfrak{g}$, then, at any point, by the isomorphism in the previous result, we get that they also form a basis of $\mathcal{V}_{x}$ (under the action of the diffeomorphism). This defines the \textit{Vertical Subbundle} $$\mathcal{V}=\bigsqcup_{x\in E}\mathcal{V}_{x}$$
	\end{Obs}
	\begin{Obs}
		The name "vertical" comes from the definition $\mathcal{V}=Ker(d\pi)$. If we take any smooth vector field $X_E\in\mathfrak{X}(E)$, then the differential of the projection $d\pi:TE\rightarrow TM$ induces a new smooth vector field $X_M=d\pi (X_E)$ on the base space. The "vertical vector fields" are the ones who get annihilated when projected, since they belong to the Ker of $d\pi$.
	\end{Obs}
	\section{Connections and horizontal subbundle}
	In this section we introduce the notions of horizontal sub-bundle and Ehresmann connection. More information on this subject can be found in [2] chap. 6 pag. 251, 252.\\
	\\
	We have seen that on a principal $G$-bundle there is a canonical well defined vertical subbundle $\mathcal{V}$ of the tangent bundle $TE$. We say that there is an \textit{horizontal distribution} $\mathcal{H}$ if $TE=\mathcal{H}\oplus\mathcal{V}$. This means that for every point $x\in E$ we have $T_{x}E=\mathcal{H}_{x}\oplus\mathcal{V}_{x}$ with $\mathcal{H}_{x}\cap\mathcal{V}_{x}=0$.
	\\
	\\
	In general, there is no canonically defined horizontal distribution on a principal $G$-bundle. This comes from the fact that the complement of $\mathcal{V}_x$ is not unique. We will see that choosing an horizontal distribution means choosing a connection on the principal bundle. 
	\begin{Def}
		We say that an horizontal distribution $\mathcal{H}$ is \textit{right invariant} if $dr(\mathcal{H})\in \mathcal{H}$.
		Instead we say that $\mathcal{H}$ is a \textit{smooth distribution} if it is a smooth sub-bundle of the tangent bundle.
	\end{Def}
	Let $(E,M,\pi,G)$ be a principal $G$-bundle with a vertical and horizontal distribution, so that $TE=\mathcal{H}\oplus\mathcal{V}$. Let $\nu_{x}:\mathcal{H}_{x}\oplus\mathcal{V}_{x}\rightarrow \mathcal{V}_{x}$ be the standard projection on the vertical subbundle. Obviously, if $X_{x}\in T_{x}E$ is a vector field on $E$ at a point $x$, we call $\nu_{x}(X_{x})$ its \textit{vertical component}.
	\begin{Def}\label{Def_6.6}
		We call an \textit{Ehresmann connection}, or \textit{connection on a principal bundle} a $\mathfrak{g}$-valued 1-form $\omega:TE\rightarrow \mathfrak{g}$ that respects the following properties:
		\begin{itemize}
			\item Given any $A\in\mathfrak{g}$ and $x\in E$ we have $\omega_{x}(\bar{A}_{x})=A$;
			\item $G$-equivariance for $g\in G$ we have $r^*_g\omega=Ad(g^{-1})\omega$;
			\item $\omega$ is $C^\infty$.
		\end{itemize}
	\end{Def}
	If we have an horizontal distribution, we can find a map that satisfies those conditions. Consider the following composition:
	$$\omega_{x}=dj_{x}^{-1}\circ \nu:T_{x}E\xlongrightarrow{\nu} \mathcal{V}_{x}\xlongrightarrow{dj_{x}}\mathfrak{g}$$
	This map takes a vector, selects its horizontal component and maps it to the corresponding Lie algebra element.
	\begin{Theo}
		If $\mathcal{H}$ is a smooth right-invariant horizontal distribution on a principal $G$-bundle $(E,M,\pi,G)$, then the 1-form $\omega_{x}=dj_{x}^{-1}\circ \nu$ respects the following properties:
		\begin{itemize}
			\item Given any $A\in\mathfrak{g}$ and $x\in E$ we have $\omega_{x}(\bar{A}_{x})=A$;
			\item $G$-equivariance for $g\in G$ we have $r^*_g\omega=Ad(g^{-1})\omega$;
			\item $\omega$ is $C^\infty$.
		\end{itemize}
	\end{Theo}
	\begin{proof}
		Consider any $A\in\mathfrak{g}$ and any $x\in E$. Then the fundamental vector field in this point is in the vertical sub-bundle and so the projection $\nu$ leaves it invariant:
		$$\omega_{x}(\bar{A}_{x})=(dj_{x}^{-1}\circ \nu)(\bar{A}_{x})=dj_{x}^{-1}(\bar{A}_{x})=
		dj_{x}^{-1}(dj_{x}(A))=A$$
		This proves the first property.\\
		\\
		As for the second property, we recall that in proposition \ref{Prop_5.1.1} we proved that for a fundamental vector field  we have:
		$$dr_g(\bar{A})=\overline{Ad(g^{-1})(A)}$$
		If $\overline{X_x}$ is a vertical vector, we need to prove that:
		$$r^*_h\circ \omega_{x}(\overline{X_{x}})=Ad(h^{-1})\omega_{x}(\overline{X_{x}})$$. 
		The pullback of a 1 form gives us:
		$$r^*_h\circ \omega_{x}(\overline{X_{x\cdot h}})=\omega_{x\cdot h}(dr_h(\overline{X_{x}}))$$
		If $X_{x}\in\mathcal{V}$ is vertical then it will be a fundamental vector field of an element $X\in\mathfrak{g}$ and so by proposition \ref{Prop_5.1.1} we get:
		$$\omega_{x\cdot h}(dr_h(\overline{X_{x}}))=\omega_{x\cdot h}\overline{(Ad(h^{-1})(X_{x\cdot h}))}$$
		Now applying the previous property 2 times:
		$$\omega_{x\cdot h}(\overline{Ad(h^{-1})(X_{x\cdot h})})=Ad(h^{-1})(X)=Ad(h^{-1})\omega_{x}(\overline{X_{x}})$$
		If instead $X_{x}$ is horizontal then by the invariance horizontal distribution the field $dr(X)$ will still be horizontal ad so it will be annihilated by $\omega$ since it contains the projection on the vertical subbundle. Thus proving:
		$$r^*_h\circ \omega_{x}(X_{x})=Ad(h^{-1})\omega_{x}X_{x}=0$$
		As for the last property, we need to prove that $\omega$ is smooth in a neighbourhood of an arbitrary point $x\in E$. Since $E$ is a manifold, we choose a chart $(U_E,\phi_E)$ centered in a point $x\in E$. Let $\{T^i\}$ be a basis of the Lie algebra $\mathfrak{g}$ and $\{\bar{T^i}\}$ the associated fundamental vectors. Those fundamental vectors are all smooth by proposition \ref{Prop_5.1.2}. By the previous isomorphism, those span the whole vertical sub-bundle at any point. Furthermore, since by hypothesis $\mathcal{H}$ is a smooth distribution  of horizontal vectors, there is a basis of smooth vector fields $X^i$ in any point that spans the whole space. Thus, since $TE=\mathcal{H}\oplus\mathcal{V}$, we can expand any element of $TE$ as a linear combination:
		$$V=a_i \bar{T^i}+b_iX^i;\hbox{ with } V\in TE, a_i,b_i\in \mathcal{F}$$
		The coefficeints $a_i,b_i$ are smooth by the previous observations. Now, applying $\omega$, we get that the horizontal components are annihilated:
		$$\omega(a_i \bar{T^i}+b_iX^i)=\omega(a_i \bar{T^i})=a_i T^i$$
	\end{proof}
	Clearly, an Ehresmann connection arises naturally on a principal bundle once one selects an horizontal smooth right-invariant distribution on it.\\
	\\
	The correspondence goes both ways: if one can find a smooth $G$-equivariant 1-form that takes value on the lie algebra $\mathfrak{g}$ of $G$ such that $\omega(\bar{A})=A$, then one can construce the horizontal distribution. In particular, one defines: $$\mathcal{H}_{x}=Ker(\omega_{x})$$
	\begin{Prop}\label{Prop_5.3.1}
		If $\mathcal{H}$ is a smooth right-invariant distribution on a principal $G$-bundle $E$, then $\forall g\in G$, the differential of the right-action $dr_g:TE\rightarrow TE$ commutes with the horizontal and the vertical projection operators.
	\end{Prop}
	\begin{proof}
		Let $Hor:TE\rightarrow\mathcal{H}$ be the horizontal projection and $x\in E$. Since both the horizontal and the vertical distributions are right-invariant, it means that for any $X_{x}\in T_xE$ we have $dr_h( X_{x})=dr_h \nu(X_{x})+dr_hHor(X_{x})$ and each of the two terms is still vertical/horizontal, so that:
		$$dr_h (X_{x})=dr_h\circ \nu(X_{x})+dr_h\circ Hor(X_{x})=
		\nu\circ dr_h(X_{x})+Hor\circ dr_h(X_{x})$$
	\end{proof}
	The next theorem shows that given any 1-form $\omega$ with certain oproperties, we can always construct an horizontal distribution. The proof is pretty lenghty.
	\begin{Theo}
		Consider a principal $G$-bundle $(E,M,\pi,G)$ where we can find a smooth $\mathfrak{g}$-valued and $G$-equivariant 1-form on it, such that $\omega(\bar{A})=A$ for every fundamental vector field. Then $Ker(\omega)=\mathcal{H}$ is a smooth right-invariant horizontal distribution on $TE$.
	\end{Theo}
	\begin{proof}
		This proof can be found in [2] chap. 6 pag. 256-259.
	\end{proof}
	\section{The horizontal Lift of a vector field}
	In this section we will introduce the notion of horizontal lift of a vector field and look at some properties of it. More informations on this topic can be found in [2] chap. 6 pag 259, 260; [3] chap. 5 pag 286-289.\\
	\\
	Suppose that $\mathcal{H}$ is an horizontal distribution on a principal bundle $(E,M,\pi,G)$. Let $X\in\mathfrak{X}(M)$ be a vector field on $M$. Then for every point $x\in E$, we have $\mathcal{V}_{x}=Ker(d\pi_{x})$ and so we have some induced isomorphisms:
	$${T_{x}E\over \mathcal{V}_{x}}\simeq \mathcal{H}_{x}\simeq T_{\pi(x)}M$$
	\begin{Def}
		Let $(E,M,\pi,G)$ be a principal $G$-bundle with a smooth right-invariant horizontal distribution $\mathcal(H)$. Let $X$ be a vector field on $M$. We call \textit{lift} of $X$ to $E$ any vector field $\tilde{X}$ such that $d\pi(\tilde{X})=X$. We call the unique horizontal vector field $\mathcal{X}\in \mathcal{H}$ such that $d\pi(\mathcal{X})=X$ the \textit{horizontal lift} of $X$ to $E$.
	\end{Def}
	Note that the lift is unique up to a vertical vector field. However, we can find a unique vector field lying in $\mathcal{H}$ that is the lift of $X$, which corresponds to the lift with a 0 vertical component.
	Thus, for any $x\in E,$ there is a unique horizontal vector $\mathcal{X}_{x}\in\mathcal{H}_{x}$ such that:
	$$d\pi(\mathcal{X}_{x})=X_{\pi(x)}\in T_{\pi(x)}M$$
	\begin{Obs}
		Given any principal $G$-bundle on $M$ with an horizontal distribution, we see from the previous series of isomorphisms that the tangent space of the basis manifold $M$ can be identified with the horizontal tangent space of the principal bundle.\\
		\\
		We are now going to prove that if $\mathcal{H}$ is smooth and right-invariant, the lift of any vector field will be smooth and right invariant as well.
	\end{Obs}
	\begin{Prop}
		Let $(E,M,\pi,G)$ be a principal bundle with a smooth right-invariant horizontal distribution $\mathcal{H}$. Then the horizontal lift $\mathcal{X}$ of any vector field $X\in\mathfrak{X}(M)$ is smooth and right-invariant as well.
	\end{Prop}
	\begin{proof}
		Suppose that $p\in M$ and $x\in\pi^{-1}(p)$. Then we must have by definition of lift: $d\pi(\mathcal{X}_{x})=X_p$. Suppose now to take any other point $y\in\pi^{-1}(p)$ with $x\neq y$. Then, we must have, by proposition \ref{Prop_3.5.1}, that there exists $g\in G$ such that $y=dr_{g}x=x\cdot g$. Recalling that, by observation \ref{Obs_3.5.1}, $\pi\circ r_{g}=\pi$ we have:
		$$d\pi(dr_{g}(\mathcal{X}_{x}))=d\pi\circ dr_{g}(\mathcal{X}_{x})=d\pi\mathcal{X}_{x}=d\pi\mathcal{X}_{y}$$
		By uniqueness of the horizontal lift, $\mathcal{X}_{x}=\mathcal{X}_{y}$ and this proves right-invariance.\\
		\\
		As for the smoothness, we first of all choose a trivializing chart on $M$ like $(U,\phi)$ and consider the trivialization 
		$$\varphi_U:\pi^{-1}(U)\rightarrow U\times G$$
		Now we define the vector field:
		$$Z_{(p,g)}=(X_p,0)\in T_{(p,g)}(U\times G)$$
		Clearly $Z$ is a smooth vector field on $U\times G$. Given the projection $\eta:U\times G\rightarrow U$ we have:
		$$d\eta Z_{(p,g)}=X_p$$
		\begin{comment}
			To see this, note that $d\eta:T(U\times G)\rightarrow TU$ and so we get: $d\eta(Z)\in TU$ is a vector field. In particular, by definition of differential, we must have for any $(p,g)\in U\times G$:
			$$d\eta_{(p,g)}(Z_{(p,g)})f=Z_{(p,g)}(f\circ \eta_{(p,g)})$$
			And since $\eta$ projects the points from $U\times G$ to $U$, we get:
			$$d\eta_{(p,g)}(Z_{(p,g)})=X_p$$
		\end{comment}
		We can also define a smooth vector field on $\pi^{-1}(U)$ like:
		$$Y=d\varphi^{-1}(Z)E$$
		For the same reason as before, given that $\pi$ is a projection, we must have:
		$$d\pi(Y_{x})=X_{\pi(x)}$$
		We know that the projection $Hor(Y)$ is smooth by hypothesis, If now we decompose $Y_{x}=\nu(Y_{x})+Hor(Y_{x})$ and apply the differential of the projection, recalling that $\mathcal{V}=Ker(d\pi)$, we get:
		$$d\pi(Y_{x})=d\pi(Hor(Y_{x}))=Hor(d\pi(Y_{x}))=X_{\pi(x)}$$
		This means that the smooth vector field $Hor(Y)$ is the unnique horizontal lift of $X$.
	\end{proof}
	\section{Curvature on a principal bundle}
	In this section we will introduce the notion of curvature on a principal bundle endowed with an Ehresmann connection and look at some properties of it. In the context of Y-M theory, the curvature form will be used to define the kinetic energy term of the lagrangian. For more informations on this topic the reader can look at [2] chap. 6 pag. 270-274 and [3] chap. 5 pag. 261-285.
	\begin{Def}
		Let $(E,M,\pi,G)$ be a principal $G$-bundle and $\omega$ an Ehresmann connection on it. We define the \textit{curvature form} of $\omega$ as:
		$$\Omega=d\omega+{1\over 2}[\omega,\omega]$$
	\end{Def}
	Clearly, by definition of exterior derivative and by extension of the commutator on the space of $\mathfrak{g}$-valued forms, $\Omega\in \Omega^2(E,\mathfrak{g})$. 
	\begin{Theo}\label{Theo_5.5.1}
		Let $(E,M,\pi,G)$ be a principal $G$-bundle and $\omega$ an Ehresmann connection on it and let $\Omega$ the curvature of $\omega$. Then the following properties hold:
		\begin{itemize}
			\item the curvature is \textit{horizontal}: for every $X_p,Y_p\in T_pE$,
			$$\Omega_p(X_p,Y_p)=d\omega_p(Hor(X_p),Hor(Y_p))$$
			\item $\Omega$ is \textit{$G$-equivariant}: for every $g\in G$ we have $r^*_g(\Omega)=Ad(g^{-1})(\Omega)$;
			\item the curvature respects the \textit{second Bianchi identity}: $d\Omega=[\Omega,\omega]$.
		\end{itemize}
	\end{Theo}
	\begin{proof}
		To prove the first property, let us consider all of the possible cases:
		\begin{itemize}
			\item \textit{case 1}: the vectors are horizontal. Clearly in this case $\Omega_p(X_p,Y_p)=d\omega(X_p,Y_p)$ since the connection annihilates all of the horizontal components.
			\item \textit{case 2}: one vector is horizontal and the other is vertical. Then:
			$$\Omega_p(X_p,Y_p)=\Omega_p(Hor(X_p),\nu(Y_p))=$$
			$$d\omega_p(Hor(X_p),\nu(Y_p))+{1\over 2}[\omega_p,\omega_p](Hor(X_p),\nu(Y_p))$$
			The second therm vanishes since the connection annihilates the horizontal components. As for the first therm: one can prove that if $\omega$ is a $1$-form, then:
			$$d\omega(X,Y)=X\omega(Y)-Y\omega(X)-\omega([X,Y])$$
			The proof can be found in [1] chap. 5 pag 232,233.\\
			Recall that any vertical vector field is a fundamental vector field: $Y_p\rightarrow \overline{Y}_p$. In our case, the first term is 0 since $\omega(Y_p)=\omega(\overline{Y}_p)=Y\in\mathfrak{g}$ by the defining properties of the connection. Moreover, $X_pY=0$ since its argument is a constant function of $p$. The second term vanishes because $X_p$ is horizontal. As for the first term, it vanishes as well since the commutator between an horizontal and a vertical vector field is 0 ([2] chap. 6 pag. 260).
			\item\textit{case 3}: both vectors are vertical. By properties of the connection $1$-form:
			$$\Omega(X_p,Y_p)=d\omega_p(\overline{X_p},\overline{Y_p})+{1\over 2} ([\omega(\overline{X}),\omega(\overline{Y})]-[\omega(\overline{Y}),\omega(\overline{X}_p)])=$$
			$$=d\omega_p(\overline{X_p},\overline{Y_p})+[X,Y]$$
			Moreover:
			$$d\omega(\overline{X}_p,\overline{Y}_p)=\overline{X}_p\omega(\overline{Y}_p)-\overline{X}_p\omega(\overline{Y}_p)-\omega([\overline{X}_p,\overline{Y}_p])$$	
			The first two terms are 0 since the arguments fed to the vector fields are constant functions of $p$. As for the last term, the commutator of a vertical vector field is still a vertical vector field, so that we get:
			$$d\Omega_p(X_p,Y_p)=-[X,Y]+[X,Y]=0$$
			This proves the first property.
		\end{itemize}
		As for the second property, it is just a straightforward calculation:
		$$r^*_g(\Omega)=r^*_g(d\omega)+r^*_g({1\over 2}[\omega,\omega])$$
		By proposition \ref{Prop_3.4.3}:
		$$r^*_g(\Omega)=(r^*_g\omega)+{1\over 2}[r^*_g\omega,r^*_g\omega]=Ad(g^{-1})\omega+{1\over 2}[Ad(g^{-1})\omega,Ad(g^{-1})\omega]=Ad(g^{-1})\Omega$$
		The third property can be found by a simple expansion:
		$$d\Omega=d^2\omega+{1\over 2}([d\omega,\omega]-[\omega,d\omega])=[d\omega,\omega]=[\Omega-{1\over 2}[\omega,\omega],\omega]=$$
		$$=[\Omega,\omega]-{1\over 2}[[\omega,\omega],\omega]=[\Omega,\omega]$$
	\end{proof}
	\chapter{Associated bundles}
	In this chapter we introduce the concept of associated bundle, together with some examples and properties. We will see how to use this kind of bundle to construct a covariant derivative on the underlying principal bundle. The most important associated bundle for gauge theories will be the adjoint bundle, which we will introduce later. More informations on these topics can be found in [2] chap. 6 pag. 275-286, and [3] chap. 4 and 5.
	\section{The associated bundle of a principal bundle}
	In this section we define the concept of associated bundle and look at some properties of it, together with some examples. More informations on this topic can be found in [2] chap. 6 pag. 275, 276; [3] chap. 4 pag. 237.
	\begin{Def}
		Let $(P,M,\pi,G)$ be a principal $G$-bundle and $\rho:G\rightarrow GL(V)$ a representation on a finite dimensional vector space $V$. We define the \textit{associated bundle} as the set: 
		$$E=P\times_\rho V={P\times V\over \sim}$$
		where $\sim$ is the following equivalence relation:
		$$(x,v)\sim(x\cdot g,\rho(g^{-1})v)$$ 
	\end{Def}
	\begin{Obs}
		There is a natural projection: indicating with $[x,g]$ the equivalence class of $(x,g)\in P\times V$.
		$$\pi_E:E\rightarrow M\hbox{ acting like: }\pi_E([x,g])=\pi(x)$$
		where $\pi$ is the projection of the principal bundle. Clearly, this projection is well defined as it does not depend on the choice of the representative. This is a consequence of $\pi\circ r=\pi$ for principal bundles.
	\end{Obs}
	\begin{Obs}
		It can be shown that the associated bundle has a manifold structure: REFERENZA
	\end{Obs}
	\begin{Prop}\label{Prop_6.1.1}
		Let $\rho:G\rightarrow GL(V)$ be a finite dimensional representation of a Lie group $G$, let $M$ be any manifold. Then there is a fiber-preserving diffeomorphism:
		$$\phi:(M\times G)\times_\rho V\rightarrow M\times V$$
		Acting like: $\phi([(x,g),v])=(x,\rho(g)v)$.
	\end{Prop}
	\begin{proof}
		This map is well defined:
		$$\phi([(x,g)\cdot h,\rho(h^{-1})v])=(x,\rho(gh)\rho(h^{-1})v)=(x,\rho(g)v)$$
		This map has an inverse:
		$$\psi:M\times V\rightarrow (M\times G)\times_\rho V\hbox{ acting like: }
		\psi(x,v)=[(x,1),v]$$
		Clearly: $\phi\circ \psi=\mathbb{I}$ is the identity. Now it remains to show that they commmute with the projections and that they are smooth. CONTINUA
	\end{proof}
	\begin{Theo}
		Given a principal $G$-bundle $(P,M,\pi,G)$ and a representation $\rho$, there is a vector bundle called \textit{associated bundle}, namely $(E,M,\pi_E,V)$.
	\end{Theo}
	\begin{proof}
		\begin{comment}
			The proof follows from proposition \ref{Prop_6.1.1}. In fact the principal bundle is locally trivial and this implies that the associated bundle is locally trivial as well, with fiber $V$
		\end{comment}
		REFERENZA TEOREMA
	\end{proof}
	\begin{Prop}\label{Prop_6.2.2}
		Let $(P,M,\pi,G)$ be a principal $G$-bundle and $\rho:G\rightarrow GL(V)$ a finite dimensional representation. Let $E$ be the associated bundle and $x\in M$. Then, for every point $p$ in the fiber $P_x$, there is a linear isomorphism:
		$f_p:V\rightarrow E_x$, acting like $f_p(v)=[p,v]$ where $E_x$ is the fiber of $E$ at $x$.
	\end{Prop}
	\begin{proof}
		This map is clearly linear. To prove injectivity, take $[p,v]=[p,w]$ points in $E_x$. Then $(p,w)=(p\cdot g,\rho(g^{-1})v)$ for some $g\in G$. However, the action of $G$ on $P$ is free and so $g=e$. Thus, $w=v$. This proves injectivity. To prove surjectivity, take any $[q,w]\in E_x$. We have $q=p\cdot g$ for some $g\in G$ and so we have $[q,w]=[p\cdot g,w]=[p,\rho(g)w])=f_p(v)$.
	\end{proof}
	\begin{Obs}\label{Obs_6.2.3}
		It is not hard to see that $f_{p\cdot g}=f_p\circ \rho(g)$. In fact:
		$$f_{p\cdot g}(v)=[p\cdot g,v]=[p,\rho(g)v]=f_p(\rho(g)v)$$
	\end{Obs}
	\section{Tensorial forms}
	In this section we will define the concept of tensorial forms on a principal bundle. We will find a connection between those forms and forms on the associated bundle. More information on this topic can be found in [2] chap. 6 pag. 277-280; and [3] chap. 5, pag. 261-270.
	\begin{Def}
		Let $(P,M,\pi,G)$ be a principal $G$-bundle on $M$ and $\rho$ a representation of $G$.  Then any $V$-valued $k$-form $\omega$ on $P$ is said to be \textit{right-equivariant of type $\rho$} if:
		$$r^*_g\omega=\rho(g^{-1})\omega$$
		instead $\omega$ is said to be \textit{horizontal} if it vanishes whenever one of it's arguments is a vertical vector. If a form is both right-equivariant of type $\rho$ and vertical it is called \textit{tensorial of type $\rho$}. We label the space of all smooth tensorial $k$-forms of type $\rho$ with $\Omega^k_\rho(P,V)$.
	\end{Def}
	\begin{Ex}
		By theorem \ref{Theo_5.5.1}, the curvature $\Omega$ is a tensorial $2$-form of type Ad.
	\end{Ex}
	\begin{Theo}(\textbf{Musical isomorphism})\label{Mus_Iso}
		There is a well defined linear isomorphism:
		$$\Omega^k(M,E)\simeq \Omega^k_\rho(P,V)$$
	\end{Theo}
	\begin{proof}
		Let $\omega\in\Omega^k_\rho(P,V)$, $x\in M$ and $v_1,...,v_k\in T_xM$ be tangent vectors. For any point in the fiber $p\in P_x$ select some lifted vectors $u_i$ such that $d\pi(u_i)=v_i$. Define $$\omega^\flat_x(v_1,...,v_k)=f_p(\omega_p(u_1,...,u_k))$$
		This map is well defined as it does not depend neither on the fiber point $p\in P$, nor on the lift. To see this, suppose to take any other set of lifted vectors $\{u'_i\}$ such that $d\pi(u'_i)=v_i$. Then we must have that $u_i-u_i'$ are  all vertical due to
		$$d\pi(u_i-u'_i)=0$$
		However, by definition, $\omega$ is horizontal and multilinear, so that:
		$$\omega_p(u_1+vert.,...,u_k+vert.)=\omega_p(u_1,...,u_k)$$
		it does not depend on the choice of the lift. To see that this map does not depend on the point of the fiber, choose any other point $q\in P_x$. Then, there will exist a $g\in G$ such that $p\cdot g=q$. Since $\pi\circ dr=\pi$, the lifted vectors of $v_i$ will be $dr_g(u_i)$. Evaluating:
		$$f_{p\cdot g}(\omega_p(dr_g(u_1),...,dr_g(u_k)))=
		f_{p\cdot g}(r^*_g\omega_p(u_1,...,u_k))$$
		By proposition \ref{Obs_6.2.3} and by assumed equivariance of $\omega$:
		$$f_{p\cdot g}(r^*_g\omega_p(u_1,...,u_k))=f_p\circ \rho(g)\circ \rho(g^{-1})\omega(u_1,...,u_k)=f_p(\omega(u_1,...,u_k))$$
		To show that the association $\omega\rightarrow\omega^\flat$ is an isomorphism, we construct its inverse. Suppose $\alpha\in\Omega^k(M,E)$ is a $E$-valued $k$-form. Then one can define $\alpha^\sharp\in \Omega^k_\rho(P,V)$ as follows. Take any $p\in P$ and $x=\pi(p)\in M$ and any set $u_1,...,u_k\in T_pP$. Then we set:
		$$\alpha^\sharp_p(u_1,...,u_k)=f^{-1}_p(\alpha(d\pi(u_1),...,d\pi(u_k)))$$
		From this definition, $\alpha^\sharp$ is clearly horizontal. The right-equivariance follows obviously from the properties of $f$. Now it remains to check that $\omega^{\flat\sharp}=\omega$ and $\alpha^{\sharp\flat}=\alpha$.\\
		Let $\omega\in\Omega_\rho^k(P,V)$, $x\in M$, $p\in P_x$, $v_i\in T_xM$ and $u_i$ such that $d\pi(u_i)=v_i$.
		Then, $(\omega^\flat)^\sharp_p(u_1,...,u_k)=f^{-1}_p(\omega^\flat_x(d\pi(u_1),...,d\pi(u_k)))$ by definition. Substituting further we get:
		$$(\omega^\flat)^\sharp_p(u_1,...,u_k)=
		f^{-1}_p(f_p(\omega_p(u_1,...,u_k)))=\omega_p(u_1,...,u_k)$$
		Analogously one proves that $\alpha^{\sharp\flat}=\alpha$. This completes the proof.
	\end{proof}
	\section{The covariant derivative}
	In this section we will define the notion of covariant derivaative on a principal bundle and see some properties of it. More informations on this topic can be found in [2] chap. 6 pag 280-286.
	\begin{Prop}
		Let $\omega$ be a right equivariant form of type $\rho$ on a principal bundle $(P,M,\pi,G)$. Then, $d\omega$ is right equivariant of type $\rho$ as well.
	\end{Prop}
	\begin{proof}
		The exterior derivative commutes with the pullback, so this claim follows immediately from proposition \ref{Prop_3.4.3}.
	\end{proof}
	\begin{Obs}
		In general, the exterior derivative does not preserve horizontality.
	\end{Obs}
	\begin{Def}
		Let $(P,M,\pi,G)$ be a principal bundle and $\omega\in \Omega^k(P,V)$. We define the \textit{horizontal component} of $\omega$ as the form $\omega^h$ such that for $p\in P$ and $v_1,..,v_k\in T_pP$:
		$$\omega^h_p(v_1,...,v_k)=\omega(Hor(v_1),..,Hor(v_k))$$
	\end{Def}
	\begin{Prop}
		Let $\omega$ be a right equivariant form of type $\rho$ on a principal bundle $(P,M,\pi,G)$. Then, $\omega^h$ is equivariant form of tyoe $\rho$ as well.
	\end{Prop}
	\begin{proof}
		The proof follows immediately from proposition \ref{Prop_5.3.1}.
	\end{proof}
	\begin{Obs}
		As a corollary of the last statement, if $\omega$ is right-equivariant of type $\rho$, then $(d\omega)^h$ is tensorial of type $\rho$. In particular:
		$$\omega\in \Omega^k_\rho(P,V)\Rightarrow (d\omega)^h\in\Omega^{k+1}_\rho(P,V)$$
	\end{Obs}
	\begin{Def}
		Let $(P,M,\pi,G)$ be a principal bundle adn $\rho:G\rightarrow GL(V)$ a representation of $G$. We define the \textit{covariant derivative} or \textit{exterior covariant derivative} of a $k$-form with values in $V$ the map:
		$$D:\Omega^k(P,V)\rightarrow \Omega^{k+1}(P,V)$$
		$$\omega\rightarrow (d\omega)^h$$
	\end{Def}
	It is clear from the previous propositions that: $D:\Omega^k_\rho(P,V)\rightarrow \Omega^{k+1}_\rho(P,V)$ the covariant derivative restricts to a map on tensorial forms of type $\rho$.
	\begin{Prop}
		The covariant derivative of a tensorial form of type $\rho$ is an antiderivation of degree 1.
	\end{Prop}
	\begin{proof}
		Consider $\omega\in\Omega^k_\rho(P,V)$ and $\tau\in \Omega^l_\rho(P,V)$, then:
		$$D(\omega\wedge \tau)=(d(\omega\wedge\tau))^h=(d\omega\wedge \tau+(-)^{deg\omega}\omega\wedge  d\tau)^h=$$
		$$=(d\omega)^h\wedge \tau^h+(-)^{deg\omega}\omega^h\wedge(  d\tau)^h$$
		Since $\omega,\tau$ are by hypothesis horizzontal, we have that $\omega^h=\omega$ and $\tau^h=\tau$. Thus:
		$$D(\omega\wedge \tau)=(d\omega)^h\wedge \tau+(-)^{deg\omega}\omega\wedge (d\tau)^h$$
	\end{proof}
	\begin{Obs}
		We know there is an isomorphism, by theorem \ref{Mus_Iso}, between:
		$$\Omega^k_\rho(P,V)\simeq\Omega^k(M,E)$$ where $E$ is the principal bundle induced by $\rho$. This means that the covariant derivative restricted to tensorial forms of type $\rho$ induces a linear mapping:
		$$D:\Omega^k(M,E)\rightarrow\Omega^{k+1}(M,E)$$
	\end{Obs}
	\begin{Ex}\label{Ex_6.3.1}
		Consider a principal bundle with a connection $A$ on it. By definition, $A$ is ad $Ad$-equivariant smooth form, but it is not horizontal. Thus, $A$ does not belong to $\Omega^1_{Ad}(P,\mathfrak{g})$. The curvature of $A$ was defined as:
		$F=dA+{1\over 2}[A,A]$ and was proved to be equivariant and horizontal in \ref{Theo_5.5.1}. Moreover, from the same result:
		$$F(X,Y)=dA(Hor(X),Hor(Y))=(dA)^h(X,Y)=DA(X,Y)$$
		Thus, $F$ is the covariant derivative of $A$.
	\end{Ex}
	We now look at a general formula for the covariant derivative of a tensorial form. In practice, we will work with $Ad$-invariant forms.
	\begin{Def}\label{Def_6.3.3}
		Let $(P,M,\pi,G)$ be a principal bundle and $\rho:G\rightarrow GL(V)$ a representatioan for $G$. Let $\omega\in \Omega^k(P,V)$, $\tau\in\Omega^l(P,V)$ be vector valued forms and let $d\rho$ be the induced representation on the Lie algebra $\mathfrak{g}$. We define the following product: given $p\in P$ and $v_1,...,v_{k+l}\in T_pM$:
		$$\omega\cdot \tau={1\over k!l!}\sum_\sigma sgn(\sigma)d\rho(\omega(v_{\sigma(1)},...,v_{\sigma(k)}))\tau(v_{\sigma(k+1),...,v_{\sigma(k+l)}})$$
	\end{Def}
	This product is by definition alternating and multilinear, so it is a k-covector with values in $V$.
	\begin{Obs}
		Clearly, in the case of $\rho=Ad$ being the adjoint representation, we get:
		$$\omega\cdot \tau={1\over k!l!}\sum_\sigma sgn(\sigma)[\omega(v_{\sigma(1)},...,v_{\sigma(k)}),\tau(v_{\sigma(k+1),...,v_{\sigma(k+l)}})]$$
		And in this case we also write $[\omega,\tau]$ instead of $\omega\cdot \tau$
	\end{Obs}
	\begin{Prop}\label{Prop_6.3.4}
		Let $(P,M,\pi,G)$ be a principal bundle with a connection $A$ and $\rho:G\rightarrow Gl(V)$ a representation on $G$. If $\omega\in\Omega^k_\rho(P,V)$ then:
		$$D\omega=d\omega+A\cdot \omega$$
	\end{Prop}
	\begin{proof}
		Let $p\in P$ and $v_1,...,v_{k+1}\in T_pP$. We need to show that:
		$$(d\omega)^h_p(v_1,...,v_{k+1})=(d\omega)_p(v_1,...,v_{k+1})+{1\over k!}\sum_\sigma sgn(\sigma)d\rho(A_p(v_{\sigma(1)}))\omega_p(v_{\sigma(2)},...,v_{\sigma(k+1)})$$
		By linearity, we can decompose any vector into the sum of horizontal and vertical vectors, so in the proof it suffices to distinguish between 3 main cases:
		\begin{itemize}
			\item Case 1: the vectors $v_i$ are all horizontal. In this case the term 
			$${1\over k!}\sum_\sigma sgn(\sigma)d\rho(A_p(v_{\sigma(1)}))\omega_p(v_{\sigma(2)},...,v_{\sigma(k+1)})$$
			is 0 since the connection annihilates the horizontal vectors. The equality is thus obvious.
			\item Case 2: all $v_i$ are horizontal except for one vector. Suppose this vector is $v_1$ without loss of generality. The first term in the above relation can be decomposed ass follows ([1] chap. 5 pag. 233): 
			$$d\omega_p(v_1,...,v_{k+1})=\sum_i (-)^{i+1}v_i\omega_p(v_1,...,v_{i-1},v_{i+1},...,v_{k+1})+$$
			$$+\sum_{i<j}\omega_p([v_i,v_j],v_1,...,v_{i-1},v_{i+1},...,v_{j-1},v_{j+1},...,v_k)$$
			Since $\omega$ is assumed to be horizontal, the only non 0 term in the first sum is $v_1\omega(v_2,...,v_k)$. As for the second sum, for the same reason, the only non 0 terms are:
			$$\sum_{j=2}^{k+1}(-)^{j+1}\omega_p([v_1,v_j],v_2,...,v_{j-1},v_{j+1},...,v_{k+1})$$
			However, the commutator between horizontal and vertical vectors is 0 ([2] chap. 6 pag. 260). This completely annohilates the second sum. We thus get:
			$$d\omega_p(v_1,...,v_{k+1})=v_1\omega_p(v_2,...,v_{k+1})$$
			Now, let us call $f$ the map $\omega(v_2,...,v_{k+1})$. The map $f$ is then a smooth map on $P$. Let $\gamma$ be the smooth curve in $G$ with initial point $\gamma(0)=e$ and initial tangent vector $\gamma'(0)=X$. Consider the following:
			$$v_1f(p)=dj_p(X)f(p)=dj_p\bigg( d\gamma\bigg({d\over dt}\bigg|_0\bigg)\bigg)f=d(j_p\circ \gamma)\bigg({d\over dt}\bigg|_0\bigg)f={d\over dt}\bigg|_0 (f\circ j_p\circ \gamma)=$$
			$$={d\over dt}\bigg|_0(f(p\cdot \gamma))={d\over dt}\bigg|_0\omega_{p\cdot \gamma(t)}(dr_{\gamma(t)}v_2,...,dr_{\gamma(t)}v_{k+1})$$
			Where in the last line we have used the right-invariance of the horizontal vector fields. One can express the previous form in the pullback notation and use the right-equivariance:
			$${d\over dt}\bigg|_0\omega_{p\cdot \gamma(t)}(dr_{\gamma(t)}v_2,...,dr_{\gamma(t)}v_{k+1})={d\over dt}\bigg|_0r^*_{\gamma(t)}\omega_{p}(v_2,...,v_{k+1})=$$
			$$={d\over dt}\bigg|_0\rho(\gamma^{-1}(t))\omega_{p}(v_2,...,v_{k+1})=-d\rho(X)\omega_{p}(v_2,...,v_{k+1})$$
			As for the term
			$${1\over k!}\sum_\sigma sgn(\sigma)d\rho(A_p(v_{\sigma(1)}))\omega_p(v_{\sigma(2)},...,v_{\sigma(k+1)})$$
			by verticality of the connection $A$, the only non 0 terms are the ones for $\sigma(1)=1$, so that the above formula becomes::
			$${1\over k!}\sum_\sigma sgn(\sigma)d\rho(A_p(v_1)\omega_p(v_{\sigma(2)},...,v_{\sigma(k+1)})$$
			However, since $v_1$ is vertical, it is the fundamental vector field of an element $X$ of $\mathfrak{g}$ and by the defining properties of the connection:
			$$d\rho(A(v_1))=d\rho(X)$$
			Since the form $\omega$ is alternating, the previous sum becomes:
			$$d\rho(X)\omega_p(v_2,..,v_{k+1})$$
			Clearly, one gets $D\omega_p(v_1,...,v_{k+1})=0$ by summing the previous terms.
			\item Case 3: at least two vectors $v_i,v_j$ are vertical. Let us assume without loss of generality that $v_1,v_2$ are vertical. Then, we split the exterior covariant derivative as usual:
			$$(d\omega)^h_p(v_1,...,v_{k+1})=(d\omega)_p(v_1,...,v_{k+1})+{1\over k!}\sum_\sigma sgn(\sigma)d\rho(A_p(v_{\sigma(1)}))\omega_p(v_{\sigma(2)},...,v_{\sigma(k+1)})$$
			The first term can be rewritten as before:
			$$d\omega_p(v_1,...,v_{k+1})=\sum_i (-)^{i+1}v_i\omega_p(v_1,...,v_{i-1},v_{i+1},...,v_{k+1})+$$
			$$+\sum_{i<j}\omega_p([v_i,v_j],v_1,...,v_{i-1},v_{i+1},...,v_{j-1},v_{j+1},...,v_k)$$
			The first sum is 0 since at least one argument of $\omega$ is horizontal. The second sum is 0 as well since the commutator between two veritcal vector fields is vertical and all of the term which involve commutators between horizontal and vertical fields are 0
			The last term in the expansion is also 0 since at least one argumento of $\omega$ is always vertical.
		\end{itemize}
	\end{proof}
	\begin{Obs}\label{Obs_6.3.5}
		The exterior covariant derivative does not square to 0. In fact, given a connection $A$ and a $k$-form $\omega$ with values in $V$:
		$$D^2\omega=D(d\omega+A\cdot \omega)=d^2\omega+d(A\cdot \omega)+A\cdot d\omega+A\cdot A\cdot \omega$$
		A priori only the first term is 0. bhowever, one can prove that $DF=0$, where $F$ is the curvature of a connection $A$. Since $F$ is an $Ad$-tensorial $2$-form we can apply the formula shown in proposition \ref{Prop_6.3.4} in the associated adjoint bundle and get:
		$$DF=dF+[A,F]$$
		However, by theorem \ref{Theo_5.5.1}, $dF=[F,A]$ and so $DF=0$.
	\end{Obs}
	\chapter{Gauge transformations}
	In this chapter we will introduce the concept of gauge transformations. Furthermore, we will see how the connection 1-forms and the associated curvatures will behave under such transformations. More informations on those topics can be found in [3] chap. 5.
	\section{Gauge transformations on a principal bundle}
	In this section we use the previous knowledge to construct the notion of gauge transformations. We will also see some properties of connections and curvature on principal bundles, which will be useful in the context of Yang-Mills theory. More informations on this topic can be found in [3] chap. 5 and [2] chap. 6.
	\begin{Def}
		Let $(P,M,\pi,G)$ be a principal $G$-bundle. We call a \textit{gauge transformation} any bundle automorphism, i.e a diffeomorphism $f:P\rightarrow P$ such that:
		\begin{itemize}
			\item $f$ is fiber preserving: $\pi\circ f=\pi$;
			\item $f$ is $G$-equivariant: $f(p\cdot g)=f(p)\cdot g$ for any $p\in P, g\in G$.
		\end{itemize}
		A gauge transformation $f$ is called \textit{local} if it is defined on an open subset $U\subset P$. If the gauge transformation is defined on the whole principal bundle then it is called \textit{global}.
	\end{Def}
	\begin{Obs}
		The set of all possible gauge transformations is a group under the composition of diffeomorphism and is denoted with $\mathcal{G}(P)$. This is in general an infinite dimensional Lie group.
	\end{Obs}
	We now try to give an alternative, more useful, description of gauge transformations.
	\begin{Def}
		Let $(P,M,\pi,G)$ be a principal $G$-bundle. We define the set of \textit{smooth gauge $G$-valued maps on $P$} as:
		$$C^\infty(P,G)^G=\{\sigma:P\rightarrow G|\sigma\hbox{ is smooth },\sigma\circ r_g=c_{g^{-1}}\circ \sigma\}$$
	\end{Def}
	\begin{Obs}
		This set is clearly a group under the pointwise multiplication. Namely:
		$\sigma\cdot \sigma'(p)=\sigma(p)\cdot \sigma'(p)$.
	\end{Obs}
	\begin{Prop}
		There is a well defined group isomorphism:
		$$\mathcal{G}(P)\longrightarrow C^\infty(P,G)^G\hbox{ acting like }f\rightarrow \sigma_f$$
		where $f(p)=p\cdot \sigma_f(p)$.
	\end{Prop}
	\begin{proof}
		Let $f\in\mathcal{G}(P)$, then $f$ is fiber preserving. This means that $f(p)$ is in the same fiber of $p$ and so there will be a $g\in G$ such that $f(p)=p\cdot g$. Define $g=\sigma_f(p)$. Now we show that $\sigma_f(p)\in C^\infty(P,G)^G$. This map is clearly smooth and goes from $P$ to $G$. Moreover:
		$$(p\cdot h)\sigma_f(p\cdot h)=f(p\cdot h)=f(p)\cdot h=p\cdot \sigma_f(p)\cdot h$$
		This in turn implies that $\sigma\in C^\infty(P,G)^G$ since:
		$$\sigma(r_h(p))=c_{h^{-1}}\sigma_f(p)$$
		The inverse mapping instead goes like:
		$$\sigma_f\rightarrow f$$
		Clearly, $f$ defined by the above relation is fiber preserving. To show the equivariance one computes:
		$$f(p\cdot g)=(p\cdot g)\sigma(p\cdot g)=p\cdot h^{-1}h\cdot\sigma(p)\cdot h=f(p)\cdot h$$
		And so $f$ lies in $\mathcal{G}(P)$.
	\end{proof}
	This result allow us to represent gauge transformations with maps in $C^\infty(P,G)^G$. However, we can look at yet another isomorphism, which will allow us to see gauge transformations as map from the base manifold $M$ to $G$.
	\begin{Def}
		Let $(P,M,\pi,G)$ be a principal $G$-bundle. We call a \textit{physical gauge transformation} any smooth map $\tau\in C^\infty(U\subset M,G)$, where $U$ is an open subset of $M$.
	\end{Def} 
	\begin{Prop}\label{Prop_7.1.2}
		Let $(P,M,\pi,G)$ be a principal $G$-bundle and $s:U\rightarrow P$ a smooth local gauge. Then there is an induced isomorphism:
		$$C^\infty(P|_U,G)^G\longrightarrow C^\infty(U,G)\hbox{ acting like: }\sigma\rightarrow \sigma\circ s$$
		The inverse mapping is $\tau\rightarrow \sigma$ where: 
		$$\sigma\circ r_g\circ s=c_{g^{-1}}\circ\tau\hbox{ for every } g\in G$$
	\end{Prop}
	\begin{proof}
		Let $\sigma\in C^\infty(P|_U,G)^G$ and $s:U\rightarrow P$ be a smooth local gauge. The map $\sigma\circ s$ is clearly smooth and so it is indeed an element of $C^\infty(U,G)$.\\
		Now we construct the inverse mapping: let $\tau\in C^\infty(U,G)$. Then we define 
		$$\sigma\circ r_g\circ s=c_{g^{-1}}\circ\tau$$
		Clearly, since the above property holds for every $g\in G$, we have:
		$$\sigma\circ r_g\circ s=c_{g^{-1}}\circ \tau=c_{g^{-1}}\circ \sigma\circ s$$
		This proves that the inverse map is well defined. Lastly, those maps are clearly one the inverse of the other. This proves the claim.
	\end{proof}
	\begin{Obs}
		The previous result tells us that after selecting a local gauge, we can express every bundle automorphism as a $G$-valued map on the manifold. Moreover, by observation \ref{Obs_3.5.2}, we know that two local gauges $s_{i,j}:U\rightarrow P$ defined on the same open set, differ by a smooth map: $$s_i(x)=s_j(x)\cdot g_{ji}(x)$$
		Thus, any gauge transformaation corresponds to a change in the local gauge.
	\end{Obs}
	Now we briefly look at how a gauge transformation acts on the associated bundle of a principal bundle.
	\begin{Prop}\label{Prop_7.1.3}
		Let $(P,M,\pi,G)$ be a principal $G$-bundle and $\rho$ a representation of $G$. Let $E=P\times_\rho V$ be the associated bundle under this representation. Then there is a natural action of the gauge group on the associated bundle:
		$$\mathcal{G}(P)\times E\rightarrow E\hbox{ acting like: }(f,[p,v])\rightarrow [f(p),v]=[p\cdot \sigma_f(p),v]$$
	\end{Prop}
	\begin{proof}
		The map is well defined since it does not depend on the representative:
		$$[f(q),w]=[f(p\cdot g),\rho(g^{-1})v]=[f(p)\cdot g,\rho(g^{-1})v]$$
	\end{proof}
	\section{Gauge transformations of forms}
	In this section we compute how the connection $1$-forms and the curvature change under a change of local gauge. More information on this topic can be found in: [3] chap. 5 pag. 270-285.
	\begin{Obs}
		Consider a connection $1$-form $A$ on a principal bundle $(P,M,\pi,G)$. Then, $A$ is an element of $\Omega^1(P,\mathfrak{g})$. For any local gauge $s:U\rightarrow P$, we can define a $\mathfrak{g}$-valued form on the manifold through the pullback:
		$$A_M=s^*A;\hbox{ so that }A_M\in \Omega^1(U,\mathfrak{g})$$
		Clearly, given a basis for the Lie algebra $\{T^a\}$, we can expand:
		$$A_M=A_{Ma}\otimes T^a$$
		So, given any section of the principal bundle, we can pullback the connection $1$-form to a form on the manifold, which however depends on the section.
	\end{Obs}
	\begin{Prop}\label{Prop_7.2.1}
		Let $G$ be a Lie group right-acting on a manifold $P$ and let $\mu:P\times G\rightarrow P$ be the action. Then the differential of the action is given by:
		$$d\mu_{(p,g)}:T_pP\oplus T_gG\rightarrow T_{p\cdot g}P$$
		$$d\mu_{(p,g)}(X_p,dl_gA)=dr_g(X_p)+(\overline{A}_{pg})$$
		where $X_p\in T_pP,A\in\mathfrak{g}$
	\end{Prop}
	\begin{proof}
		The differential of the action is by definition a map like:
		$$d\mu_{(p,g)}:T_pP\oplus T_gG\rightarrow T_{p\cdot g}P$$
		Thus, any element of $T_pP\oplus T_gG$ can be seen as $(X_p,Y_g)$, where $Y_g$ can be related to a specific element of the Lie algebra $\mathfrak{g}$ through the diffeomorphism $dl_g$ given by the left translation: $Y_g=dl_gA$.
		By definition of differential:
		$$d\mu_{(p,g)}(X_p,dl_gA)f=(X_p,dl_gA)(f\circ \mu(p,g))=(X_p(f\circ r_g(p)),dl_gA(f\circ l_p(g)))$$
		where we have indicated with $l_p(g)=p\cdot g$ the left translation with $p$. Clearly, the first term can be rewritten like: $dr_gX_p(f)$. As for the second term instead, by definition of differential:
		$$dl_gA(f\circ l_p(g))=dl_p\circ dl_gA(f)$$
		However, $dl_p\circ dl_g=dj_{pg}$ and so we get the result:
		$$d\mu_{(p,g)}(X_p,dl_gA)=dr_g(X_p)+\overline{A}_{p\cdot g}$$
	\end{proof}
	\begin{Theo} \label{Theo_7.2.1}
		Let $(P,M,\pi,G)$ be a principal $G$-bundle and $A$ a connection on it. Let $s_1,s_2:U\rightarrow P$ be local gauges and $A_{i}=s_i^*A$ the pulled-back connections on the manifold. Then 
		$$A_i=Ad(g_{ji}^{-1})A_j+\mu_{ji}$$
		Where $g_{ji}$ is the transition function between the local trivializations $s_i,s_j$ while the form $\mu_{ji}=g_{ji}^*\theta$ is the pullback of $\theta$ the Maurer-Cartan form.
	\end{Theo}
	\begin{proof}
		By construction, $A_i=s_i^*A$ so that for any vector field $X\in \Gamma(U)$, we have:
		$$s^*A(X)=A(ds(X))$$
		The two sections induce trivializations by proposition \ref{Prop_3.5.3}. Furthermore, by observation \ref{Obs_3.5.3}, we know that the relation between the two sections at any point $x\in U$ is:
		$$s_i(x)=s_j(x)\cdot g_{ji}(x)=\mu(s_j(x),g_{ji}(x))$$
		By taking the differential:
		$$ds_{i,x}(X_x)=d\mu_{(s_{j}(x), g_{ji}(x))}(ds_{j,x}(X_x),dg_{ji,x}(X_x))$$
		Now, the mapping $g_{ji}:M\rightarrow G$ has as differential $dg_{ji,x}:T_xM\rightarrow T_{g_{ji}(x)}G$, so that it takes $X_x$ into a tangent vector to $G$ at $g_{ji}(x)$. Thus, since for every $h\in G$ the left action is a diffeomorphism, there exists an element of the Lie algebra, which we will call $T$, such that 
		$$dl_{g_{ji}(x)}(T)=dg_{ji,x}(X_x)$$
		This in turn implies:
		$$T=dl_{g^{-1}_{ji}(x)}\circ dg_{ji,x}(X_x)$$
		Moreover, applying the result obtained in proposition \ref{Prop_7.2.1} we find:
		$$d\mu_{(s_{j}(x), g_{ji}(x))}(ds_{j,x}(X_x),dg_{ji,x}(X_x))=d\mu_{(s_{j}(x), g_{ji}(x))}(ds_{j,x}(X_x),dl_{g_{ji}(x)}(T))=$$
		$$=dr_{g_{ji}(x)}\circ ds_{j,x}(X_x)+\overline{T}_{s_{j}(x)\cdot g_{ji}(x)}$$
		Finally, feeding this to $A_i$, we get:
		$$A_{i,s_i(x)}(dr_{g_{ji}(x)}\circ ds_{j,x}(X_x)+\overline{T}_{s_{j}(x)\cdot g_{ji}(x)})=r^*_{g_{ji}(x)}A_{i,{s_i(x)}}(ds_{j,x}(X_x))+T$$
		Where in the last line we have applied the defining properties of the connection $1$-forms. Substituting back the expression for $T$ we get:
		$$r^*_{g_{ji}(x)}A_{i,{s_i(x)}}(ds_{j,x}(X_x))+dl_{g^{-1}_{ji}(x)}\circ dg_{ji,x}(X_x)=r^*_{g_{ji}(x)}A_{i,{s_i(x)}}(ds_{j,x}(X_x))+(g^*_{ji}\theta)_x(X_x)$$
		By once again applying the defining properties of the connection:
		$$A_i=Ad(g_{ji}^{-1})A_j+\mu_{ji}$$
		This completes the proof.
	\end{proof}
	\begin{Obs}\label{Obs_7.2.2}
		Recall that the curvature of a connection is defined as:
		$$F=dA+{1\over 2}[A,A]$$
		Knowing the transformation rule for the pulled-back connection, we can find the analogue for the curvature: ler $s_{i,j}:U\rightarrow P$ be two local gauges, then: $F_{i,j}=s_{i,j}^*F$ and we get, by applying the results found in proposition \ref{Prop_3.4.3}:
		$$s^*F=ds^*A+{1\over 2}[s^*A,s^*A]$$
		By substituting the transformation rules for the connection one gets:
		$$s_i^*F=F_i=d(Ad(g_{ji}^{-1})A_j)+d\mu_{ji}+{1\over 2}[Ad(g_{ji}^{-1})A_j+\mu_{ji},Ad(g_{ji}^{-1})A_j+\mu_{ji}]=$$
		$$=dr^*_{g_{ji}}A_j+{1\over 2}[r^*_{g_{ji}}A_j,r^*_{g_{ji}}A_j]+d\mu_{ji}+{1\over 2}[\mu_{ji},\mu_{ji}]$$
		Using again proposition \ref{Prop_3.4.3} and expressing $\mu_{ji}=g_{ji^*\theta}$ we find:
		$$s_i^*F=F_i=r^*_{g_{ji}}F_j+g_{ji}^*(d\theta+{1\over 2}[\theta,\theta])$$
		Finally, by example \ref{Ex_3.4.2}, the last term is 0 since it is the curvature induced by the Maurer-Cartan form. From the $G$-equivariance of the curvature found in theorem \ref{Theo_5.5.1}, we thus get:
		$$F_i=Ad(g_{ji}^{-1})F_j$$
	\end{Obs}
	\begin{Prop}\label{Prop_7.2.2}
		Let $(P,M,\pi,G)$ be a principal $G$-bundle and $G$ an abelial Lie group. Then, given a connection $A$ on $P$, the pullback of its curvature $F$ is independent of the choice of the local gauge.
	\end{Prop}
	\begin{proof}
		If $G$ is abelian, $F$ is gauge invariant. This means that for any change of local section, the pullback of $F$ remains invariant and so $F$ is defined gobally as a closed 2-form on $M$: $F\in \Omega^2(M,\mathfrak{g})$.
	\end{proof}
	Lastly, we would like to understand how the curvature transforms from the point of view of the associated bundle. Recall that proposition \ref{Mus_Iso} gave us the following isomorphism:
	$$\Omega^k_\rho(P,V)\simeq \Omega^k(M,E)$$
	The curvature of a connection is, as proved in proposition \ref{Theo_5.5.1}, $Ad$-equivariant and thus belongs to $\Omega^2_{Ad}(P,\mathfrak{g})$. This implies that once we have a connection on $P$ principal bundle, we can construct a $2$-form on the associated bundle $F_M$.
	\begin{Prop}
		Let $F\in \Omega^2_Ad(P,\mathfrak{g})$ be a curvature form on a principal bundle and $F_m\in\Omega^2(M,E)$ be the corresponding $2$-form on the associated bundle. The action of $\phi\in\mathcal{G}(P)$ on $F_M$ is induced by the one of $F$ and is:
		$$F\rightarrow \phi^{-1}\cdot F$$
	\end{Prop}
	\begin{proof}
		Let $s:U\rightarrow P$ be a local gauge. By observation \ref{Obs_7.2.2}, the action of $\phi\in\mathcal{G}$ on $F$ is:
		$$F\rightarrow Ad(g^{-1})F$$
		where $g\in C^\infty(U,G)$ is the smooth map corresponding to $\phi$ in the isomorphism of proposition \ref{Prop_7.1.2}. The form $F_M$ is constructed through theorem \ref{Mus_Iso} as follows. Let $x\in M,X_x,X_y\in T_xM$, $p\in P_x$ and $\tilde{X}_p,\tilde{Y}_p\in T_pP$ their horizontal lifts. We have:
		$$F_{M,x}(X_x,Y_x)=[p,F_p(\tilde{X}_p,\tilde{Y}_p)]$$
		Now, applying $\phi$ we get:
		$$[p,Ad(g^{-1})F_p(\tilde{X}_p,\tilde{Y}_p)]\sim [p\cdot \sigma^{-1}_\phi(p),F_p(\tilde{X}_p,\tilde{Y}_p)]=[\phi^{-1}(p),F_p(\tilde{X}_p,\tilde{Y}_p)]$$
		But this is exactly the action of $\phi^{-1}$ on $E$ of proposition \ref{Prop_7.1.3}. Thus, we get:
		$$F_M\rightarrow \phi^{-1}\cdot F_M$$
	\end{proof}
	\section{Products on associated bundles}
	In this section we will define some operations on forms that take values on the associated bundle of a principal bundle. Those will be useful to define the full Y-M lagrangian. More informations on this topic can be found in [3] chap. 7 pag. 409-413.
	\begin{Def}
		Let $(E,M,\pi,\mathbb{R}^n)$ a vector bundle. We call a \textit{bundle metric} a section $\braket{,}_E\in \Gamma(E^*\otimes E^*)$. This is a smooth assignment to each point of a non degenerate, symmetric, bilinear product on the fiber.
	\end{Def}
	There is a canonical choice for this metric in the case of an associated bundle induced by a representation on a vector space endowed with an inner product.
	\begin{Prop}
		Let $(P,M,\pi,G)$ be a principal bundle and $(E,M,\pi_E,V)$ be the induced associated bundle from $\rho:G\rightarrow GL(V)$. Let $V$ be a vector space with an inner product $\braket{,}_V$. Then there is a natural choice for a bundle metric on each fiber: let $x\in M$ and $p\in P_x$, we set 
		$$\braket{[p,v],[p,w]}_{E_x}=\braket{v,w}_V$$
	\end{Prop}
	\begin{proof}
		This product is well defined as it does not depend on the choice of the representative. To see this it suffices to rewrite the product through the use of the isomorphism of proposition \ref{Prop_6.2.2}. Moreover, the non degeneracy, symmetry and bilinearity follow from the definition.
	\end{proof}
	\begin{Ex}
		If we consider the adjoint associated bundle of a principal bundle, we have.
		$$\braket{[p,v],[p,w]}_{Ad}=\braket{v,w}_\mathfrak{g}$$
		From theorem \ref{Theo_2.8.2} we know that if $\mathfrak{g}$ is compact and semisimple, we can use the Killing form as a non degenerate inner product. This induces the following bundle metric:
		$$\braket{[p,v],[p,w]}_{Ad}=Tr(\hbox{ad}_v\circ\hbox{ad}_w)$$
		Moreover, the Killing form, from proposition \ref{Prop_2.8.2}, is invariant under the action of Lie algebra automoprhism. In particular, it is $Ad$-invariant. This immediately implies by proposition \ref{Prop_7.1.3} that the scalar product induced in the adjoint bundle $Ad(P)$ is invariant under gauge transformations, since they act with the adjoint representation.
	\end{Ex}
	\begin{Def}
		Let $(P,M,\pi,G)$ a principal bundle and $\rho$ a representation of $G$. Let $\omega_{1,2}\in \Omega^k_(M,E)$ be two $E$-valued forms on $M$. Then we define an inner product as follows: choose a local frame $\{e_i:M\rightarrow E\}$ of $E$ and expand $\omega_{1,2}=\omega_{1,2}^i\otimes e_i$, and set:
		$$\braket{,}_{E}:\Omega^k(M,E)\times \Omega^k(M,E)\rightarrow C^\infty(M)$$
		$$\braket{\omega_1,\omega_2}=\braket{\omega_1^i,\omega_2^j}\braket{e_i,e_j}_E$$
	\end{Def}
	We now wish to generalize the notion of Hodge operator and dual exterior derivative to the framework of associated bundles.
	\begin{Def}
		Let $\omega\in\Omega^k(M,E)$ be a $E$-valued form, where $(E,M,\pi,\mathbb{R}^n)$ is a vector bundle. Choosing a local frame $\{e_i\}$, we define the $\star$ operator as:
		$$\star:\Omega^k(M,E)\rightarrow \Omega^{dim(M)-k}(M,E)$$
		$$\star\omega=\star\omega^i\otimes e_i$$ 
	\end{Def}
	Clearly, by linearity, this does not depend on the choice of the basis.\\
	\\
	By cosntruction, this generalized Hodge operator satisfies very similar properties to the one described in section \ref{Sec_1.7}. We can also define the adjoint of the exterior covariant derivative:
	\begin{Def}
		Let $D$ be the exterior covariant derivative on an associated bundle. Then we define the \textit{co-exterior covariant derivative} as:
		$$D^*=(-)^{k(n-k)}sgn(\braket{,}_E)\star\circ D\circ\star$$
		where $n$ is the dimension of the base manifold.
	\end{Def}
	\begin{Obs}
		In the case of an adjoint associated bundle on a compact semisimple algebra, the bundle metric on the associated bundle $E$ is positive definite so that the term $sgn(\braket{,}_E)$ is just 1.
	\end{Obs}
	\begin{Prop}
		The co-exterior derivative on an associated bundle $E$ satisfies:
		$$\braket{D^*\omega,\eta}_E=\braket{\omega,D\eta}_E$$
	\end{Prop}
	SHOW ANTIDERIVATION
	\chapter{The Yang-Mills Gauge theory}
	In this chapter we will use the knowledge exposed previously to construct the famous Yang-Mills Lagrangian and look at some of its properties. More informations on this topic can be found in REFERENZA
	\section{The Maxwell's equations}
	In this section we will see a geometric formulation for the Maxwell's equations. We will first of all analyze them in a flat space-time configuration, then proceed to look at them in curved space-time with a background non-flat metric. More references can be found in REFERENZA\\
	\\
	We will refer to $\mathbb{R}^n$ equipped with a flat metric $\eta=diag(-,-,...,_,+,+,...,+)$ with signature $(p,q)$, as $\mathbb{R}^{q,p}$. Our starting setting is Minkowski space, so we consider $\mathbb{R}^{1,3}$ as the base manifold. We are working in natural units, so that the speed of light is normalized to $c=1$.\\
	\\
	Given the coordinate system $\{t,x,y,z\}$, an orthonormal basis for the space-time $\mathbb{R}^{1,3}$, a basis for the tangent space is $\{\partial_t,\partial_x,\partial_y,\partial_z\}$. We have an induced ordered orthonormal basis of co-vectors for the cotangent space: $\{dt,dx,dy,dz\}$. We will use the orientation induced by this basis and define the volume form:
	$$\omega=dt\wedge dx\wedge dy\wedge dz$$
	This basis induces the following relations:
	\begin{multicols}{2}
		\begin{itemize}
			\item $\star dt= -dx\wedge dy\wedge dz$
			\item $\star dx= -dt\wedge dy\wedge dz$
			\item $\star dy= dt\wedge dx\wedge dz$
			\item $\star dz=-dt\wedge dx\wedge dy$
			\item $\star dt\wedge dx=-dy\wedge dz$
		\end{itemize}
		\columnbreak
		\begin{itemize}
			\item $\star dt\wedge dy=dx\wedge dz$
			\item $\star dt\wedge dz=-dx\wedge dy$
			\item $\star dx\wedge dy=dt\wedge dz$
			\item $\star dx\wedge dz=-dt\wedge dy$
			\item $\star dy\wedge dz=dt\wedge dx$
		\end{itemize} 
	\end{multicols}
	Which can be easily checked by applying proposition \ref{Prop_1.7.2}.
	Classically, in empty space, the Maxwell's equations are: (see REFERENZA)
	\begin{equation}\label{M.E.}
		\nabla \cdot \vec{B} =0 \hspace{20 pt} \nabla \times \vec{B}={\partial \vec{E}\over \partial t}\hspace{20 pt}\nabla \cdot \vec{E} =0 \hspace{20 pt} \nabla \times \vec{E}=-{\partial \vec{B}\over \partial t}
	\end{equation}
	Here $\vec{E}=(E_1,E_2,E_3)$ and $\vec{B}=(B_1,B_2,B_3)$ are obviously the electric and magnetic field.
	We introduce the following $2$-form:
	\begin{Def}
		We define the \textit{electromagnetic  field strength} as the following 2-form:
		$$F=E_1 dt\wedge dx+E_2 dt\wedge dy+E_3 dt\wedge dz-B_1 dy\wedge dz-B_2 dz\wedge dx-B_3 dx\wedge dy$$
	\end{Def}
	In terms of matrices, we can represent this as:
	$$F=\begin{pmatrix}
		0 && E_1 && E_2 && E_3\\
		-E_1 && 0 && -B_3 && B_2\\
		-E_2 &&  B_3 && 0 && -B1\\
		-E_3 && -B_2 && B_1 && 0\\ 
	\end{pmatrix}$$
	Since $F$ is a form, this matrix is skew-symmetric. Moreover, assuming that the fields $\vec{E},\vec{B}$ are smooth, $F$ is clearly smooth as well i.e. $F\in\Omega^2(\mathbb{R}^{1,3})$.
	\begin{Prop}
		The Maxwell's equations are given by:
		$$dF=0\hspace{20 pt} d^*F=0$$
	\end{Prop}
	\begin{proof}
		This is just a straightforward calculation. The fields are to be intended as functions like:
		$$\vec{E},\vec{B}:\mathbb{R}^{1,3}\rightarrow\mathbb{R}^3$$
		In particular, the components $E_i,B_i$ are functions of space-time. Their differentials are:
		$$dE_i=\partial_\mu E_idx^\mu\hspace{20 pt}dB_i=\partial_\mu B_idx^\mu$$
		where $dx^\mu$ indicates the basis elements $dx^0=dt,dx^1=dx,dx^2=dy,dx^3=dz$. Taking the exterior derivative of the field strength:
		$$dF=dE_1\wedge dt\wedge dx+dE_2\wedge dt\wedge dy+dE_3\wedge dt\wedge dz-$$
		$$dB_1\wedge dy\wedge dz-dB_2\wedge dz\wedge dx-dB_3\wedge dx\wedge dy$$
		Since the wedge product between two equal elements of the basis is 0, we obtain, after some algebra:
		$$dF=(\partial_z E_2-\partial_y E_3-\partial_t B_1)dt\wedge dy\wedge dz+(\partial_z E_1-\partial_x E_3-\partial_t B_2)dt\wedge dx\wedge dz+$$
		$$(\partial_y E_1-\partial_x E_2-\partial_t B_3)dt\wedge dx\wedge dy+ (\partial_x B_1+ \partial_y B_2+\partial_z B_3)dx\wedge dy\wedge dz$$
		The term $(\partial_x B_1+ \partial_y B_2+\partial_z B_3)$ is clearly $\nabla\cdot \vec{B}$, while the other 3 terms are obviously the components of:
		$$\partial_t\vec{B}+\nabla\times E$$
		By asking for $dF=0$ we recover the first and the fourth equation in \ref*{M.E.}.\\
		\\
		To recover the last two equations, we compute: 
		$$\star F=-B_1 dt\wedge dx-B_2dt\wedge dy-B_3 dt\wedge dz-E_1 dy\wedge dz-E_2 dz\wedge dx-E_3 dx\wedge dy$$
		Now, by evaluating $d(\star F)$ we get:
		$$d(\star F)=-dB_1\wedge dt\wedge dx-dB_2\wedge dt\wedge dy-dB_3\wedge dt\wedge dz-$$
		$$dE_1\wedge dy\wedge dz-dE_2\wedge dz\wedge dx-dE_3\wedge dx\wedge dy$$
		After a bit of algebra and re-arranging we find:
		$$d(\star F)=-\bigg[(\partial_z B_2-\partial_y B_3+\partial_t E_1)dt\wedge dy\wedge dz+
		(\partial_z B_1-\partial_x B_3-\partial_t E_2)dt\wedge dx\wedge dz
		$$
		$$+(\partial_y B_1-\partial_x B_2+\partial_t E_3)dt\wedge dx\wedge dy+(\partial_x E_1+\partial_y E_2+\partial_z E_3)dx\wedge dy\wedge dz\bigg]$$
		Clearly, by the same reasoning above, we recover the last 2 equations. This proves the initial claim.
	\end{proof}
	From the previous result we see that the electromagnetic field strength is a closed form. By the Poincaré lemma, \ref{P.L.} we can say that there exists a $1$-form $A\in\Omega^1(\mathbb{R}^{1,3})$, called \textit{vector potential}, such that $F=dA$. Clearly, this potential is not unique, but it is defined up to an additional term of the form $da$ where $a$ is a smooth function.\\
	The general form of $A$ and $F$ in the above basis is:
	$$F={1\over 2}F_{\mu\nu}dx^\mu\wedge dx^\nu\hspace{20 pt} A=A_\mu dx^\mu$$
	And now, taking the exterior derivative of $A$, we clearly get:
	$$dA=dA_\mu\wedge dx^\mu={1\over 2}\partial_\nu A_\mu dx^\nu\wedge dx^\mu$$
	From this we obtain:
	$$F_{\mu\nu}=\partial_\mu A_\nu-\partial_\nu A_\mu$$
	Looking at the Maxwell's equations, $dF=0$ is automatically satisfied. As for $d^*F=0$, we can explicitly compute this in the orthonormal basis:
	$$\star F=\star dA={1\over 2}\star(\partial_\mu A_\nu dx^\mu\wedge dx^\nu)=$$
	$$-(\partial_tA_x-\partial_x A_t)dy\wedge dz-(\partial_yA_t-\partial_t A_y)dx\wedge dz+$$
	$$-(\partial_t A_z-\partial_z A_t)dx\wedge dy+(\partial_x A_y-\partial_y A_x)dt\wedge dz+$$
	$$+(\partial_z A_x-\partial_x A_z)dt\wedge dy+(\partial_y A_z-\partial_z A_y)dt\wedge dx=$$
	$$=-F_{tx} dy\wedge dz- F_{yt}dx\wedge dz -F_{tz}dx\wedge dy+F_{xy}dt\wedge dz+F_{zx}dt\wedge dy+F_{yz}dt\wedge dx$$
	Applying now the exterior derivative we get:
	$$d\star F=d\star dA=$$
	$$=(-\partial_t F_{tz}+\partial_xF_{xz}+\partial_y F_{yz})dt\wedge dx\wedge dy+(\partial_t F_{ty}-\partial_x F_{xy}-\partial_z F_{zy})dt\wedge dx\wedge dz+$$
	$$+(-\partial_t F_{tx}+\partial_yF_{yx}+\partial_z F_{zx})dt\wedge dy\wedge dz +(\partial_xF_{xt}+\partial_yF_{yt}+\partial_zF_{xt}))dx\wedge dy\wedge dz$$
	Equating this to 0, it can be re-written as:
	$$\braket{dx^\mu,dx^\mu}\partial_\nu F_{\nu\mu}=0$$
	Or equivalently, in terms of the vector potential $A$:
	$$\braket{dx^\mu,dx^\mu}\partial_\nu (\partial_\mu A_\nu-\partial_\nu A_\mu)=0$$
	\section{The Maxwell's equations in curved space-time}
	\section{The Electromagnetic Lagrangian}
	In this section we will explicitly construct the Lagrangian of the electromagnetic field. More references on this topic can be found in REFERENZA
	\\\\
	The Lagrangian is the mathematical object which allows us to describe the laws of the physical system. The aim is to construct a Lorentz-invariant Lagrangian, from which we can recover our equations: $dF=d^*F=0$.
	\begin{Prop}\label{Prop_8.1.2}
		The lagrangian for the Maxwell's equations is the map $\mathcal{L}:M\rightarrow \mathbb{R}$ such that:
		$$\mathcal{L}\omega=-{1\over 2}F\wedge \star F$$
		where $\omega$ is the volume form.
	\end{Prop}
	\begin{proof}
		We explicitly compute in coordinates the lagrangian, by substituting the previous expansions for $F,\star F$:
		$$F\wedge \star F={1\over 4}F_{\mu\nu}F_{\alpha\beta}dx^\mu\wedge dx^\nu \wedge (\star dx^\alpha\wedge dx^\beta)=$$
		$$=(-F_{tx}F_{tx}+F_{ty}F_{yt}-F_{tz}F_{tz}+F_{xy}F_{xy}-F_{xz}F_{zx}+F_{yz}F_{yz})dx\wedge dx\wedge dy\wedge dz$$
		Moreover, we have that: $F^{\mu\nu}=\eta^{\mu\alpha}\eta^{\nu\beta}F_{\alpha\beta}$ which in turn implies the following identities:
			\begin{itemize}
				\item $F^{ti}=-F_{ti}$ where $i=x,y,z$;
				\item $F^{ij}=F_{ij}$
			\end{itemize}
		$$F\wedge \star F=F_{\mu\nu}F^{\mu\nu}dt\wedge dx\wedge dy\wedge dz$$
		This gives the general well known form of the electromagnetic lagrangian:
		$$\mathcal{L}=-{1\over 2}F_{\mu\nu}F^{\mu\nu}$$
	\end{proof}
	\section{The U(1) Yang-Mills theory}
	In this section we will use the Maxwell's equation to deduce the general form of the Yang-Mills lagrangian for electromagnetism. More references on this topic can be found in: REFERENZA
	\\\\
	The lagrangian is the mathematical object used to describe the dynamics of our fields. A priori there are infinitely many lagrangians that we could consider. However, we would like to set some restrictions to them. Namely, any lagrangian considered should have the following qualities:
	\begin{itemize}
		\item they should be invariant under some symmetries;
		\item they must be renormalizable.
	\end{itemize}
	We will not focus on renormalizability since it would require some knowledge which is not exposed in this work. As for the symmetries, we wish our lagrangian to be at least:
	\begin{itemize}
		\item Lorentz invariant;
		\item Gauge invariant.
	\end{itemize}
	By gauge invariance we mean that the lagrangian is invariant under local transformations encoded in a Lie group. In particular, we will assume the existence of an a-priori $G$-principal bundle on our space-time manifold, where $G$ is the symmetry group. In our analysis, we also ask for $G$ to be compact and to have a semisimple Lie algebra. This requirement will ensure, in the $n$-dimensional case, that the group is equipped with a positive definite scalar product: the Killing form.
	\\\\
	We now wish to recover the Maxwell's equations and the electromagnetic lagrangian of the previpus section in a new differntial geometry setting, which will make the geuge symmetry and the Lorentz invariance more manifest. We assume the existence of a principal $G$-bundle $(P,\mathbb{R}^{1,3},\pi,G)$ where $G$ is a Lie group. Given any representation $\rho:G\rightarrow GL(V)$, we can also define an associated bundle $(E,\mathbb{R}^{1,3},\pi_A,G)$ where $E={P\times V\over \sim}$ over the equivalence relation:
	$$(p,v)\sim(p\cdot g,\rho(g^{-1})v)$$
	Two points in the space $E$ that belong to the same equivalence class represent the same  physical state. To find an analogue for the vector potential, we take a connection 1-form $A$ on our principal bundle $A\in\Omega^1(P,\mathfrak{g})$.\\
	By definition, the connection $1$-form has the following properties:
	\begin{itemize}
		\item for every $X\in \mathfrak{g}$ and $p\in P$ it holds $A(\overline{X}_p)=X$ where $\overline{X}$ is the fundamental vector field of $X$.
		\item $A$ is smooth;
		\item for every $g\in G$ it holds $dr_gA=Ad(g^{-1})A$ is right equivariant.
	\end{itemize}
	\begin{Prop}
		The transformation rule of the connection $A$ under a gauge transformation induced by $U(1)$ is the same one of the vector potential in the electromagnetic setting.
	\end{Prop}
	\begin{proof}
		Fixing a local gauge $s:U\subset M\rightarrow P$, we can pull-back the connection 1 form on the manifold:
		$$A_{\mathbb{R}^{1,3}}=s^*A\in\Omega^1(M,\mathfrak{g})$$
		Thus, having chosen a basis for $\mathfrak{g}$, indexed with $\{T^a\}$, and a chart $(U,\phi)$ on $\mathbb{R}^{1,3}$ we can express the form as a linear combination:
		$$A_{\mathbb{R}^{1,3}}=A_{a,\mu} dx^\mu\otimes T^a=A_a\otimes T^a$$
		Here $a$ indicates the Lie algebra indices and $\mu$ the spatial indices. In the electromagnetic setting, we have only one vector potential. For this reason, to recover the Maxwell description, we set $\mathfrak{g}=\mathbb{R}$. There are many Lie groups $G$ with Lie algebra $\mathbb{R}$, however, we wish for $G$ to be compact. The only compact group with Lie algebra $\mathbb{R}$ is $U(1)$. For this reason, we set our principal bundle to be a $U(1)$ principal bundle. With this choice, we also recover the gauge symmetry of the vector potential. To see this, consider two local gauges on the same open set $s_{1,2}:U\rightarrow P$, then, by theorem \ref{Theo_7.2.1}, the pulled back $1$-forms are related by:
		$$A_1=Ad(g_{21}^{-1})A_2+\mu_{21}$$
		For a matrix Lie group, $g_{21}(x)$ is a matrix and so the Maurer-Cartan form is just a left matrix multiplication: $$\theta_g(h)=gh\hbox{ with $g,h$ matrices}$$ 
		Moreover, for $G=U(1)$ abelian, the adjoint representation is trivial and the pullback of the Maurer-Cartan form is:
		$$\mu_{21}=g_{21}^*\theta=g_{21}^{-1}\cdot dg_{21}$$ 
		where $dg_{21}$ is the differential of each component of the matrix $g_{21}$. Thus, the transformation rule for our connection is:
		$$A_1=A_2+g_{21}^{-1}\cdot dg_{21}$$
		In particular, for $x\in U\subset M$ we have: $g_{21}(x)=e^{T\alpha(x)}$ and so, taking as generator of the Lie algebra $T$:
		$$dg_{21,x}=e^{T\alpha(x)}\cdot \partial_\mu\alpha(x)dx^\mu\otimes T$$
		putting everything together, we get:
		$$A_1=A_{1,\mu}dx^\mu\otimes T=A_{2,\mu}dx^\mu\otimes T+\partial_\mu\alpha dx^\mu\otimes T$$
		Since in this case we have a 1-dimensional Lie algebra, choosing as our generator $T=-i$, we recover:
		$$A_1=-iA_{1,\mu}dx^\mu=-iA_{2,\mu}dx^\mu-i\partial_\mu\alpha dx^\mu$$
		Looking only at the manifold-form part of the above relation, one finds the same exact gauge transformation that our vector potential had:
		$$A_1=A_2+d\alpha$$
		where $\alpha$ is a generic function of the points.
	\end{proof}
	We have recovered an analogue for the vector potential as a connection $1$-form on a principal $U(1)$-bundle. The next step is to find the field strength tensor $F$ of the previous section.
	\begin{comment}
		\begin{Obs}
			Since the Lie algebra is of dimension 1 there is a clear isomorphism: let $T$ be a generator for $\mathbb{R}$, then
			$$A_1\otimes T\longrightarrow A_1$$
			is a $1$-form on the manifold.
		\end{Obs}
	\end{comment}
	Since $A$ is a connection, we can loot at its curvature. by definition, this is an $Ad$-tensorial $2$ form:
	$$F=dA+{1\over 2}[A,A]\in \Omega^2_{Ad}(P,\mathfrak{g})$$
	This object transforms under our gauge symmetry as found in observation \ref{Obs_7.2.2}: chosing a local gauge $s_i:U\rightarrow P$ we have
	$$s^*F=F_i\rightarrow F_j= Ad(g_{ji}^{-1})F$$
	Once again, in our current setting the symmetry group is abelian and so the adjoint representation is trivial:
	$$F_i\rightarrow F_i=F_j\hbox{ is invariant under gauge transformations}$$
	\begin{Obs}
		Since the pullback of $F$ does not change under changes of local gauges, by proposition \ref{Prop_7.2.2}, we know that this form is globally defined on $M$.
	\end{Obs}
	\begin{Prop}
		For G=$U(1)$, the first half of the Maxwell's equations holds like: $DF=dF=0$, $F=DA=dA$, where $D$ is the exterior covariant derivative with respect to the adjoint bundle and $d$ is the exterior derivative.
	\end{Prop}
	\begin{proof}
		The curvature is a type $Ad$ tensorial form $F\in \Omega^2_{Ad}(P,\mathfrak{g})$. In example \ref{Ex_6.3.1} we found that
		$$F=DA$$
		Moreover, in observation \ref{Obs_6.3.5} we proved $DF=0$. Recall that proposition \ref{Prop_6.3.4}, the covariant derivative of an $Ad$-tensorial $2$-form is given by:
		$$DF=dF+[A,F]$$
		where the product $[A,F]$ is given by definition \ref{Def_6.3.3}:
		$$[A,F](X,Y,Z)={1\over 2}\sum_\sigma sgn(\sigma)\hbox{ad}(A(X))(F(Y,Z))$$
		In the current case the gauge group is $U(1)$ abelian. This means that the adjoint representation is trivial and so the exterior covariant derivative $D$ simplifies to:
		$$F=DA=dA+{1\over 2}[A,A]=dA$$
		Also $DF=dF+[A,F]=0=d(dA)+[A,F]$ which implies $DF=dF=0$. From those identities we can say that, in the abelian setting, the Maxwell's equations are fully recovered in space-time in two different equivalent forms. In fact, chosing $T$ as a generator for $\mathfrak{g}=\mathbb{R}$ and $s$ as a local gauge, we can pullback both $A$ and $F$. Knowing from proposition \ref{Prop_3.5.3} that the pullback commutes both with the exterior derivative and the product of forms:
		$$s^*F=s^*dA=ds^*A=dA_\mu\otimes T={1\over 2}F_{\mu\nu} dx^\mu\wedge dx^\nu \otimes T$$
		$$s^*dF=0=ds^*F={1\over 2}dF_{\mu\nu}\wedge dx^\mu\wedge dx^\nu \otimes T=0$$
	\end{proof}
	We now wish to recover the electromagnetic lagrangian.
	\begin{Prop}
		The electromagnetic lagrangian is:
		$$\mathcal{L}=-{1\over 2}\braket{F_M,F_M}_{Ad}$$
		where $F_M$ is the induced curvature on $\Omega^2(M,Ad(P))$, $Ad(P)$ is the associated adjoint bundle constructed from the principal bundle $P$ and $\braket{,}_{Ad}$ is the bundle metric induced from the Killing form on $U(1)$.
	\end{Prop}
	\begin{proof}
		We check that this lagrangian reduces to the same one we had found in proposition \ref{Prop_8.1.2}. Starting from the curvature $F\in \Omega^2_{Ad}(P,\mathfrak{g})$, through theorem \ref{Mus_Iso}, we can construct a $2$-form $F_M\in \Omega^2(M,E)$. Choosing a local frame $\{e_i\}$, we can expand $F_M=F^i_{M}\otimes e_i$, where $F_M^i\in \Omega^2(M)$, and define the product: $$\braket{F_M,F_M}_{Ad}=\braket{F_M^i,F_M^j}\braket{e_i,e_j}_{Ad}$$
		In the last expansion $\braket{e_i(p),e_j(p)}_{Ad}=B_{\mathfrak{u}(1)}(e_i(p),e_j(p))$ is induced by the Killing form, which is a non degenerate, negative definitem symmetric product in our setting. This implies that locally we can choose an orthonormal frame $e_i$ for this produc. Let now $\omega$ be the volume form, then $\braket{F_M^i,F_M^j}\omega=F_M^i\wedge\star F_M^j$ and in our basis the Killinig form takes the form: $B_{\mathfrak{u}(1)}(e_i,e_j)=-\delta_{ij}$ at any $p$. Thus, we can write:
		$$\mathcal{L}\omega={1\over 2}F_M^i\wedge \star F_M^i$$
		and since the Lie algebra is $1$ dimensional, the index $i$ has only one possible value:
		$$\mathcal{L}\omega={1\over 2}F_M\wedge \star F_M$$
		This is exactly the same form of proposition \ref{Prop_8.1.2}. Now it only remains to show that the solution of this lagrangian satisfies the Maxwell's equations. \\\\
		The first half follows from the properties of $F$. In fact, we already proved $DF=dF+A\cdot F=0$ and through the construction of theorem \ref{Mus_Iso}, it immediatly follows that $DF_M=0$.\\
		\\
		The remaining part of the maxwell's equations can be found through the Euler-Lagrange equations. DA FARE
	\end{proof}
	\section{The SU(N) Yang-Mills theory}
	In this section we will generalize the construction of the electromagnetic lagrangian with higher dimensional groups. This will lead us to the QCD gauge boson lagrangian. More informations on this can be found in REFERENZA\\
	\\
	The idea is
	\chapter*{Bibliography}
	\begin{itemize}
		\item[$\circ$] [1] Loring W.Tu, An Introduction to Manifolds, Springer, 2011;
		\item[$\circ$] [2] Loring W.Tu, Differential Geometry, Springer, 2017;
		\item[$\circ$] [3] Mark J.D. Hamilton, Mathematical Gauge Theory, Springer, 2017;
		\item[$\circ$] [4] Czes Kosniowski, A First Course in Algebraic Topology, Cambridge, Cambridge University Press, 1980;
		\item[$\circ$] [5] V. S. Varadarajan, Lie Groups, Lie Algebras, and Their Representations, Springer, 1984;
		\item[$\circ$] [6] Theodor Bröcker , Tammo Dieck, Representations of Compact Lie Groups, Springer, 1985;
		\item[$\circ$] [7] Brian C. Hall, Lie Groups, Lie Algebras, and Representations, Springer, 2015;
		\item[$\circ$] [8] F. Warner, Foundations of Differentiable Manifolds and Lie Groups, Springer, New York, 1983;
		\item[$\circ$] [9] John M. Lee, Introduction to Smooth Manifolds, Springer, 2012;
	\end{itemize}
	
	\addcontentsline{toc}{chapter}{Bibliography}
\end{document}
